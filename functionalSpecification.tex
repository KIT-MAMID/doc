% das Papierformat zuerst
\documentclass[a4paper, 11pt]{article}
\usepackage[margin=3cm]{geometry}
\usepackage[utf8]{inputenc}
\usepackage[T1]{fontenc}
\usepackage{hyperref} % clickable refs
\usepackage{graphicx}
\usepackage[toc, numberedsection]{glossaries}
\usepackage{float}
\usepackage{amssymb}
\usepackage{url}

% TODO: template übersetzten

\makeglossaries

%Hack for referencing labels
\makeatletter
\def\namedlabel#1#2{\begingroup
    #2%
    \def\@currentlabel{#2}%
    \phantomsection\label{#1}\endgroup
}
\makeatother
% End: Hack for referencing labels

% Glossar: alle Einträge, aber ohne extra Referenzen
% http://tex.stackexchange.com/questions/115635/glossaries-suppress-pages-when-using-glsaddall
\newcommand*{\glsgobblenumber}[1]{}
\makeatletter
\newcommand*{\glsaddnp}[2][]{
  \glsdoifexists{#2}{
    \def\@glsnumberformat{glsgobblenumber}
    \edef\@gls@counter{\csname glo@#2@counter\endcsname}
    \setkeys{glossadd}{#1}
    \@gls@saveentrycounter
    \@do@wrglossary{#2}
  }
}
\renewcommand{\glsaddallunused}[1][]{
  \edef\@glo@type{\@glo@types}
  \setkeys{glossadd}{#1}
  \forallglsentries[\@glo@type]{\@glo@entry}{
    \ifglsused{\@glo@entry}{}{
      \glsaddnp[#1]{\@glo@entry}}}
}
\makeatother

\renewcommand{\glsnamefont}[1]{\mdseries #1} % glossary entries shouldn’t be bold

% Glossar

% So sieht ein Glossar-Eintrag aus:
%
%\newglossaryentry{dijkstra}{
%  name={Dijkstra’s Algorithmus},
%  description={ein Algorithmus, um den optimalen Pfad in einem gerichteten Graphen zu finden}
%}
%\newglossaryentry{arc}{
%  name={Arc-Flags},
%  description={eine Technik, um Routenberechnung zu beschleunigen},
%  see={dijkstra}
%}
%
% Und so kann er im Dokument verwendet werden:
%
% lorem ipsum dolor sit \gls{arc}, consectetur
%
% End: Glossar

% usage: \counteditem{prefix}{refName} -> item `/prefixXX/` with label `prefix:refName` (where XX is counted in increments of 10)
\makeatletter
\newcommand{\oitem}[2]{
  % define the counter
  \@ifundefined{c@oitem#1}{\newcounter{oitem#1}}{} % initialized at 0
  \addtocounter{oitem#1}{10}
  \item[\namedlabel{#1:#2}{/#1\arabic{oitem#1}/}]
}
\makeatother

% usage: \testfall{szenario}{ablauf}{ergebnis} oder \testfall[\ref{F:getesteteFunktion}]{szenario}{ablauf}{ergebnis}
\newcommand{\testfall}[4][]{
  \begin{description}
    \ifthenelse{\equal{#1}{}}
               {} % optional argument #1 is empty: skip
               {\item[Testet] #1}
    \item[Vorbedingungen] #2
    \item[Ablauf] #3
    \item[Erwartetes Ergebnis] #4
  \end{description}
}

\begin{document}

% place a symbol before clickable links
% this has to come *after* \begin{document} because hyperref installs a \AtBeginDocument hook that updates the ref command.
\newcommand{\refsymbol}[0]{\scalebox{0.5}{$\nearrow$}}
\let\oldref\ref
\renewcommand{\ref}[1]{\refsymbol\oldref{#1}}
\let\oldgls\gls
\renewcommand{\gls}[1]{\refsymbol\oldgls{#1}}
\let\oldGls\Gls
\renewcommand{\Gls}[1]{\refsymbol\oldGls{#1}}
\let\oldglspl\glspl
\renewcommand{\glspl}[1]{\refsymbol\oldglspl{#1}}
\let\oldGlspl\Glspl
\renewcommand{\Glspl}[1]{\refsymbol\oldGlspl{#1}}
\let\oldglslink\glslink
\renewcommand{\glslink}[2]{\refsymbol\oldglslink{#1}{#2}}
\let\oldhyperref\hyperref
\renewcommand{\hyperref}[2][notActuallyOptional]{\refsymbol\oldhyperref[#1]{#2}}
\let\oldautoref\autoref
\renewcommand{\autoref}[1]{\refsymbol\oldautoref{#1}}

\newcommand{\abbildung}[1]{\autoref{fig:#1}}
\newcommand{\mamid}[0]{\textit{KIT-MAMID} }

% alle Glossareintraege
\newacronym{gui}{GUI}{Graphical User Interface}
\newacronym{cli}{CLI}{Command Line Interface}
\newacronym{CRUD}{CRUD}{Create / Read / (Update|Modify) / Delete}
\newacronym{ICMP}{ICMP}{Internet Control Message Protocol}
\newacronym{API}{API}{Application Programming Interface}
\newacronym{JSON}{JSON}{JavaScript Object Notation}
\newacronym{HTTP}{HTTP}{HyperText Transfer Protocol}
\newacronym{LAN}{LAN}{Local Area Network}
\newacronym{WAN}{WAN}{Wide Area Network}

\newglossaryentry{cluster}{
	name={cluster},
	description={Aggregation of \glspl{host}},
	plural={clusters}
}
\newglossaryentry{MongoDB}{
	name={MongoDB},
	description={A NoSQL based database server},
	plural={MongoDB}
}
\newglossaryentry{replica set}{
	name={replica set},
	description={Multiple \gls{MongoDB} instances distributed over multiple \glspl{host} sharing a copy of the same data},
	plural={replica sets}
}
\newglossaryentry{sharding}{
	name={Sharding},
	description={A type of \gls{MongoDB} deployment where data sets are spread across multiple \glspl{host} or \glspl{replica set}. \\ More information: {\small \url{https://docs.mongodb.com/manual/sharding/}}}
}
\newglossaryentry{administrator}{
	name={administrator},
	description={The person responsible for managing the hardware \& software deployed on the \gls{cluster}. Uses \mamid for \gls{MongoDB} \gls{replica set} administration},
	plural={administrators}
}
\newglossaryentry{host}{
	name={host},
	description={Network node with a unique hostname},
	plural={hosts}
}
\newglossaryentry{slave}{
	name={slave},
	description={Program running on \gls{host} managing \gls{MongoDB} processes},
	plural={slaves}
}
\newglossaryentry{master}{
	name={master},
	description={Program responsible for managing \glspl{slave} and providing \acrshort{API} for \gls{cluster} management},
	plural={masters}
}
\newglossaryentry{inventory}{
	name={inventory},
	description={Persistently stored list of slaves and their state (see \ref{D:Inventory})},
	plural={inventories}
}
\newglossaryentry{active mode}{
	name={active mode},
	description={Marks the \gls{slave} as available to host \gls{MongoDB} processes that are be part of \glspl{replica set}.},
	plural={active modes}
}
\newglossaryentry{maintenance mode}{
	name={maintenance mode},
	description={Marks the \gls{slave} as unavailable \& inhibits reconfiguration through \gls{master}, but does not otherwise affect running \gls{MongoDB} instances on the \gls{host}.},
	plural={maintenance modes}
}
\newglossaryentry{disabled mode}{
	name={disabled mode},
	description={A \gls{slave} in this mode does not run any \gls{MongoDB} instances controlled by the \gls{master}, hence has no \gls{MongoDB} instances in any \gls{replica set}.},
	plural={disabled modes}
}
\newglossaryentry{risk group}{
	name={risk group},
	description={A set of \glspl{host} sharing a common risk of failure, e.g. a shared power supply. Modeled through sets of \glspl{slave} since a 1:1 relationship exists between hosts and slaves},
	plural={risk groups}
}
\newglossaryentry{persistent storage}{
	name={persistent storage},
	description={Storage capable of storing data between power outage},
	plural={persistent storages}
}
\newglossaryentry{volatile storage}{
	name={volatile storage},
	description={Storage incapable of storing data between process lifetime},
	plural={volatile storages}
}
\newglossaryentry{root data directory}{
	name={root data directory},
	description={Base directory holding all files of a \gls{slave} and its \gls{MongoDB} instances},
	plural={root data directories}
}
\newglossaryentry{arbiter}{
	name={arbiter},
	description={In charge of breaking ties on a \gls{replica set} election with even number of members. More information: {\small \url{https://docs.mongodb.com/manual/core/replica-set-elections/}}},
	plural={arbiters}
}
\newglossaryentry{degraded}{
	name={degraded},
	description={State of a \gls{replica set} with member in \gls{maintenance mode} or \gls{disabled mode} or otherwise reporting failure},
	plural={degraded}
}

\begin{titlepage}
\makeatletter
\begin{center}
~\\[4em]
{\Huge KIT-MAMID}\\[.8em]\huge{Monitor and Manager for In-Memory Databases}\\[2em]
{\huge Functional specification}\\[1em]
{\large\today}\\[2.5em]
{\LARGE
Niklas Fuhrberg\\
Anton Schirg\\
Christian Schwarz\\
Janis Streib\\
Bob Weinand\\[3em]}
supervised by\\[2em]
{\Large
Dr. Marek Szuba\\[1em]}
at\\[1em]
{\Large
Karlsruhe Institute of Technology\\
SCC}

\end{center}
\makeatother
\end{titlepage}
\newpage
\tableofcontents
\newpage

% -------------------------------------------------------------- HIER BEGINNT DAS DOKUMENT WIRKLICH ---------------------------------
\section{Introduction}
\mamid is a manager for database \glspl{cluster}, facilitating creation, administration and monitoring of \gls{MongoDB} \glspl{replica set}.

\mamid assists the \gls{administrator} during initial setup, continuous operation, maintenance cases and expansion of the cluster.

Each cluster \gls{host} runs a \gls{slave} instance of \mamid.
A single \gls{master} instance coordinates the \glspl{slave}' membership in the \glspl{replica set}.

\section{Zielbestimmung}
\subsection{Musskriterien}

\begin{description}

\oitem{MK}{} The \gls{administrator} interacts with \mamid through a web \acrshort{gui}.

% inventory ops
\oitem{MK}{inventory_definition} The \gls{master} maintains a list of \glspl{slave} called \gls{inventory}.
\oitem{MK}{} The \gls{administrator} can add \glspl{slave} to the \gls{inventory}.
\oitem{MK}{} The \acrshort{gui} visualizes the \gls{inventory}.
\oitem{MK}{} The \gls{administrator} can remove a \gls{slave} in \gls{disabled mode} from the \gls{inventory}.
\oitem{MK}{spec_physical_interdep} The \gls{administrator} can specify \glspl{physical interdependency} between \gls{inventory} \glspl{slave}.

% inventory ops -> slave
\oitem{MK}{available_slave_modes} The \gls{administrator} can set set a \gls{slave} in the \gls{inventory} to one of the following modes: \gls{active mode}, \gls{maintenance mode}, \gls{disabled mode}.
\oitem{MK}{available_slave_types} The \gls{administrator} can set a \gls{slave} as either \glslink{persistent storage}{persistent} \glslink{volatile storage}{volatile}.
\oitem{MK}{root_data_directory} The \gls{administrator} can specify a \gls{root data directory} per \gls{slave}.

% replica set ops
\oitem{MK}{replica_set_create} The \gls{administrator} can create a new \gls{replica set}.
\oitem{MK}{replica_set_config_profiles} The \gls{administrator} can --- on creation of a replica set (\ref{MK:replica_set_create}) --- specify that it can be used as a configuration server for a sharded-cluster \footnote{\url{https://docs.mongodb.com/manual/core/sharded-cluster-config-servers/\#replica-set-config-servers}}.
\oitem{MK}{replica_set_member_counts} The \gls{administrator} can select the counts $p,v \in $ of (\glslink{persistent storage}{\textbf{p}ersisent} / \glslink{volatile storage}{\textbf{v}olatile}) \gls{replica set} members, where $(p+v) > 2 \land p >= 1 \land (p+v) \text{ odd}$.
\oitem{MK}{} The \acrshort{gui} visualizes the list of configured \glspl{replica set}.
\oitem{MK}{} The \gls{administrator} can destroy a \gls{replica set}.
\oitem{MK}{deploy_replica_set_config} The \gls{master} deploys the \gls{replica set} configuration to the cluster.

% slaves
\oitem{MK}{} The \gls{slave} controls \gls{MongoDB} processes on the \gls{cluster} \glspl{host}.

% monitoring features
\oitem{MK}{detect_slave_unexpected_behavior} The \gls{master} detects when a \gls{slave} in the \gls{inventory} behaves unexpectedly.
\oitem{MK}{} The \gls{master} informs the \gls{administrator} by e-mail about problems in the cluster (\ref{MK:detect_slave_unexpected_behavior}).
\oitem{MK}{} The \acrshort{gui} visualizes \glspl{slave} behaving unexpectedly (\ref{MK:detect_slave_unexpected_behavior}).

\end{description}

\subsection{Wunschkriterien}
\begin{description}
	
% inventory
\oitem{WK}{manual_autodiscovery} The \gls{master} auto-discovers new \glspl{slave} on the \gls{administrator}'s request.
\oitem{WK}{continuous_autodiscovery} The \gls{master} continuously auto-discovers \glspl{slave}.
\oitem{WK}{} The \gls{master} monitors reachability of \gls{cluster} \glspl{host} via ICMP.

% slaves
\oitem{WK}{} The \gls{administrator} can specify arbitrary \gls{MongoDB} command line parameters per \gls{slave}.
\oitem{WK}{} The \gls{administrator} can specify a \gls{MongoDB} configuration file template.

% automatic repair
\oitem{WK}{} The \gls{master} automatically repairs \gls{degraded} \glspl{replica set} with \emph{suitable} \glslink{active mode}{active} \glspl{slave}.

% replica sets
\oitem{WK}{deploy_arbiters} The \gls{master} deploys \gls{MongoDB} \glspl{arbiter} for configured \glspl{replica set}, removing some restrictions in \ref{MK:replica_set_member_counts}.
\oitem{WK}{} The \gls{master} exposes machine metrics of the \glspl{slave}.
\oitem{WK}{} The \gls{master} exposes the \glspl{replica set}' replication status.

% other
\oitem{WK}{} The \gls{administrator} can interact with \mamid via a \acrshort{cli}.
\oitem{WK}{http_api} The \gls{administrator} can interact with \mamid via a stable, documented \gls{HTTP API}.
\end{description}

\subsection{Abgrenzungskriterien}
\begin{description}
\oitem{AK}{} The \gls{master} does not implement support for higher-layer \gls{MongoDB} features, e.g. \emph{Sharding}.
\oitem{AK}{} \mamid neither deploys the operating system nor other required software (such as \gls{MongoDB} binaries) to the \gls{cluster} \glspl{host}.
\end{description}

\section{Produkteinsatz}

\subsection{Anwendungsbereiche}
\begin{itemize}
\item foo
\end{itemize}

\subsection{Zielgruppe}
\begin{itemize}
\item foo
\end{itemize}

\subsection{Betriebsbedingungen}
\begin{itemize}
\item zu Hause (test)
\end{itemize}

\section{Produktumgebung}
%Kevin

\subsection{Software}\label{subsec:Software}

\begin{itemize}
\item Operating system: Solaris (Indiana OS), Linux
\end{itemize}

\subsection{Hardware}

\begin{itemize}
\item foo
\end{itemize}
\section{System model}
\includegraphics[width=\textwidth]{module_overview}
\begin{description}
	\oitem{SM}{masterapiserver} Master::APIServer\\
	Provides an \acrshort{API} for \gls{cluster} status reporting and administrative activity. %TODO functional refs
 	\oitem{SM}{} Master::Controller\\
	Controls flow of events between different \gls{master} submodules.
	Updates the inventory database (\ref{SM:inventory}) as state change is recognized by the monitor.
	\oitem{SM}{inventory} Master::Inventory\\
	Database containing
	\begin{itemize}
		\item list of \glspl{slave} and associated state
		\item list of configured \glspl{replica set}.
	\end{itemize}
	\oitem{SM}{} Master::ClusterAllocator\\ %TODO funct ref
	It lays out the \gls{cluster}, i.e. decides the distribution of mongod instances onto cluster hosts.\\
	It respects constraints as described in functional specifications X, Y and Z.\\ %TODO fill refs
	It communicates with the \glspl{slave} using \ref{SM:masterslaveproto} to enforce the \gls{cluster} layout.
	\oitem{SM}{} Master::Monitor\\
	Is responsible for observing whether \glspl{slave} are still alive using \ref{SM:masterslaveproto}.%TODO funct ref: alive
	\oitem{SM}{masterslaveproto} MasterSlaveProtocol\\
	Is responsible for communication between \gls{master} and \gls{slave}.
	\oitem{SM}{} Slave::Controller\\
	Dispatches instructions to the \glspl{MongoDB} client and 
	the process manager. It notifies via \ref{SM:masterslaveproto} whether 
	operations failed or succeeded as well as not locally recoverable 
	failures.
	\oitem{SM}{mongodbclient} Slave::MongoDBClient\\
	Implements communication with \gls{MongoDB} processes.
	\oitem{SM}{} Slave::ProcessManager\\
	Spawns and controls \gls{MongoDB} processes.
	%Angular und so...
	\oitem{SM}{gui_model} GUI::Model\\
	Data structures representing \mamid entities that are displayed or managed through the web interface (\ref{SM:gui_view}).
	\oitem{SM}{gui_view} GUI::View\\
	Web interface and its UI elements.
	\oitem{SM}{} GUI::Controller\\
	Handles UI intercation, updates model and asserts semantic consistency between \ref{SM:gui_view} and \ref{SM:gui_model}.
	\oitem{SM}{notif_mail} NotificationManager::MailNotifier\\
	Sends e-mail messages.
	\oitem{SM}{notif_apiclient} NotificationManager::APIClient\\
	\acrshort{API} client for \ref{SM:masterapiserver}.
	\oitem{SM}{} NotificationManager::Controller\\
	Uses \ref{SM:notif_apiclient} and \ref{SM:notif_mail} to implement notification of the \gls{administrator}.
\end{description}

\section{Product Data}
\subsection{Inventory (contains slaves)}
% hostname(PRIMARY KEY) | slaveport | mongod-portrange | persistent/volatile | root data directory | mode (active, maint, disabled)

List of tuples containing the following data

\begin{description}
	\oitem{D}{} hostname
	\oitem{D}{} slave port $\in \mathbb{N}$
	\oitem{D}{} mongod-portrange $\in \{{[i, j]} \mid i,j \in \mathbb{N}\}$
	\oitem{D}{} persistency $\in \{persistent, volatile\}$
	\oitem{D}{} root data directory
	\oitem{D}{} mode $\in \{\text{active}, \text{maintenance}, \text{disabled}\}$
\end{description}

\subsection{Replica Set Configuration}
%----------------------------- ADMIN SETTABLE ------------------------------------------------------------  ---Allocator settable--  
% replset | p = num of persistent slaves | v = num of volatile slaves | sharding config server (true|false) | p+v mongod processes 
List of tuples containing the following data

\begin{description}
	\oitem{D}{} replica set name
	\oitem{D}{} number of persistent slaves $p$
	\oitem{D}{} number of volatile slaves $v$
	\oitem{D}{} sharding configuration server $\in {\text{true}, \text{false}}$
	\oitem{D}{} member mongod processes (list of $p+v$)
\end{description}

\subsection{MongoD processes}
%(hostname | port != slaveport) PRIMARY KEY
List of tuples containing the following data

\begin{description}
	\oitem{D}{} hostname
	\oitem{D}{} port
\end{description}

\subsection{Physical Interdependencies}
%sets of slaveids
%Each set marks the contained nodes as physically dependent.
%Disjoint sets!
\begin{description}
	\oitem{D}{} Mutually disjoint sets of \glspl{slave}, denoting which \glspl{slave} must not be member of the same \gls{replica set}.
\end{description}

\section{Funktionale Anforderungen}
% Format: Substantiviertes Verb am Anfang („Bestimmen von X“, nicht „Bestimmung von X“, „Bestimme X“ oder „X bestimmen”).

%TODO cli paramters + config file template wf spezifizieren

\subsection{\acrfull{gui}}
The \acrshort{gui} acts as a frontend to the functionality provided by \ref{SM:masterapiserver}. Hence, all functionality described in this subsection is realized through \acrshort{API} calls to the \gls{master}.
\begin{description}
	\oitem{F}{gui_crud_inventory} \acrshort{CRUD} \glspl{slave} in (\gls{inventory}).
	\oitem{F}{} Specify type of slave (\ref{MK:available_slave_types}) on insertion into inventory.
	\oitem{F}{} Specify root data directory of slave (\ref{MK:root_data_directory}).
	\oitem{F}{} \acrshort{CRUD} physical interdependencies (\ref{MK:spec_physical_interdep}) between \gls{host}.
	\oitem{F}{} Set mode of host (\ref{MK:available_slave_modes}).
	\oitem{F}{} \acrshort{CRUD} \gls{replica set} configurations (\ref{MK:replica_set_create}).
	\oitem{F}{} Display error reports (\ref{F:master_api_error_reports}).
	\oitem{WF}{} Display machine metrics of the \glspl{slave}. 
	\oitem{WF}{} Display replication status of the configured \glspl{replica set}.
	\oitem{WF}{} Provide interface to select auto-discovered slaves when adding to inventory (\ref{F:gui_crud_inventory}).
\end{description}

\subsection{Master}
\begin{description}
	% API Server
	\oitem{F}{} Provide \acrshort{API} to \acrshort{CRUD} slaves in \gls{inventory} (\ref{MK:inventory_definition}).
	\oitem{F}{} Provide \acrshort{API} to \acrshort{CRUD} physical interdependencies between \glspl{host}.
	\oitem{F}{} Provide \acrshort{API} to set \gls{cluster} host mode (as defined in \ref{MK:available_slave_modes}).
	\oitem{F}{} Provide \acrshort{API} to \acrshort{CRUD} replica set configurations (\ref{MK:replica_set_create}, \ref{MK:replica_set_config_profiles}, \ref{MK:replica_set_member_counts}).
	\oitem{F}{master_api_error_reports} Provide \acrshort{API} to retrieve error reports (\ref{MK:detect_slave_unexpected_behavior}).
	\oitem{WF}{} Provide \acrshort{API} to retrieve a list of auto-discovered \glspl{slave} (\ref{WK:manual_autodiscovery}).
	\oitem{WF}{} Provide \acrshort{API} exposing machine metrics of the \glspl{slave}.
	\oitem{WF}{} Provide \acrshort{API} exposing replica set replication status of the configured \glspl{replica set}.
	% Monitor
	\oitem{F}{} Monitor configuration \& state of \gls{MongoDB} instances running on \gls{cluster} hosts (\ref{MK:detect_slave_unexpected_behavior}).
	\oitem{WF}{} Monitor reachability of \gls{cluster} hosts via \acrshort{ICMP}.
	% Inventory
	\oitem{F}{} Persist and provide the cluster configuration \& state (\gls{inventory}).
	% Controller
	\oitem{WF}{} Continuously auto-discover \glspl{slave} on the cluster network and add them to the \gls{inventory} (\ref{WK:continuous_autodiscovery}).
	\oitem{WF}{} Auto-repair \gls{degraded} \glspl{replica set} utilizing \ref{F:layout_cluster_config}.
	% Cluster Allocator
	\oitem{F}{layout_cluster_config} Lay out the cluster configuration, i.e. decide on a \gls{replica set} configuration
	\begin{description}
		\oitem{F}{} respecting physical interdependencies between \gls{cluster} \glspl{host} (\ref{MK:spec_physical_interdep}).
		\oitem{F}{} respecting constraints given by administrator regarding number of persistent \& volatile replica set members per cluster host (\ref{MK:replica_set_member_counts} and \ref{MK:available_slave_types}).
		\oitem{F}{} respecting the mode of hosts (\ref{MK:available_slave_modes}).
		\oitem{WF}{} adding \gls{MongoDB} \glspl{arbiter} where necessary (\ref{WK:deploy_arbiters}).
	\end{description}
	\oitem{F}{} Communicate the \gls{MongoDB} instance configuration to the \glspl{slave} (\ref{MK:deploy_replica_set_config}).
\end{description}

\subsection{MasterSlaveProtocol}
\begin{description}
	\oitem{F}{} Transport \gls{replica set} configuration description of \gls{cluster} \glspl{host}.
	\oitem{F}{} Transport reachability check messages.
	\oitem{WF}{} Transport machine metrics of \glspl{slave}.
	\oitem{WF}{} Transport \gls{replica set} replication status of \gls{MongoDB} processes on \glspl{slave}.
	% Wunschkriterien erweiteretes Monitoring
\end{description}

\subsection{Slave}
\begin{description}
	% Process manager
	\oitem{F}{} Spawn / Control / Kill \gls{MongoDB} processes.
	% MongoDB Client
	\oitem{F}{} Communicate with \gls{MongoDB} processes using \ref{SM:mongodbclient}.
	% Controller
	\oitem{F}{} Enforce configuration description received from \gls{master} through \ref{SM:masterslaveproto} (also: \ref{MK:deploy_replica_set_config}). 
	\oitem{F}{} Report current configuration through \ref{SM:masterslaveproto}.
	\oitem{WF}{} Report machine metrics through \ref{SM:masterslaveproto}.
	\oitem{WF}{} Report \gls{replica set} replication status through \ref{SM:masterslaveproto}.
\end{description}

\subsection{NotificationManager}
\begin{description}
	% API Client
	\oitem{F}{} Consume error reporting \acrshort{API} of \gls{master} \ref{F:master_api_error_reports}.
	% Controller
	\oitem{F}{} Send error reports from \ref{F:master_api_error_reports} to configured list of contacts.
	\oitem{F}{} Read configured list of contacts from a text-based configuration file.
	% Mail Notifier
	\oitem{F}{} Send error reports via \glslink{email}{e-mail messages} to a contact's email address.
\end{description}

\subsection{\acrlong{cli}}
The \acrshort{cli} is an optional functional requirement. It acts as a frontend to the functionality provided by \ref{SM:masterapiserver}. Hence, all functionality described in this subsection is realized through \acrshort{API} calls to the \gls{master}.
\begin{description}
	\oitem{WF}{} \acrshort{CRUD} \glspl{slave} in (\gls{inventory}).
	\oitem{WF}{} Specify type of slave (\ref{MK:available_slave_types}) on insertion into inventory.
	\oitem{WF}{} Specify root data directory of slave (\ref{MK:root_data_directory}).
	\oitem{WF}{} \acrshort{CRUD} physical interdependencies (\ref{MK:spec_physical_interdep}) between \gls{host}.
	\oitem{WF}{} Set mode of host (\ref{MK:available_slave_modes}).
	\oitem{WF}{} \acrshort{CRUD} \gls{replica set} configurations (\ref{MK:replica_set_create}).
	\oitem{WF}{} Display error reports (\ref{F:master_api_error_reports}).
	\oitem{WF}{} Display machine metrics of the \glspl{slave}. 
	\oitem{WF}{} Display replication status of the configured \glspl{replica set}.
\end{description}

\section{Nichtfunktionale Anforderungen und Qualitätsziele}

% http_api

\begin{description}
\oitem{NF}{ntest} test
\end{description}

\section{Userinterface}
%Form:
%Name [: Beschreibung]
%Beschreibung: satz ohne subjekt, also klein und ohne Punkt; Strichpunkt getrennt
\subsection{Hauptfenster}

%\begin{enumerate}
%\item
%\end{enumerate}

%text...

\section{Globale Testfälle und Szenarien}
% Vorbedinung, Aktionen, Nachbedingung 
\subsection{Funktionssequenzen}
Bei allen Testfällen gilt als Vorbedingung, dass \mamid gestartet ist (außer es ist explizit gefordert, dass es gestoppt ist).
\begin{description}
\itemsep 1em
\subsubsection{Kernfunktionen}
\oitem{TF}{testen} a testcase.
\testfall[\ref{WK:beschlArc}] %testet:
    {foo} %vorbedingung
    {bar} % ablauf
    {jon} % erwa. ergebnis
\end{description}
\subsection{Testszenarien}
% all. vorbed.
\begin{description}
\oitem{TS}{test} blah
\begin{itemize}
\item ...
\end{itemize}
\end{description}

\subsection{Datenkonsistenzen}

\begin{description}
\oitem{TD}{blah} blah
\end{description}

\section{Entwicklungsumgebung}
\begin{description}
\item[Team communication] Slack
\item[Dokumentation] \LaTeX{}
\item[UML-Planungswerkzeug] UMLet
\item[IDE] ...
\item[Qualitätssicherung] ...
\item[Versionskontrolle] Git
\end{description}
\newpage
\glsaddallunused
\makeatletter
\newglossarystyle{myAltlist}{
  \glossarystyle{altlist} % base this style on altlist
  \renewcommand*{\glossaryentryfield}[5]{
  \item[\glsentryitem{##1}\glstarget{##1}{##2}]
    \mbox{}\par\nobreak\@afterheading
    ##3\glspostdescription\space On page ##5.
  }
}
\makeatother
\glsaddall
\printglossary[type=main, title={Glossary}, toctitle={Glossary}, style=myAltlist]

\end{document}
