\section{Introduction}

This document describes the design of the Application \mamid (\emph{Monitor and Manager for In-memory Databases}).

It thorougly explains the architecture of a \mamid deployment, the distribution of responsibilites among the various components,
individual Go structures, their methods and the relationships among them.

While the textual documentation delivers detailed information on individual components, the various UML diagrams
--- and in particular the class diagram --- clarify the interoperation between components and the architecture of \mamid.

In order to facilitate lecture of this document, the reader should be familiar with the contents of the \emph{Functional Specification}
document.\\
In particular, the following sections of the \emph{Functional Specification} are crucial for proper understanding of the
application design:
\begin{itemize}
  \item the technical terms introduced listed in the \emph{Glossary}
  \item the naming of \mamid components as described in the \emph{System Model}.
\end{itemize}

\section{Deviation from the Functional Specification}

%TODO

gekuerzte wunschkriterien? was koennen wir sicher nicht mit diesem design? was verschieben wir auf die implementierung?

(all components are designed with extensibility in mind. for many optional criteria, it is mostly a question of time constraints during the implementation phase, not a question of whether they are feasible or not)

\section{Overview}

\mamid is a manager for database clusters, facilitating creation, administration and monitoring of a MongoDB Replica Set deployment.

Explicit support for volatile storage on primary Replica Set members with lower-prioritized secondaries on persistent
storage is a key differentiator of \mamid.

As the manager of such distributed systems with high-availibility requirements, \mamid faces a series of non-trivial problems:
\begin{itemize}
  \item unreliable hardware
  \item unreliable communication between nodes
  \item resilience against unexpected failures of the above 
  \item finding a deployment layout that fulfills a set of availibility requirements
  \item monitoring \& state tracking of the deployed MongoDB processes
  \item abstraction of the above complexity from the administrator
  %TODO more?
\end{itemize}

In order to achieve these goals, a distributed architecture is necessary.

\mamid implements several well-established design patterns, facilitating understanding of design and implementation and improving
testability of individual components.

As specified in the \emph{Functional Specification}, the language of choice for all components but the GUI is \emph{Golang}.
While \emph{Golang} is an object-oriented language, it differs from languages like \emph{Java} in certain critical approaches:
\begin{itemize}
  \item no generics
  \item no classes, only \codeinline{struct} with \emph{method sets}
  \item this is most notable in the lack of \emph{Constructors} and \emph{inheritance}
  \item however, inhertance can be simulated to a certain degree through \emph{embedding} %TODO ref go spec
\end{itemize}

Overall, this leads to a situation where many traditional design patterns do not apply as well (as e.g. to Java) and hence need
to be adopted.

Given these constraints, \mamid still employs a variety of design patterns across its different components:

% GUI Design pattersn
The architecture of the \refgo{gui} follows the \textbf{Model-View-Controller} pattern.
% is this actually MVC? @janis should elaborate on this.
% TODO explain in short the idea of MVC proposed by Angular. It should be written down somewhere -> quote it

% Master Design Patterns
The \textbf{Client/server architecture pattern} is used to decouple the \refgo{gui} functionality from the \refgo{master}:
The \refgo{gui} acts as a frontend to the functionality provided by the \refgo{master} HTTP API (\refgo{masterapi}).

Being the most complex package of the project, several decoupling patterns are employed in the \refgo{master}:

The \textbf{repository pattern} is the most one employed inside the \refgo{master}:
The external \refgo{gorm} database abstraction layer holds several tables of the structures in \refgo{model}.\\
\refgo{Monitor}, \refgo{ProblemManager}, \refgo{Deployer} and \refgo{ClusterAllocator} are loosely coupled and communicate mostly by
modifying the database.\\
Furthermore, a variation of the \textbf{publish/subsribe pattern} is implemented by the \refgo{master.Bus}.

% MasterSlave Protocol
The \textbf{remote proxy} decoupling pattern is employed in the {Master Slave Protocol (\refgo{msp}) package:
it implements communication bewteen \refgo{master} and \refgo{slave}.
\refgo{msp.MSPClient} exposes RPC stubs that call the the \refgo{msp.MSPConsumer} implemented on the \refgo{slave}.
Another term used to describe this relationship between master and slaves is the \textbf{Multi-Server/Single-Client architecture}.
%TODO or is it peer2peer? check SWT slides

% Slave design patterns
\textbf{Dependency injection} is employed extensively in both \refgo{master} and \refgo{slave} to increase testability through mocks.
A good example for this are the references to \codeinline{gorm.DB} passed into \refgo{master.Monitor}, \refgo{ProblemManager},
\refgo{Deployer} and \refgo{ClusterAllocator}.

% Notification Manager patterns
The \textbf{strategy pattern} is used inside the \refgo{notifier} to encapsulate the concrete notification delivery
to per communication channel (\refgo{notifier.Notifier}, increasing extensibility for different communication channels in the future.
% TODO actually a strategy or more of a template method?

%TODO more design patterns
