\section{Notifier}
\beginpackage{notifier}

\begin{figure}[H]
	\includegraphics[width=\textwidth]{notifier_classes}
	\caption{Notifier}
\end{figure}

The \refgo{notifier} polls the \refgo{masterapi} for problems (see Section \ref{masterapi:problems}).

It notifies a predefined list of contacts whenever a new \reftype{Problem} occurs.

The implementation is generic enough to support a variety of different communication channels.
However, in the basic implementation, only email notifications are supported.

\struct{Controller}{
        The \reftype{Controller} coordinates the flow of events and information inside the \refgo{notifier}.

        It leverages a command line flags parser, the contacts file \reftype{Parser}, \reftype{APIClient} and \reftype{Notifier}
        to realize the required functionality.
}{
	\property{lastProblems}[[]\reftype{Problem}]{
		List of recent problems.
	}
        \property{notifiers}[[]\reftype{Notifier}]{
                List of fully initialized \reftype{Notifier} which are called when a new \reftype{Problem} arises.
        }
}{
	\method{main}
	{
		A Go application starts by initializing the Go 'main' package and then running the \textit{main()} function.

                The \reftype{Notifier} works in a loop:
		\begin{itemize}
                        \item Receive a list of \reftype{Problem} through the \reftype{APIClient}.
                        \item Call \refmethod{diffProblems} to determine which new \codeinline{Problems} occured
                                to prevent repeated alerts on the same problem only a new \reftype{Problem} will be sent.
			\item Call \refmethod{notify} to handle delivery of notifications via a \reftype{Notifier}.
		\end{itemize}
	}
	\method{diffProblems}[[]Problem](
		\param{received}[[]\reftype{Problem}]{List of recently sent problems}
		){
		Compares \refproperty{lastProblems} to the passed array of problems and returns the new ones.
	}
	\method{notify}(
		\param{problem}[\reftype{Problem}]{Problem currenty staged for notification}
	){
		Handles delivery of notifications via a \reftype{Notifier}.
	}
}

\struct{Parser}{
	Used by the \reftype{Controller} to parse the contacts file.
}{
  \item \textit{none}
}{
	\method{Parse}[[]\reftype{Contact}, error](
		\param{path}[string]{Path to the contacts file}
		){
			Parses the contacts file.
			Returns an array of contacts.
		}
	\method{ParseConfig}[\reftype{SMTPRelay}, string, string, error](
		\param{path}[string]{Path to the config file}
		){
			Parses the config file.
			Returns the SMTPRelay, apiHost and contactsFile.
		}}

\struct{APIClient}{
        Receives new array of \refgo{notifier.Problem} from the \refgo{masterapi}.
}{
  \item \textit{none}
}{
	\method{Receive}[[]\reftype{Problem}](
		\param{host}[string]{
			Hostname of the API server.
		}
	){
	  Uses \refgoalt{masterapi.problems.getAll}{[GET] /problems} to get all the problems and returns them. 
	}
}

\struct{Problem}{		
     A Problem object as specified in section \ref{masterapi:problems} of \refgo{masterapi}.
}{
	 \property{Id}[uint]{Unique ID of the problem generated and used by the ORM layer (primary \& foreign key)
                             and Master API (as simple identifier).\\
                             If the \emph{same} problem occurrs after it was resolved before,
                             its different incarnations are distinguishable by their \refproperty{Id}.}
	 \property{Description}[string]{Short human-readable description of the problem, suitable for subject lines of error messages.}
	 \property{LongDescription}[string]{Long human-readable description of the problem.}
	 \property{FirstOccurred}[time.Time]{The point in time when the particular problem first occurred.}
	 \property{LastUpdated}[time.Time]{The point in time when the particular problem was last observed, e.g. the time of the last monitor run.}
	 \property{Slave}[uint]{The identifier of the slave. May be \codeinline{0} if not tied to a particular slave.}
	 \property{ReplicaSet}[uint]{The identifier of the Replica Set. May be \codeinline{0} if not tied to a particular Replica Set.}
}

\interface{Notifier}{
        Abstract notification delivery interface.

        Any concrete communication mechanism supported by the \refgo{notifier} implements the \reftype{Notifier} interface.
	}{
	\method{Send}(
		\param{problem}[\reftype{Problem}]
                ){Deliver a notification about \refparam{problem} using the concrete communication channel supported by the implementation.}
}

\struct{EmailNotifier}{
        Implements \reftype{Notifier} to send emails via SMTP to a list of \reftype{EmailContact}.
}{
        \property{Contacts}[[]\reftype{EmailContact}]{The list of contacts to be notified via email.}
        \property{Relay}[\reftype{SMTPRelay}]{The SMTP relay through which notification emails are delivered. See \reftype{SMTPRelay} for details.}
}{
	\method{Send}(
		\param{problem}[\reftype{Problem}]{The \reftype{Problem} subject to the notification}
		){
                        Generates an email message containing information encapsulated in \refparam{problem} and delivers it using \refmethod{sendMailToContacts}.
		}
	\method{sendMailToContacts}[](
		\param{msg}[string]{}
		){
			Sends the messages created by \refmethod{Send} to a list of email contacts.
                        Uses the Go \codeinline{net/smtp} package.
		}
}

\interface{Contact}{
	Interface for contact data of the recipents of problem notifications.\newline
	Contact data is read from the contacts file.
}

\struct{EmailContact}{
	Implements \reftype{Contact}.
}{
	\property{Address}[string]{Email address of one contact}
}
\struct{SMTPRelay}{
        Credentials required for delivery of emails through an SMTP server reachable at \refproperty{Hostname}.

        The role of this SMTP server is commonly referred to as \emph{relay} or \emph{smarthost}.
}{
	\property{Hostname}[string]{Hostname of the SMTP Relay}
	\property{MailFrom}[string]{The email address from which the notifications will be sent}
}
