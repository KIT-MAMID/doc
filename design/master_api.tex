
% Args: Method, URI, 
\newcommand{\apicall}[8][]{
\begin{minipageparskip}
	\paragraph{\framebox{\uppercase{#1}} #2 - #3}
	\begin{description}[leftmargin=!,labelwidth=\widthof{\bfseries Return Codesaa}] %aa intentional -  more distance
		\item[Parameters]
		\if\relax\detokenize{#4}\relax
		\emph{None}
		\else
		\begin{tabularx}{\linewidth}{c|c|*1{>{\centering\arraybackslash}X}@{}}
			\textbf{Parameter} & \textbf{Data Type} & \textbf{Description}\\
			\hline
			#4
		\end{tabularx}
		\fi
		
		\item[Returns]
		\begin{description}[leftmargin=!,labelwidth=\widthof{\bfseries Status Codeaa}] %aa intentional -  more distance
			\item[Status Code] #5
			\ifthenelse{\equal{#6}{}}
			{} % optional argument #1 is empty: skip
			{\item[Headers] \begin{tabularx}{\linewidth}{c|*1{>{\centering\arraybackslash}X}@{}}
					\textbf{HTTP Header} & \textbf{Content}\\
					\hline
					#6
			\end{tabularx}}
			\ifthenelse{\equal{#7}{}}
			{\item[Body] \emph{Empty}} % optional argument #1 is empty: skip
			{\item[Body] #7}
		\end{description}
		
		\item[Error Codes]
		\begin{tabularx}{\linewidth}{c|*1{>{\centering\arraybackslash}X}@{}}
			\textbf{HTTP Status Code} & \textbf{Reason}\\
			\hline%
\ifthenelse{\equal{#1}{put}}{400 & Invalid parameters in object}{}%
\ifthenelse{\equal{#1}{get}}{404 & Object does not exist}{}%
\ifthenelse{\equal{#1}{post}}{404 & Object does not exist\\400 & Invalid parameters in object or parameter may not be changed}{}%
\ifthenelse{\equal{#1}{delete}}{404 & Object does not exist}{}%
\if\relax\detokenize{#8}\relax
\else
\\#8
\fi
		\end{tabularx}
	\end{description}
\end{minipageparskip}
\vspace{1em}
}

\section{Master API}
\beginpackage{masterapi}
%TODO General api description
The Master API (\refgo{masterapi}) is used by the GUI to CRUD Slaves, Replica Sets, Risk Groups and display Problems.

It is also used by the Notifier to get the current problems.

The reader should recognize that --- aside from \reftype{MasterAPIServer} --- this document does not describe the interface
of the structs of the package.

There exists a 1:1 mapping between
\begin{itemize}
  \item \hyperref[masterapi:apiobjects]{API Objects} and \codeinline{structs} in \refgo{masterapi}
  \item \hyperref[masterapi:apimethods]{API Methods} and \codeinline{methods} of \codeinline{structs} in \refgo{masterapi}
\end{itemize}

Hence, a definition of the HTTP REST Interface is
\begin{itemize}
  \item sufficient to specify the behavior of the \refgo{masterapi} package
  \item closer to the actual usage by other components of \mamid
\end{itemize}

%Uses RESTful HTTP
%Generic error responses
\subsection{API Objects} \label{masterapi:apiobjects}
API Objects are exchanged in JSON format.
\subsubsection{Slave}
\begin{lstlisting}
{
	"id": <int>,
	"hostname": <string>,
	"slave_port": <int>,
	"mongod_port_range_begin": <int (inclusive)>,
	"mongod_port_range_end": <int (exclusive)>,
	"persistent_storage": <bool>,
	"root data directory": <string>,
	"desired_state": <{active, maintenance, disabled}>
	"observed_and_desired_state_mismatch": <bool>
}
\end{lstlisting}
\subsubsection{Replica Set}
\begin{lstlisting}
{
	"id": <int>,
	"name": <string>,
	"persistent_node_count": <int>,
	"volatile_node_count": <int>,
	"configure_as_sharding_config_server": <bool>
}
\end{lstlisting}
\subsubsection{Risk Group}
\begin{lstlisting}
{
	"id": <int>,
	"name": <string>
}
\end{lstlisting}
\subsubsection{Problems}\label{masterapi:problems}
\begin{lstlisting}
{
	"id": <int>,
	"description": <string>,
	"long_description": <string>,
	"first_occurred": 
		<Combined date and time in UTC according to ISO 8601>
	"last_updated": 
		<Combined date and time in UTC according to ISO 8601>
	"slave_id": <int>,
	"replica_set_id": <int>
}
\end{lstlisting}
\subsection{API Methods} \label{masterapi:apimethods}
All requests and responses are made with the \emph{Content-Type}-Header set to \emph{application/json}. Requests containing any other content type are honored with the HTTP-Status-Code \emph{406 - Not Acceptable}.
\subsubsection{Slaves}
%Do not add default error codes
\apicall[get\relax]{/slaves}{Gets all slaves}
	{}
	{200}
	{}
	{array of slave objects}
	{}\label{\gocurpackage.slaves.getAll}

\apicall[put]{/slaves}{Adds a slave}
	{<body> & slave object & slave to add}
	{201}
	{Location & Reference to added slave}
	{}
	{}\label{\gocurpackage.slaves.add}
	
\apicall[get]{/slaves/<id>}{Gets a slave}
	{id & integer & id of the slave to get}
	{200}
	{}
	{slave object}
	{}\label{\gocurpackage.slaves.getById}
	
\apicall[post]{/slaves/<id>}{Updates a slave}
	{id & integer & id of the slave to update\\
	 <body> & slave object & the content of the updated slave}
	{200}
	{}
	{}
	{}\label{\gocurpackage.slaves.update}
	
\apicall[delete]{/slaves/<id>}{Deletes a slave}
	{id & integer & id of the slave to delete}
	{200}
	{}
	{}
	{}\label{\gocurpackage.slaves.delete}
	
\apicall[get]{/slaves/<id>/problems}{Gets the problems of a slave and its Mongods}
	{id & integer & id of the slave to get the problems of}
	{200}
	{}
	{array of problem objects}
	{}\label{\gocurpackage.slaves.getProblems}
	
\subsubsection{Replica Sets}
%Do not add default error codes
\apicall[get\relax]{/replicasets}{Gets all replica sets}
	{}
	{200}
	{}
	{array of replica set objects}
	{}\label{\gocurpackage.replicasets.getAll}

\apicall[put]{/replicasets}{Adds a replica set}
	{<body> & replica set object & replica set to add}
	{201}
	{Location & Reference to added replica set}
	{}
	{}\label{\gocurpackage.replicasets.add}

\apicall[get]{/replicasets/<id>}{Gets a replica set}
	{id & integer & id of the replica set to get}
	{200}
	{}
	{replica set object}
	{}\label{\gocurpackage.replicasets.getById}

\apicall[post]{/replicasets/<id>}{Updates a replica set}
	{id & integer & id of the replica set to update\\
	 <body> & replica set object & content of the updated replica set}
	{200}
	{}
	{}
	{}\label{\gocurpackage.replicasets.update}

\apicall[delete]{/replicasets/<id>}{Deletes a replica set}
	{id & integer & id of the replica set to delete}
	{200}
	{}
	{}
	{}\label{\gocurpackage.replicasets.delete}

\apicall[get]{/replicasets/<id>/slaves}{Gets a replica set's slaves}
	{id & integer & id of the replica set to get the slaves of}
	{200}
	{}
	{array of slave objects}
	{}\label{\gocurpackage.replicasets.getSlaves}

\apicall[get]{/replicasets/<id>/problems}{Gets the problems of a replica set}
	{id & integer & id of the replica set to get the problems of}
	{200}
	{}
	{array of problem objects}
	{}\label{\gocurpackage.replicasets.getProblems}

\subsubsection{Risk Groups}
%Do not add default error codes
\apicall[get\relax]{/riskgroups}{Gets all risk groups}
	{}
	{200}
	{}
	{array of risk groups objects}
	{}\label{\gocurpackage.riskgroups.getAll}

\apicall[put]{/riskgroups}{Adds a risk group}
	{<body> & risk group object & risk group to add}
	{201}
	{Location & Reference to added risk group}
	{}
	{}\label{\gocurpackage.riskgroups.add}

\apicall[get]{/riskgroups/<id>}{Gets a risk group}
	{id & integer & id of the risk group to get}
	{200}
	{}
	{risk group object}
	{}\label{\gocurpackage.riskgroups.getById}

\apicall[post]{/riskgroups/<id>}{Updates a risk group}
	{id & integer & id of the risk group to update\\
	 <body> & risk group object & content of the updated risk group}
	{200}
	{}
	{}
	{}\label{\gocurpackage.riskgroups.update}

\apicall[delete]{/riskgroups/<id>}{Deletes a risk group}
	{id & integer & id of the risk group to delete}
	{200}
	{}
	{}
	{}\label{\gocurpackage.riskgroups.delete}
	
\apicall[put]{/riskgroups/<risk\_group\_id>/slaves/<slave\_id>}{Assigns a slave to a risk group}
	{risk\_group\_id & integer & id of the risk group to assign the slave to\\
	 slave\_id & integer & id of the slave to assign to the risk group}
	{200}
	{}
	{}
	{}\label{\gocurpackage.riskgroups.assignSlave}

\apicall[delete]{/riskgroups/<risk\_group\_id>/slaves/<slave\_id>}{Removes a slave from a risk group}
	{risk\_group\_id & integer & id of the risk group to remove the the slave from\\
	slave\_id & integer & id of the slave to remove from to the risk group}
	{200}
	{}
	{}
	{}\label{\gocurpackage.riskgroups.removeSlave}
\apicall[get]{/riskgroups/<id>/slaves}{Gets a risk group's slaves}
	{id & integer & id of the risk group to get the slaves of}
	{200}
	{}
	{array of slave objects}
	{}\label{\gocurpackage.riskgroups.getSlaves}
	
\subsubsection{Problems}
%Do not add default error codes
\apicall[get\relax]{/problems}{Gets all problems}
	{}
	{200}
	{}
	{array of problem objects}
	{}\label{\gocurpackage.problems.getAll}
	
\apicall[get]{/problems/<id>}{Gets a problem}
	{id & integer & id of the problem to get}
	{200}
	{}
	{problem object}
	{}\label{\gocurpackage.problems.getById}

\struct{MasterAPIServer}
{
	% _an_ HTTP Server http://blog.apastyle.org/apastyle/2012/04/using-a-or-an-with-acronyms-and-abbreviations.html
	Starts an HTTP Server, listens for requests and routes them to the appropriate handler Methods.
}{
	\property{DB}[gorm.DB]{Initialized handle to the database.}
	\property{ClusterAllocator}[\refgo{master.ClusterAllocator}]{Reference to the \codeinline{ClusterAllocator} used to trigger a recalculation when the User changes a replica set or a slave}
}{
	\method{Run}{Runs the HTTP Server and routes the requests.}
}
