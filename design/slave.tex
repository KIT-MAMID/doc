\section{Slave}
\beginpackage{slave}

\begin{figure}[H]
	\includegraphics[width=\textwidth]{slave_classes}
	\caption{Slave}
\end{figure}

The \refgo{slave} is installed on the individual host nodes and started by the main service manager of the operating system during the boot process.

\subsection{Application / main()}{
  A Go application starts by initializing the Go 'main' package and then running the \textit{main()} function.
  This function performs early initialization of the slave's main datastructures:
  \begin{itemize}
    \item Parse command line flags using an external parser package (\codeinline{CLIFlagsParser}).
    \item Initialize the MSPListener.
    \item Initialize the Controller with the MSPListener instance.
    \item Transfer control to MSPListener to wait for incoming connections.
  \end{itemize}
}

\struct{busyTable}{
  Manages a map of Mutexes keyed by Port.
}{
  \property{mutexes}[map[\refgo{msp.PortNumber}]*sync.Mutex]{
    Map of mutexes.\\
    \refproperty{mutexes} must only be altered from a goroutine holding the \refproperty{mutexesLock}.
  }
  \property{mutexesLock}[sync.Mutex]{
    Mutex controlling access to \refproperty{mutexes}.
  }
}{
  \method{AcquireLock}[*sync.Mutex](
    \param{port}[\refgo{msp.PortNumber}]{Port to get a lock for.}
  ){
    Returns a mutex with an active lock. The mutex must be unlocked by caller.
  }
}

\struct{Controller}{
  Handles MSP requests by implementing the \refgo{msp.Consumer} interface.
  Hence, it coordinates the work required to fulfill requests from the master.
  Most importantly, it leverages the \reftype{ProcessManager} and \reftype{MongodConfigurator} to spawn and configure instances of MongoDB.
}{
  \property{busyTable}[\reftype{busyTable}]{
    Holds a mutex per instance of MongoDB.\\
    \vspace{-0.4em}\\ % dirty hack because we do not support paragraphs in property descriptions
    \refgo{msp.Consumer.EstablishMongodState} may be called repeatedly by \refgo{master} while an earlier state establishment request is still executing.\\
    However, a MongoDB instance must not be configured concurrently.\\
    Hence, \refproperty{busyTable} is used to ensure sequential configuration.
  }
  \property{mongodCredentials}[msp[\refgo{msp.PortNumber}]\refgo{msp.MongodCredential}]{Credentials for MongoDB admin users.}
  \property{mongodHardShutdownTimeout}[time.Duration]{Timeout after receiving a stopping state until force killing the Mongod process.}
  \property{procManager}[\reftype{ProcessManager}]{Process and Mongod file Manager.}
  \property{configurator}[\reftype{MongodConfigurator}]{Applies changes over the MongoDB protocol.}
}

\struct{ProcessManager}{
  Starts processes, may provide a list of alive processes and eventually kill these.
}{
  \property{runningProcesses}[map[\refgo{msp.PortNumber}]*exec.Cmd]{Holds the active process controls per instance.}
  \property{dataDir}[string]{Root directory of MongoDB data.}
  \property{command}[string]{Executable to run to start a Mongod process.}
}{
  \method{SpawnProcess}[error](
    \param{m}[\refgo{msp.Mongod}]{Mongod information about what exactly to spawn.}
  ){
    Spawns a Mongod process as requested by the given Mongod inside the data root directory and creates any missing files.
  }
  \method{HasProcess}[bool](
    \param{p}[\refgo{msp.PortNumber}]{Port number to check against.}
  ){
    Checks whether a Mongod process listenting on a given port is running.
  }
  \method{RunningProcesses}[[]\refgo{msp.PortNumber}]{
    Returns the PortNumbers of the currently running processes.
  }
  \method{KillProcess}[error](
    \param{p}[\refgo{msp.PortNumber}]{PortNumber to identify process.}
  ){
    Kills process by given \refgo{msp.PortNumber}. Does not error if the process is already dead, only if it could not be killed.
  }
  \method{KillProcesses}[error]{
    Kills all remaining Processes. Errors if it could not kill some processes.
  }
  \method{CreateManagedDirs}[error]{
    Creates directories as needed for running the slave.
  }
  \method{DestroyDataDirectory}[error](
    \param{m}[\refgo{msp.Mongod}]{Mongod to remove.}
  ){
    Removes all physically stored data of a Mongod.
  }
  \method{UpdateKeyFile}[error](
    \param{m}[\refgo{msp.Mongod}]{Mongod whose keyfile to store.}
  ){
    Stores the keyfile of a Mongod in the filesystem.
  }
  \method{buildMongodCommandLine}[[]string](
    \param{m}[\refgo{msp.Mongod}]{Mongod to build command arguments for.}
  ){
    Computes the arguments to be passed to \codeinline{exec.Command} when spawning Mongod instances.
  }
}

\interface{MongodConfigurator}{
  Applies or returns configuration of a mongod instance by PortNumber.
}{
  \method{MongodConfiguration}[{(\refgo{msp.Mongod}, *\refgo{msp.Error})}](
    \param{p}[\refgo{msp.PortNumber}]{PortNumber to connect to locally.}
  ){
    Reads configuration from a local MongoDB instance.
  }
  \method{ApplyMongodConfiguration}[*\refgo{msp.Error}](
    \param{config}[\refgo{msp.Mongod}]{Configuration to apply.}
  ){
    Connects to the local MongoDB instance given in \refparam{config} parameter and applies the configuration.
  }
}

\struct{ConcreteMongodConfigurator}{
  Implements \reftype{MongodConfigurator} interface and uses \reftype{mgoContext} to communicate with local instances.
}{
  \property{MongodSoftShutdownTimeout}[time.Duration]{Timeout to pass to \reftype{mgoContext.ShutdownWithTimeout}.}
  \property{MongodResponseTimeout}[time.Duration]{Timeout before any command to \reftype{mgoContext} is being failed.}
}{
  \method{connect}[*\reftype{mgoContext}, *\refgo{msp.Error}]{Establishes a communication to a Mongod instance.}
}

\struct{mgoContext}{
  Wraps mgo.Session and all the admin commands.
}{
  \property{Session}[*mgo.Session]{Session object of the MongoDB client library.}
  \property{Port}[msp.PortNumber]{Port the session is connected to.}
  \property{LoginSuccessful}[bool]{Indicator of successful connection and authentication.}
}{
  \method{ReplSetInitiate}[{(bool, *\refgo{msp.Error})}](
    \param{c}[mgo.config]{Replica Set configuration object.}
    \param{force}[bool]{Whether to force initialization.}
  ){
    Sends replSetInitiate command.\\
    Returns true if already initialized, false if freshly initialized. Error on failure.
  }
  \method{CreateUser}[*\refgo{msp.Error}](
    \param{user}[string]{Username.}
    \param{password}[string]{Password of user.}
    \param{purpose}[string]{Reason for user creation to include in error message upon failure.}
    \param{roles}[[]string]{Roles of user.}
  ){
    Creates a new MongoDB user.
  }
  \method{ReplSetStepDown}[*\refgo{msp.Error}](
    \param{seconds}[int]{Period for secondaries to catch up.}
  ){
    Give up being a primary, triggering a reelection.
  }
  \method{ShutdownWithTimeout}[*\refgo{msp.Error}](
    \param{seconds}[int]{Timeout until shutdown is forced by the Mongod.}
  ){
    Trigger a Mongod shutdown, delayed by a given timeout.
  }
  \method{ParseCmdLineShardingRole}[{(string, *\refgo{msp.Error})}]{
    Get the sharding role from the Mongod.
  }
  \method{ReplSetReconfig}[*\refgo{msp.Error}](
    \param{c}[mgo.config]{Replica Set configuration object.}
  ){
    Send replSetReconfig command.
  }
  \method{ReplSetGetConfig}[{(mgo.config, *\refgo{msp.Error})}]{
    Returns the current Replica Set configuration.
  }
  \method{ReplSetGetStatus}[{(\reftype{replSetState}, *\refgo{msp.Error})}](
    \param{s}[*mgo.status]{Will be passed the Replica Set status returned by the Mongod.}
  ){
    Fetch the Replica Set Status.
  }
  \method{IsMaster}[*\refgo{msp.Error}](
    \param{m}[*mgo.master]{Will be passed the return value of isMaster command.}
  ){
    Send isMaster command.
  }
  \method{Close}{
    Closes the session.
  }
}

\goenum{replSetState}
{}
{
  \goenumitem{replSetStartup}{Replica Set has been initiated.}
  \goenumitem{replSetPrimary}{Replica Set is active and the Mongod is a primary.}
  \goenumitem{replSetSecondary}{Replica Set is active and the Mongod is a secondary.}
  \goenumitem{replSetRecovering}{Replica Set is currently recovering.}
  \goenumitem{replSetUnknown}{Status is unknown.}
  \goenumitem{replSetRemoved}{Mongod has been removed from the Replica Set.}
}
