\newcommand{\pluseq}{\mathrel{{+}{=}}}
\newcommand{\minuseq}{\mathrel{{-}{=}}}
\SetKwInOut{Input}{input}
\SetKwInOut{Output}{output}
\SetKwProg{Fn}{Function}{}{}
\SetKw{Invariant}{invariant}
\IncMargin{0.5em}

We start with the removal of superfluous Replica Set members.\\
However, in order to determine how many Replica Set members can be removed, we need to examine the observed situation.

\begin{algorithm}[H]
\caption{Count members of a Replica Set that are \& intended to be in stable state (running)}\label{alg:EffectiveMemberCount}

\Input{ReplicaSet $r$}
\Output{Number of $p_e$ (\emph{persistent}) and $v_e$ (\emph{volatile}) member processes of $r$
        that are fully operational and are planned to remain in that state.}
\BlankLine
\Fn{EffectiveMemberCount(r ReplicaSet)}{

$p_e, v_e = 0$

\For{$m \in \text{r.Mongods}$}{
	\If{$\text{m.ObservedState.ExecutionState} = Running$ \\ 	%	#this line evaluates to false if m.ObservedState = NULL
		$\land \text{m.DesiredState.ExecutionState} = Running$}{ %#this line evaluates to false if m.DesiredState = NULL 

		\uIf{$m.ParentSlave.PersistentStorage$}{
			$p_e \pluseq 1 $
		}\Else{
			$v_e \pluseq 1 $
		}
	}
}
\Return $p_e, v_e$
}
\end{algorithm}

Given Algorithm \ref{alg:EffectiveMemberCount}, we can now proceed with the removal of superfluous Replica Set members.

\begin{algorithm}[H]
\caption{Destroy members of disabled replica sets where possible without violating p/v constraints.}
\ForEach{r in ReplicaSets}{
	
	$p_e, v_e \gets $ EffectiveRunningMembers(r)
	
        \ForEach(// \textbf{x} is symbolic placeholder for \textbf{p} or \textbf{v}){$\textbf{x}_e \in \{p_e, v_e\}$}{
		\While{$x_e > r.\textbf{x}$}{
			\textbf{destroy} any $m \in r.Mongods$ where $m.ParentSlave$ is of type $\textbf{x} \land \textbf{disabled}$\;
			$\textbf{x}_e \minuseq 1$
		}
		
		\Invariant minimum number of disabled slaves are running members of $r$\;
		
		\While{$\textbf{x}_e > r.\textbf{x}$}{
			\textbf{destroy} any $m \in r.Mongods$ where $m.ParentSlave$ is of type $\textbf{x}$\;
			$\textbf{x}_e \minuseq 1$
		}
		
		\Invariant at most $r.\textbf{x}$ fully operational members 
	}
	
	\Invariant \textit{desired state}: at most (r.p|r.v) member processes of $r$\;
	// desired state = the state that will be deployed
	
}
\end{algorithm}

With all superflous members removed, we now approach the other side of the problem
and add new members to Replica Sets that need them.\\
Again, we need to examine the current state of the cluster.

\begin{algorithm}[H]
\caption{Count recovering and active members of a Replica Set}
\label{alg:AlreadyAddedMemberCount}
\Input{ReplicaSet $r$}
\Output{Number of $p_a$ and $v_a$ member processes of $r$ that \begin{itemize}
		\item are \textbf{recovering}, i.e. soon-to-be fully operational
		\item fully operational (\textbf{running})
	\end{itemize} and should actually be in that state.}
\BlankLine
\Fn{AlreadyAddedMemberCount(r ReplicaSet)}{
	$p_a, v_a = 0$
	
	\For{$m \in \text{r.Mongods}$}{
		\If{$\text{m.ParentSlave.ConfiguredState} \not= Disabled$ \\
                        $\land \text{m.DesiredState.ExecutionState} \not\in \{ NotRunning, Destroyed\}$}{ 
			
			\uIf{$m.ParentSlave.PersistentStorage$}{
				$p_a \pluseq 1 $
			}\Else{
			$v_a \pluseq 1 $
		}
	}
}
\Return $p_a, v_a$
}
\end{algorithm}

Knowing the needs of individual replica sets, we spawn new Mongod processes as needed.\\
Spawning of processes is constrained by
\begin{itemize}
  \item the Risk Group memberships of the slaves
  \item the length of the Mongod port range configured for the slave
\end{itemize}

The following algorithm utilizes \texttt{PriorityQueues}
\begin{description}
  \item[\texttt{RQ}] to prioritize heavier-degraded Replica Sets (i.e. those with the most missing members)
  \item[\texttt{RGSQ}] to prioritize little-used slaves as hosts for new Mongods (i.e. those slaves with the most unused capacity)
\end{description} 

The \texttt{ARGMAX} returns the element from the left operand whose metric (right operand) is the maxium for all elements of the left operand.

\begin{algorithm}[H]
\label{alg:addmongods}
\caption{Spawn Mongods on under-provisioned Replica Sets respecting RiskGroup \& p/v constraints.}

$p_a, v_a \gets $ AlreadyAddedMemberCount(r)

\ForEach(// \textbf{x} is symbolic placeholder for \textbf{p} or \textbf{v}){$\textbf{x}_a \in \{p_a, v_a\}$}{

	%# TODO need to fill the queue. only ReplicaSets which 'need' (use p_* to define what need means) are in  the queue 
	$RQ \gets$ PriorityQueue(R)\;% "relative amount of missing members")\;
        $S_{avail} \gets \text{all \textbf{active} slaves with} \geq \textbf{1 free port}$\;
        $RGSQ \gets$ map[RiskGroup]PriorityQueue(Slaves $\in S_{avail}$ by RiskGroup)\;%, "relative amount of available Mongod ports")\;
	
	\While{$r = RQ.pop()$; $r \neq nil$}{
	
		\Invariant $r$ needs a member of type $\textbf{x}$
		
		// find slave $s$ of type $x$ in a risk group $g$ with no other Mongod of $r$ in $g$\;
                $G_{candidates}\gets \{g \mid g \in RGSQ.keys \land \text{r has no member in g}\}$\;
                $g \gets$ ARGMAX($g \in G_{candidates}$, RGSQ[g].PeekPriority())\;
		\uIf{$g \neq nil$}{
			$s \gets RGSQ[g].pop()$\;
			\Invariant $s$ has at least one free port\;
			spawn new Mongod $m$ on $s$ and add it to $r.Mongods$\;
			compute $MongodState$ for $m$ and set the $DesiredState$ variable\;
			
			\uIf{$r$ needs another member of type $\textbf{x}$}{
				$RQ.insert(r)$ // recompute priority
                        }\Else{
                                // r.\textbf{x} constraint has been fulfilled
                        }

			\If{$s$ has free ports}{
				$RGSQ[g].insert(s)$ // recompute priority
			}
			
			
		} \Else{
                        // r.\textbf{x} constraint cannot be fulfilled
			continue\;
		}
		
	
	}

}

\end{algorithm}

It is crucial to the implementation of Algorithm \ref{alg:addmongods} that Mongods in \refgo{model.MongodExecutionState.Recovering}
\begin{itemize}
  \item are counting toward the \texttt{AlreadyAddedMemberCount} (Algorithm \ref{alg:AlreadyAddedMemberCount})
  \item are not not counting toward the \texttt{EffectiveMemberCount} (Algorithm \ref{alg:EffectiveMemberCount})
\end{itemize}

If this were not the case, the following failure scenarios would be possible:
\begin{itemize}
  \item Oscillation, violating the fundamental requirement to the algorithm stated in the theorem on \hyperref[theorem:idempotence_clusterallocator]{Idempotence of the ClusterAllocator}
  \item Invisible degradation (members in \refgo{model.MongodExecutionState.Recovering} do not hold all Replica Set data yet)
  \item Data Loss (as a result of Oscillation + Invisible degradation)
\end{itemize}
