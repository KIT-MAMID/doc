\section{MasterSlaveProtocol}
\renewcommand{\gocurpackage}{msp}

\begin{figure}[H]
	\includegraphics[width=\textwidth]{msp_classes}
	\caption{MasterSlaveProtocol}
\end{figure}

The MasterSlaveProtocol \refgo{msp} implements communication between \refgo{slave} and \refgo{master}.\\
There are two structures which may be sent directly over HTTP as request body, represented with JSON:
\begin{itemize}
  \item \codeinline{[]\reftype{Mongod}} An array of descriptions of all mongod instances of that \refgo{slave}.
  \item \codeinline{\reftype{SlaveError}} Sent (only) by \refgo{slave} in case there are general problems preventing observation of any running mongod instances.
\end{itemize}

% server (slave) side structures
\struct{Listener}{
  Structure for handling communication through the MasterSlaveProtocol on the \refgo{slave} side.
}{
  \property{Consumer}[\reftype{Consumer}]{
    Reference to an event handler, typically a controller.
  }
}{
  \method{Run}{
    Start listening for connections using the MasterSlaveProtocol.
    \begin{itemize}
      \item Listen for incoming connections from the \refgo{master}.
      \item Decode / Encode the transport format.
      \item Validate incoming request format.
      \item Delegate valid requests to the \reftype{Consumer}.
    \end{itemize}
  }
}

\interface{Consumer}{
  A Consumer handles MasterSlaveProtocol requests received by an instance of \reftype{Listener}.
}{
  \method{RequestStatus}[{(\reftype{SlaveStatus}, \reftype{SlaveError})}]{
    %TODO reference request message object here -> proto specification
    Request to provide a status report of the slave's state.
    See \reftype{Mongod} for details on the data returned by this method.
  }
  \method{EstablishMongodState}[*\reftype{SlaveError}](
    \param{m}[\reftype{Mongod}]{State description of a mongod instance that shall be established}
  ){
    %TODO reference request message object here -> proto specification
    Request to establish the configuration state $m$ of a MongoDB process on the slave.
    May return an error in case the state could not be established.
  }
}


% client (master) side structures
\struct{Client}{
  Structure for handling communication through the MasterSlaveProtocol on the \refgo{master} side.
}{
  \property{Target}[\reftype{HostPort}]{Remote endpoint it holds a connection to}
}{
  \method{Connect}(
    \param{target}[\reftype{HostPort}]{Remote endpoint to connect to}
  ){
    Establish the connection to the server on the \refgo{slave} side.
  }
  \method{RequestStatus}[{([]\reftype{Mongod}, \reftype{error})}]{
    Send a status request to the \refgo{slave} to retrieve all the running \reftype{Mongod} instances. May return an instance of \reftype{error} in case overall retrieving failed.
  }
  \method{EstablishMongodState}[*\reftype{CommunicationError}](
    \param{m}[[]\reftype{Mongod}]{Array of \reftype{Mongod} instances to establish}
  ){
    Sends the array of \reftype{Mongod} instances to the \refgo{slave} which is tasked with creating and configuring the described processes.
  }
}


% protocol specific structures
\struct{Mongod}{
  Description of a mongod instance.
}{
  \property{Port}[\reftype{PortNumber}]{The port the mongod instance listens on}
  \property{ReplicaSetName}[string]{Identifier for the replica set the mongod instance participates in}
  \property{ShardingConfigServer}[bool]
  \property{StatusError}[*\reftype{SlaveError}]{Set by \refgo{slave} when the described mongod instance errored}
  \property{LastEstablishStateError}[*\reftype{SlaveError}]{The last \reftype{SlaveError} created upon trying to establish state (\codeinline{nil} when sent by \refgo{master} or when state has successfully been established since the last establish request)}
  \property{State}[\reftype{MongodState}]{State the mongod instance is in}
  \property{ReplicaSetMembers}[[]\reftype{HostPort}]{Targets to connect the ReplicaSet to}
}

\goenum{MongodState}{
  Describes the state of the mongod instance.
}{
  \goenumitem{MongodStateDestroyed}{The mongod instance shall be killed on the \refgo{slave}; only sent by \refgo{master}}
  \goenumitem{MongodStateNotRunning}{The mongod instance was not (successfully) stated by the \refgo{slave}}
  \goenumitem{MongodStateRecovering}{The mongod instance is currently in process of applying a state change; only sent by \refgo{slave}}
  \goenumitem{MongodStateRunning}{Normal operation of the mongod instance}
}

\struct{HostPort}{
  Container for a hostname\&port pair.
}{
  \property{Hostname}[string]
  \property{Port}[\reftype{PortNumber}]
}

\typealias{PortNumber}[uint16]{
  A TCP port number as specified in \href{https://tools.ietf.org/html/rfc793\#section-3.1}{RFC 793 Section 3.1}.
}

\interface{error}{
  Generic interface for errors of the MasterSlaveProtocol. The \refgo{master.ProblemManager} will have to deal with the specific types of errors.
}

\struct{SlaveError}{
  \reftype{error} as sent by the \refgo{slave}.
}{
  \property{Identifier}[string]{Unique identifier in order to distinguish new from old errors}
  \property{Description}[string]{Short description of the error condition}
  \property{LongDescription}[string]{Long description with full information in order to ease debugging}
}

\struct{CommunicationError}{
  \reftype{error} as generated by the \refgo{master} (client) in case it cannot read or write, or the connection timed out.
}{
  \property{Message}[string]{Failure message from internal Go functions or message about the timeout}
}