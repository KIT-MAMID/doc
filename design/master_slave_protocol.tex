\section{MasterSlaveProtocol}
\beginpackage{msp}

\begin{figure}[H]
	\includegraphics[width=\textwidth]{msp_classes}
	\caption{MasterSlaveProtocol}
\end{figure}

The MasterSlaveProtocol \refgo{msp} implements communication between \refgo{slave} and \refgo{master}.\\
There are three structures which may be sent directly over HTTP as request body, represented with JSON:
\begin{itemize}
  \item \codeinline{[]\reftype{Mongod}} An array of descriptions of all Mongod instances of that \refgo{slave}.
  \item \codeinline{\reftype{Error}} Sent (only) by \refgo{slave} in case there are general problems preventing observation of any running Mongod instances.
  \item \codeinline{\reftype{RsInitiateMessage}} Sent by master to initiate the Replica Set through a Mongod using a Replica Set configuration. Is sent to a Replica Set until it succeeds and not again after it succeeded once.
\end{itemize}

For several datastructures in \refgo{msp}, similar ones exist in  \refgo{model}. The reasoning behind this is the desire to have a
stable protocol format for the \refgo{msp}: the JSON objects are auto-generated during runtime via reflection.
Hence, duplication of datastructures is a necessary tradeoff in order to sustain the principle of information hiding.

% server (slave) side structures
\struct{Listener}{
  Structure for handling communication through the MasterSlaveProtocol on the \refgo{slave} side.
}{
  \property{Consumer}[\reftype{Consumer}]{
    Reference to an event handler, typically a controller.
  }
}{
  \method{Run}{
    Start listening for connections using the MasterSlaveProtocol.
    \begin{itemize}
      \item Listen for incoming connections from the \refgo{master}.
      \item Decode / Encode the transport format.
      \item Validate incoming request format.
      \item Delegate valid requests to the \reftype{Consumer}.
    \end{itemize}
  }
}

\interface{Consumer}{
  A Consumer handles MasterSlaveProtocol requests received by an instance of \reftype{Listener}.
}{
  \method{RequestStatus}[{([]\reftype{Mongod}, *\reftype{Error})}]{
    %TODO reference request message object here -> proto specification
    Request to provide a status report of the slave's state.
    See \reftype{Mongod} for details on the data returned by this method.
  }
  \method{EstablishMongodState}[*\reftype{Error}](
    \param{m}[\reftype{Mongod}]{State description of a Mongod instance that shall be established.}
  ){
    %TODO reference request message object here -> proto specification
    Request to establish the configuration state $m$ of a MongoDB process on the slave.
    May return an error in case the state could not be established.
  }
  \method{RsInitiate}[*\reftype{Error}](
  \param{m}[\reftype{RsInitiateMessage}]{Message containing Replica Set configuration that shall be established and additional information needed to initiate the Replica Set.}
  ){
  	%TODO reference request message object here -> proto specification
  	Request to initiate a Replica Set and establish the configuration $m$ of the Replica Set.
  	For a given Replica Set this message is sent repeatedly until it succeeds and never again after that.
  	May return an error in case the Replica Set could not be initiated.
  }
}


% client (master) side structures
\struct{Client}{
  Structure for handling communication through the MasterSlaveProtocol on the \refgo{master} side.
}{
  \item \textit{none}
}{
  \method{RequestStatus}[{([]\reftype{Mongod}, *\reftype{Error})}]{
    Send a status request to the \refgo{slave} to retrieve all the running \reftype{Mongod} instances. May return an instance of \reftype{error} in case overall retrieving failed.
  }
  \method{EstablishMongodState}[*\reftype{Error}](
    \param{m}[[]\reftype{Mongod}]{Array of \reftype{Mongod} instances to establish}
  ){
    Sends the array of \reftype{Mongod} instances to the \refgo{slave} which is tasked with creating and configuring the described processes.
  }
  \method{RsInitiate}[*\reftype{Error}](
  \param{m}[\reftype{RsInitiateMessage}]{Message containing Replica Set configuration that shall be established and additional information needed to initiate the Replica Set.}
  ){
  	%TODO reference request message object here -> proto specification
  	Sends a request to initiate a Replica Set and establish the configuration $m$ of the Replica Set.
  	For a given Replica Set this message is sent repeatedly until it succeeds and never again after that.
  	May return an error in case the Replica Set could not be initiated.
  }
}


% protocol specific structures
\struct{RsInitiateMessage}{
  Datastructure telling the slave to initiate a Replica Set though one of its Mongods.
}{
  \property{Port}[\reftype{PortNumber}]{Port of the Mongod process that shall initiate the Replica Set.}
  \property{ReplicaSetConfig}[\reftype{ReplicaSetConfig}]{Intitial configuration of the Replica Set.}
}

\struct{Mongod}{
  Datastructure describing a Mongod instance controlled by the slave.
}{
  \property{Port}[\reftype{PortNumber}]{The port the Mongod instance listens on.}
  \property{ReplicaSetConfig}[\reftype{ReplicaSetConfig}]{Configuration of the Replica Set the Mongod instance participates in.}
  \property{KeyfileContext}[string]{Content of the keyfile of the Mongod (\codeinline{--keyfile} mongod option).}
  \property{StatusError}[*\reftype{Error}]{Set by \refgo{slave} when retrieving the status of the Mongod instance resulted in an error.}
  \property{LastEstablishStateError}[*\reftype{Error}]{The last \reftype{Error} created upon executing \refgo{msp.Consumer.EstablishMongodState}.\\
       (\codeinline{nil} when sent by \refgo{master} or when state has successfully been established since the last establish request).}
  \property{State}[\reftype{MongodState}]{State the Mongod instance is / should be in.}
}

\struct{ReplicaSetConfig}{
  Contains configuration of a Replica Set.
  When sent in response to \refgo{msp.Consumer.RequestStatus}, \reftype{ReplicaSetConfig} is based on what is reported by the running Mongod instance.
}{
  \property{ReplicaSetName}[string]{Name of the Replica Set.}
  \property{RootCredential}[\reftype{MongodCredential}]{Credentials of the admin user of the Replica Set.}
  \property{ShardingRole}[\reftype{ShardingRole}]{Whether to use sharding / The role of the Replica Set in the sharded cluster.}
  \property{ReplicaSetMembers}[[]\reftype{ReplicaSetMember}]{Members of the Replica Set.}
}

\goenum{ShardingRole}
{}
{
  \goenumitem{None}{No sharding on the Replica Set.}
  \goenumitem{ShardServer}{Replica Set is a normal Shard.}
  \goenumitem{ConfigServer}{Replica Set is a config Shard.}
}

\goenum{MongodState}{
  Describes the state of a Mongod instance controlled by the slave.
}{
  \goenumitem{MongodStateForceDestroyed}{The Mongod instance should be killed and all related data stored on the \refgo{slave} host should be deleted. (only sent by \refgo{master})}
  \goenumitem{MongodStateDestroyed}{The Mongod instance should be shut down without force and all related data stored on the \refgo{slave} host should be deleted if it has been shut down. (only sent by \refgo{master})}
  \goenumitem{MongodStateNotRunning}{The Mongod instance is not running / should be shut down without force but related data is kept on the \refgo{slave}.}
  \goenumitem{MongodStateRunning}{The Mongod instance is running and in sync with other Replica Set members.}
  \goenumitem{MongodStateRecovering}{The Mongod is running but not available for reads because data needs to be synced from other Replica Set members.
                                     This is a common case when adding new members to an existing Replica Set. (only sent by \refgo{slave})}
  \goenumitem{MongodStateUninitiated}{The Mongod is running but the Replica Set has not been initiated yet or the Mongod is not added to the Replica Set (yet) (only sent by \refgo{slave})}
  \goenumitem{MongodStateRemoved}{The Mongod has been removed from the Replica Set. (only sent by \refgo{slave})}
}

\struct{ReplicaSetMember}{
  Member of a Replica Set identified by a \reftype{HostPort}.
}{
  \property{HostPort}[\reftype{HostPort}]{Location of the Replica Set member.}
  \property{Priority}[float]{Priority of the member in the Replica Set.}
  \property{Votes}[int]{1 if the member is allowed to vote, 0 otherwise. If it is not allowed to vote \refproperty{Priority} has to be 0.}
}

\struct{HostPort}{
  Tuple of $(hostname, port)$.
}{
  \property{Hostname}[string]{The unique hostname of a machine resolvable by the slave.}
  \property{Port}[PortNumber]{The TCP port on the host identified by \refproperty{Hostname}}
}

\typealias{PortNumber}[uint16]{
  A TCP port number as specified in \href{https://tools.ietf.org/html/rfc793\#section-3.1}{RFC 793 Section 3.1}.
}

\struct{Error}{
  Description of an error in the slave or in the communication to it, usually corresponding to a partial error when executing a request from the master.
  See \refgo{msp.Mongod} for examples.
}{
  \property{Identifier}[string]{Unique identifier for the type of error. Set of compile-time constants in the slave implementation.
                                Intended for debugging purposes and human-to-human conversation about an error.}
  \property{Description}[string]{Short human-readable description of the error, suitable for display in subject lines, etc.}
  \property{LongDescription}[string]{Long human-readable description of the error.\\
                                     Can contain the \codeinline{error.Error()} of one or more underlying errors or other useful
                                     information for further debugging.}
}