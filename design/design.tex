%% das Papierformat zuerst
\documentclass[a4paper, 11pt]{article}
\usepackage[margin=3cm]{geometry}
\usepackage[utf8]{inputenc}
\usepackage[T1]{fontenc}
\usepackage[fleqn]{amsmath} %left aligned equations
\usepackage{hyperref} % clickable refs
\usepackage{graphicx}
\usepackage[toc, numberedsection]{glossaries}
\usepackage{float}
\usepackage{amssymb}
\usepackage{calc}
\usepackage{enumitem} 
\usepackage{url}
\usepackage{parskip}
% http://tex.stackexchange.com/questions/17730/newcommand-and-spacing
\usepackage{xspace}
% 'frame' option for figures
\usepackage[export]{adjustbox}
% fancy page headers
\usepackage{fancyhdr}
\usepackage{xcolor}
\usepackage{tabularx}
\usepackage{listings}

% page style
\pagestyle{fancy}
\fancyhead[R]{\includegraphics[width=2cm]{../assets/kitlogo}}
\fancyhead[L]{\leftmark}
\fancypagestyle{plain}{
	\rhead{\includegraphics[width=2cm]{../assets/kitlogo}}
	\lhead{\leftmark}
}

% TODO: template übersetzten

\makeglossaries

%Hack for referencing labels
\makeatletter
\def\namedlabel#1#2{\begingroup
    #2%
    \def\@currentlabel{#2}%
    \phantomsection\label{#1}\endgroup
}
\makeatother
% End: Hack for referencing labels

% Glossar: alle Einträge, aber ohne extra Referenzen
% http://tex.stackexchange.com/questions/115635/glossaries-suppress-pages-when-using-glsaddall
\newcommand*{\glsgobblenumber}[1]{}
\makeatletter
\newcommand*{\glsaddnp}[2][]{
  \glsdoifexists{#2}{
    \def\@glsnumberformat{glsgobblenumber}
    \edef\@gls@counter{\csname glo@#2@counter\endcsname}
    \setkeys{glossadd}{#1}
    \@gls@saveentrycounter
    \@do@wrglossary{#2}
  }
}
\renewcommand{\glsaddallunused}[1][]{
  \edef\@glo@type{\@glo@types}
  \setkeys{glossadd}{#1}
  \forallglsentries[\@glo@type]{\@glo@entry}{
    \ifglsused{\@glo@entry}{}{
      \glsaddnp[#1]{\@glo@entry}}}
}
\makeatother

\renewcommand{\glsnamefont}[1]{\mdseries #1} % glossary entries shouldn’t be bold

% Glossar

% So sieht ein Glossar-Eintrag aus:
%
%\newglossaryentry{dijkstra}{
%  name={Dijkstra’s Algorithmus},
%  description={ein Algorithmus, um den optimalen Pfad in einem gerichteten Graphen zu finden}
%}
%\newglossaryentry{arc}{
%  name={Arc-Flags},
%  description={eine Technik, um Routenberechnung zu beschleunigen},
%  see={dijkstra}
%}
%
% Und so kann er im Dokument verwendet werden:
%
% lorem ipsum dolor sit \gls{arc}, consectetur
%
% End: Glossar

% usage: \counteditem{prefix}{refName} -> item `/prefixXX/` with label `prefix:refName` (where XX is counted in increments of 10)
\makeatletter
\newcommand{\oitem}[2]{
  % define the counter
  \@ifundefined{c@oitem#1}{\newcounter{oitem#1}}{} % initialized at 0
  \addtocounter{oitem#1}{10}
  \item[\namedlabel{#1:#2}{/#1\arabic{oitem#1}/}]
}
\makeatother

% new page after section
\let\oldsection\section
\renewcommand\section{\clearpage\oldsection}

\newcommand{\mamidscreenshot}[1]{\includegraphics[width=\textwidth,frame]{#1}}

\newcommand{\uiel}[3]{\item \textbf{"#1" #2:} #3}

\begin{document}

% place a symbol before clickable links
% this has to come *after* \begin{document} because hyperref installs a \AtBeginDocument hook that updates the ref command.
\newcommand{\refsymbol}[0]{\scalebox{0.5}{$\nearrow$}}
\let\oldref\ref
\renewcommand{\ref}[1]{\refsymbol\oldref{#1}}
\let\oldgls\gls
\renewcommand{\gls}[1]{\refsymbol\oldgls{#1}}
\let\oldGls\Gls
\renewcommand{\Gls}[1]{\refsymbol\oldGls{#1}}
\let\oldglspl\glspl
\renewcommand{\glspl}[1]{\refsymbol\oldglspl{#1}}
\let\oldGlspl\Glspl
\renewcommand{\Glspl}[1]{\refsymbol\oldGlspl{#1}}
\let\oldglslink\glslink
\renewcommand{\glslink}[2]{\refsymbol\oldglslink{#1}{#2}}
\let\oldhyperref\hyperref
\renewcommand{\hyperref}[2][notActuallyOptional]{\refsymbol\oldhyperref[#1]{#2}}
\let\oldautoref\autoref
\renewcommand{\autoref}[1]{\refsymbol\oldautoref{#1}}

\newcommand{\abbildung}[1]{\autoref{fig:#1}}
\newcommand{\mamid}{\textit{MAMID}\xspace}

% alle Glossareintraege
\newacronym{gui}{GUI}{Graphical User Interface}
\newacronym{cli}{CLI}{Command Line Interface}

\newglossaryentry{cluster}{
  name={cluster},
  description={It's something.},
  plural={clusters}
}
\newglossaryentry{MongoDB}{
	name={MongoDB},
	description={It's something.},
	plural={MongoDB}
}
\newglossaryentry{replica set}{
	name={replica set},
	description={It's something.},
	plural={replica sets}
}
\newglossaryentry{administrator}{
	name={administrator},
	description={It's something.},
	plural={administrators}
}
\newglossaryentry{host}{
	name={host},
	description={It's something.},
	plural={hosts}
}
\newglossaryentry{slave}{
	name={slave},
	description={It's something.},
	plural={slaves}
}
\newglossaryentry{master}{
	name={master},
	description={It's something.},
	plural={masters}
}
\newglossaryentry{inventory}{
	name={inventory},
	description={It's something.},
	plural={inventories}
}
\newglossaryentry{disabled mode}{
	name={disabled mode},
	description={It's something.},
	plural={}
}
\newglossaryentry{physical interdependency}{
	name={physical interdependency},
	description={It's something.},
	plural={physical interdependencies}
}
\newglossaryentry{maintenance mode}{
	name={maintence mode},
	description={It's something.},
	plural={}
}
\newglossaryentry{persistent storage}{
	name={persistent storage},
	description={It's something.},
	plural={}
}
\newglossaryentry{volatile storage}{
	name={volatile storage},
	description={It's something.},
	plural={}
}
\newglossaryentry{root data directory}{
	name={root data directory},
	description={It's something.},
	plural={hosts}
}
\newglossaryentry{arbiter}{
	name={arbiter},
	description={It's something.},
	plural={hosts}
}
\newglossaryentry{degraded}{
	name={degraded},
	description={It's something.},
	plural={hosts}
}
\newglossaryentry{HTTP API}{
	name={HTTP API},
	description={It's something.},
	plural={hosts}
}
\newglossaryentry{active mode}{
	name={active mode},
	description={It's something.},
	plural={hosts}
}

\begin{titlepage}
\makeatletter
\begin{center}
~\\[4em]
{\Huge MAMID}\\[.8em]\huge{Monitor and Manager for In-memory Databases}\\[2em]
{\huge Design}\\[1em]
{\large\today}\\[2.5em]
{\LARGE
Niklas Fuhrberg\\
Anton Schirg\\
Christian Schwarz\\
Janis Streib\\
Bob Weinand\\[3em]}
{\Large supervised by}\\[2em]
{\LARGE
Dr Marek Szuba\\[1em]}
{\Large at}\\[1em]
{\LARGE
Karlsruhe Institute of Technology\\
SCC\\[2em]}
{\color{gray}
  \small Document Version: \newtheorem{theorem}{Theorem}


\section{master}

\renewcommand{\gocurpackage}{model}
\renewcommand{\gocurpackage}{master}

\struct{ClusterAllocator}{
  The \refstruct{ClusterAllocator} determines the layout of the cluster managed by \mamid.

  It attempts to fulfill the constraints defined through the model objects, in particular
  \begin{itemize}
    \item A \refgo{model.ReplicaSet}'s \refgo{VolatileNodeCount} \& \refgo{PersistentNodeCount}
    \item The \refgo{model.Slave}'s allowed number of Mongod instances
          (\refgo{MongodPortRangeBegin} to \refgo{MongodPortRangeEnd}).
    \item The \refgo{model.Slave.SlaveState}
    \item The configured \refgo{model.RiskGroup}s.
  \end{itemize}

  An iterative algorithm is employed to decide on a cluster layout described through
  \refgo{model.Mongod.DesiredState}s that
  \begin{itemize}
    \item attempts to fulfill the above constraints
    \item attempts an even distribution of Mongods on the different cluster hosts
    \item is a minimal change in comparison to the previous layout
  \end{itemize}

  \begin{theorem}{Idempotence of the ClusterAllocator}
    \label{theorem:idempotence_clusterallocator}
    Let $l$ be a layout of the cluster. Then $ClusterAllocator(ClusterAllocator(l)) = ClusterAllocator(l)$.
  \end{theorem}

}{
  \property{DB}[gorm.DB]{Initialized handle to the database.}
  \property{BusChannel}[chan interface{}]{Initialized channel to the application bus.} %TODO ref}
}{
  \method{LayoutCluster}{Lay out the cluster as described above.}
}

\subsubsection{Pseudocode}

The \refstruct{master.ClusterAllocator} is crucial to the stable operation of the \mamid-managed cluster.\\
Hence, it is worth defining the implementation of \refgo{master.ClusterAllocator.LayoutCluster} through pseudocode. %TODO parentheses

While studying the algorithms below, the reader should keep in mind that
\begin{itemize}
  \item changes in the cluster layout $\equiv$ change or creation of \refgo{model.Mongod.DesiredState} 
  \item the \refgo{model.Mongod.ObservedState} may change after an arbitrary amount or even never.\\
  \item changes to a ReplicaSet must not violate or further worsen the high-availibility constraints,
        in particular \refgo{model.ReplicaSet}'s \refgo{VolatileNodeCount} \& \refgo{PersistentNodeCount}\\
        $\impl$ Mongods in \refgo{model.MongodExecutionState.Recovering} are an important special-case.
\end{itemize}

\newcommand{\pluseq}{\mathrel{{+}{=}}}
\newcommand{\minuseq}{\mathrel{{-}{=}}}
\SetKwInOut{Input}{input}
\SetKwInOut{Output}{output}
\SetKwProg{Fn}{Function}{}{}
\SetKw{Invariant}{invariant}
\IncMargin{0.5em}

Invariant: Algorithm(Algorithm(x)) = Algorithm(x)

Otherwise oscillations could occur

Most interesting case: Set a slave to disabled. Then a new Mongod should be spawned and recover and when it is done the old disabled one can be deleted.

\begin{algorithm}

\caption{Count members of a Replica Set that are \& intended to be in stable state (running)}

\Input{ReplicaSet $r$}
\Output{Number of $p_e$ and $v_e$ member processes of $r$ that are fully operational and are planned to remain in that state.}
\BlankLine
\Fn{EffectiveMemberCount(r ReplicaSet)}{

$p_e, v_e = 0$

\For{$m \in \text{r.Mongods}$}{
	\If{$\text{m.ObservedState.ExecutionState} = Running$ \\ 	%	#this line evaluates to false if m.ObservedState = NULL
		$\land \text{m.DesiredState.ExecutionState} = Running$}{ %#this line evaluates to false if m.DesiredState = NULL 

		\uIf{$m.ParentSlave.PersistentStorage$}{
			$p_e \pluseq 1 $
		}\Else{
			$v_e \pluseq 1 $
		}
	}
}
\Return $p_e, v_e$
}
\end{algorithm}


\begin{algorithm}
\caption{Destroy members of disabled replica sets where possible without violating p/v constraints.}
\ForEach{r in ReplicaSets}{
	
	$p_e, v_e \gets $ EffectiveRunningMembers(r)
	
	\ForEach(// same for persistent and volatile){$\textbf{x}_e \in \{p_e, v_e\}$}{
		\While{$x_e > r.\textbf{x}$}{
			\textbf{destroy} any $m \in r.Mongods$ where $m.ParentSlave$ is $\textbf{x} \land \textbf{disabled}$\;
			$\textbf{x}_e \minuseq 1$
		}
		
		\Invariant minimum number of disabled slaves are running members of $r$\;
		
		\While{$\textbf{x}_e > r.\textbf{x}$}{
			\textbf{destroy} any $m \in r.Mongods$ where $m.ParentSlave$ is $\textbf{x}$\;
			$\textbf{x}_e \minuseq 1$
		}
		
		\Invariant at most $r.\textbf{x}$ members 
	}
	
	\Invariant \textit{desired state}: at most (r.p|r.v) member processes of $r$\;
	// desired state = the state that will be deployed
	
}
\end{algorithm}

\begin{algorithm}
\caption{Count recovering and active members of a Replica Set}

\Input{ReplicaSet $r$}
\Output{Number of $p_a$ and $v_a$ member processes of $r$ that \begin{itemize}
		\item are \textbf{recovering}, i.e. soon-to-be fully operational
		\item fully operational (\textbf{running})
	\end{itemize} and should actually be in that state.}
\BlankLine
\Fn{AlreadyAddedMemberCount(r ReplicaSet)}{
	$p_a, v_a = 0$
	
	\For{$m \in \text{r.Mongods}$}{
		\If{$\text{m.ParentSlave.ConfiguredState} != Disabled$ \\
			$\land \text{m.DesiredState.ExecutionState} != NotRunning$}{ 
			
			\uIf{$m.ParentSlave.PersistentStorage$}{
				$p_a \pluseq 1 $
			}\Else{
			$v_a \pluseq 1 $
		}
	}
}
\Return $p_a, v_a$
}
\end{algorithm}

\begin{algorithm}
\caption{Spawn Mongods on under-provisioned Replica Sets respecting RiskGroup \& p/v constraints.}

$p_a, v_a \gets $ AlreadyAddedMemberCount(r)

\ForEach(// same for persistent and volatile){$\textbf{x}_a \in \{p_a, v_a\}$}{

	%# TODO need to fill the queue. only ReplicaSets which 'need' (use p_* to define what need means) are in  the queue 
	$RQ \gets$ PriorityQueue(R)\;% "relative amount of missing members")\;
	$RGSQ \gets$ map[RiskGroup]PriorityQueue(Slaves of RiskGroup)\;%, "relative amount of available Mongod ports")\;
	
	\While{$r = RQ.pop()$; $r \neq nil$}{
	
		\Invariant $r$ needs a member of type $\textbf{x}$
		
		// find $x$-slave $s$ in a risk group $g$ with no other Mongod of $r$ in $g$\;
		$candidates \gets \{g \mid g \in RGSQ.keys \land \text{r has no member in g}\}$\;
		$g \gets$ ARGMAX($g \in candidates$, RGSQ[g].PeekPriority())\;
		\uIf{$g \neq nil$}{
			$s \gets RGSQ[g].pop()$\;
			\Invariant $s$ has at least one free port\;
			spawn new Mongod $m$ on $s$ and add it to $r.Mongods$\;
			compute $MongodState$ for $m$ and set the $DesiredState$ variable\;
			
			\If{$r$ needs another member of type $\textbf{x}$}{
				$RQ.insert(r)$ // recompute priority
			}
			
			\If{$s$ has free ports}{
				$RGSQ[g].insert(s)$ // recompute priority
			}
			
			
		} \Else{
			yield error\;
			continue\;
		}
		
	
	}

}

\end{algorithm}


}
\end{center}
\makeatother
\end{titlepage}
\newpage
\tableofcontents
\newpage

% Args: Method, URI, 
\newcommand{\apicall}[7][]{
	\subsubsection{\uppercase{#1} - #2}
	\begin{description}[leftmargin=!,labelwidth=\widthof{\bfseries Return Codesaa}] %aa intentioal -  more distance
		\item[Parameters]
		\begin{tabularx}{\linewidth}{c|c|*1{>{\centering\arraybackslash}X}@{}}
			\textbf{Parameter} & \textbf{Data Type} & \textbf{Description}\\
			\hline
			#3
		\end{tabularx}
		
		\item[Returns]
		\begin{description}[leftmargin=!,labelwidth=\widthof{\bfseries Status Codeaa}] %aa intentioal -  more distance
			\item[Status Code] #4
			\item[Headers] \begin{tabularx}{\linewidth}{c|*1{>{\centering\arraybackslash}X}@{}}
				\textbf{HTTP Header} & \textbf{Content}\\
				\hline
				#5
			\end{tabularx}
			\item[Body] #6
		\end{description}
		
		\item[Error Codes]
		\begin{tabularx}{\linewidth}{c|*1{>{\centering\arraybackslash}X}@{}}
			\textbf{HTTP Status Code} & \textbf{Reason}\\
			\hline
			#7
		\end{tabularx}
	\end{description}
}

\section{Master API}
\subsection{slaves}
\apicall[put]{/slave}
	{slave & slave object & slave to add}
	{201}
	{Location & Reference to added slave}
	{}
	{400 & Invalid parameter}


\glsaddallunused
\makeatletter
\newglossarystyle{myAltlist}{
  \glossarystyle{altlist} % base this style on altlist
  \renewcommand*{\glossaryentryfield}[5]{
  \item[\glsentryitem{##1}\glstarget{##1}{##2}]
    \mbox{}\par\nobreak\@afterheading
    ##3\glspostdescription\space On page ##5.
  }
}
\makeatother
\printglossary[type=main, title={Glossary}, toctitle={Glossary}, style=myAltlist]

\end{document}
