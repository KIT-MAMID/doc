%% das Papierformat zuerst
\documentclass[a4paper, 11pt]{article}
\usepackage[margin=3cm]{geometry}
\usepackage[utf8]{inputenc}
\usepackage[T1]{fontenc}
\usepackage[fleqn]{amsmath} %left aligned equations
\usepackage{hyperref} % clickable refs
\usepackage{graphicx}
\usepackage[toc, numberedsection]{glossaries}
\usepackage{float}
\usepackage{amssymb}
\usepackage{calc}
\usepackage{enumitem} 
\usepackage{url}
\usepackage{parskip}
% http://tex.stackexchange.com/questions/17730/newcommand-and-spacing
\usepackage{xspace}
\usepackage{xparse}
% 'frame' option for figures
\usepackage[export]{adjustbox}
% fancy page headers
\usepackage{fancyhdr}
\usepackage{xcolor}
\usepackage{tabularx}
\usepackage{listings}
\usepackage{hyperref}
\usepackage{xstring}
\usepackage[lined,boxed,commentsnumbered]{algorithm2e}
\usepackage{makecell}

% page style
\pagestyle{fancy}
\fancyhead[R]{\includegraphics[width=2cm]{../assets/kitlogo}}
\fancyhead[L]{\leftmark}
\fancypagestyle{plain}{
	\rhead{\includegraphics[width=2cm]{../assets/kitlogo}}
	\lhead{\leftmark}
}

% TODO: template übersetzten

\makeglossaries

%Hack for referencing labels
\makeatletter
\def\namedlabel#1#2{\begingroup
    #2%
    \def\@currentlabel{#2}%
    \phantomsection\label{#1}\endgroup
}
\makeatother
% End: Hack for referencing labels

% Glossar: alle Einträge, aber ohne extra Referenzen
% http://tex.stackexchange.com/questions/115635/glossaries-suppress-pages-when-using-glsaddall
\newcommand*{\glsgobblenumber}[1]{}
\makeatletter
\newcommand*{\glsaddnp}[2][]{
  \glsdoifexists{#2}{
    \def\@glsnumberformat{glsgobblenumber}
    \edef\@gls@counter{\csname glo@#2@counter\endcsname}
    \setkeys{glossadd}{#1}
    \@gls@saveentrycounter
    \@do@wrglossary{#2}
  }
}
\renewcommand{\glsaddallunused}[1][]{
  \edef\@glo@type{\@glo@types}
  \setkeys{glossadd}{#1}
  \forallglsentries[\@glo@type]{\@glo@entry}{
    \ifglsused{\@glo@entry}{}{
      \glsaddnp[#1]{\@glo@entry}}}
}
\makeatother

\renewcommand{\glsnamefont}[1]{\mdseries #1} % glossary entries shouldn’t be bold

% Glossar

% So sieht ein Glossar-Eintrag aus:
%
%\newglossaryentry{dijkstra}{
%  name={Dijkstra’s Algorithmus},
%  description={ein Algorithmus, um den optimalen Pfad in einem gerichteten Graphen zu finden}
%}
%\newglossaryentry{arc}{
%  name={Arc-Flags},
%  description={eine Technik, um Routenberechnung zu beschleunigen},
%  see={dijkstra}
%}
%
% Und so kann er im Dokument verwendet werden:
%
% lorem ipsum dolor sit \gls{arc}, consectetur
%
% End: Glossar

% usage: \counteditem{prefix}{refName} -> item `/prefixXX/` with label `prefix:refName` (where XX is counted in increments of 10)
\makeatletter
\newcommand{\oitem}[2]{
  % define the counter
  \@ifundefined{c@oitem#1}{\newcounter{oitem#1}}{} % initialized at 0
  \addtocounter{oitem#1}{10}
  \item[\namedlabel{#1:#2}{/#1\arabic{oitem#1}/}]
}
\makeatother

% new page after section
\let\oldsection\section
\renewcommand\section{\clearpage\oldsection}

\newcommand{\mamidscreenshot}[1]{\includegraphics[width=\textwidth,frame]{#1}}

\newcommand{\uiel}[3]{\item \textbf{"#1" #2:} #3}

\begin{document}

% place a symbol before clickable links
% this has to come *after* \begin{document} because hyperref installs a \AtBeginDocument hook that updates the ref command.
\newcommand{\refsymbol}[0]{\scalebox{0.5}{$\nearrow$}}
\let\oldref\ref
\renewcommand{\ref}[1]{\refsymbol\oldref{#1}}
\let\oldgls\gls
\renewcommand{\gls}[1]{\refsymbol\oldgls{#1}}
\let\oldGls\Gls
\renewcommand{\Gls}[1]{\refsymbol\oldGls{#1}}
\let\oldglspl\glspl
\renewcommand{\glspl}[1]{\refsymbol\oldglspl{#1}}
\let\oldGlspl\Glspl
\renewcommand{\Glspl}[1]{\refsymbol\oldGlspl{#1}}
\let\oldglslink\glslink
\renewcommand{\glslink}[2]{\refsymbol\oldglslink{#1}{#2}}
\let\oldhyperref\hyperref
\renewcommand{\hyperref}[2][notActuallyOptional]{\refsymbol\oldhyperref[#1]{#2}}
\let\oldautoref\autoref
\renewcommand{\autoref}[1]{\refsymbol\oldautoref{#1}}

\newcommand{\abbildung}[1]{\autoref{fig:#1}}
\newcommand{\mamid}{\textit{MAMID}\xspace}


%http://tex.stackexchange.com/questions/43002/how-to-preserve-the-same-parskip-in-minipage
\newlength{\currentparskip}
\newenvironment{minipageparskip}
{
        \setlength{\currentparskip}{\parskip}% save the value
	\begin{minipage}{\textwidth}% open the minipage
	\setlength{\parskip}{\currentparskip}% restore the value
} {
        \end{minipage}
}

\NewDocumentCommand{\refgo}{m}{%
  \IfHasLabel{#1}{%
    \hyperref[#1]{\codeinline{\refgosymboltext{#1}}}%
  }{%
   % \codeinline{#1}%
   \colorbox{red}{\texttt{#1}}%
  }%
  \let\refdisplaytext\undefined%
}

% used to produce symbol names relative to the current scope
\NewDocumentCommand{\refgosymboltext}{m}{%
  % nibble the scope hierarchy: .gocurpackage.gocurtype.gocurmethod%
  \StrLen{\gocurpackage.}[\gocurlen]%
  \StrLeft{#1}{\gocurlen}[\refgosymbolCurComponent]%
  \IfStrEq{\refgosymbolCurComponent}{\gocurpackage.}{%
    % we are in scope of the package  %
      \StrGobbleLeft{#1}{\gocurlen}[\gocurremainingpath]%
      %
      \StrLen{\gocurtype.}[\gocurlen]%
      \StrLeft{\gocurremainingpath}{\gocurlen}[\refgosymbolCurComponent]%
      \IfStrEq{\refgosymbolCurComponent}{\gocurtype.}{%
        % we are in class scope       %
        \StrGobbleLeft{\gocurremainingpath}{\gocurlen}[\gocurremainingpath]%
        %
        \StrLen{\gocurmethod.}[\gocurlen]%
        \StrLeft{\gocurremainingpath}{\gocurlen}[\refgosymbolCurComponent]%
        \IfStrEq{\refgosymbolCurComponent}{\gocurmethod.}{%
          %
          \StrGobbleLeft{\gocurremainingpath}{\gocurlen}[\gocurremainingpath]%
          \gocurremainingpath%
        }{%
          \gocurremainingpath%
        }%
      }{%
        \gocurremainingpath%
      }%
  }{%
    % we are not in scope of the package%
    #1%
  }%
}

%%%%%%%%%%%%%%%%%%%%%%%%%%%%%%%%%%%%%%%%%%%%%%%%%%%%%%%%%%%%%%%%%%%%%%%%%%%%%%%%
%                       Util-commands that are dirty hacks
%                       to enable the helper commands

% evaluate #2 if the label #1 exists, else #3.
\makeatletter
\newcommand{\IfHasLabel}[3]{%
  \@ifundefined{r@#1}% declaring a label xyz defines r@xyz
               {#3}%
               {#2}%
}
\makeatother

% Styling of inline 'code', e.g. types.
\definecolor{codeinline-gray}{gray}{0.95}
\NewDocumentCommand{\codeinline}{m}{%
  % Specify inline code / command
  % Markdown equivalent: `#1`
  \colorbox{codeinline-gray}{\texttt{#1}}%
}

%%%%%%%%%%%%%%%%%%%%%%%%%%%%%%%%%%%%%%%%%%%%%%%%%%%%%%%%%%%%%%%%%%%%%%%%%%%%%%%%

%%%%%%%%%%%%%%%%%%%%%%%%%%%%%%%%%%%%%%%%%%%%%%%%%%%%%%%%%%%%%%%%%%%%%%%%%%%%%%%%
%                               Helper Commands
% ... defining types, methods, paramters, etc

% redefine \gocurpackage to the package name that the
%   subsequent \struct, \property, \method, etc belong too
\newcommand{\gocurpackage}{none}
\newcommand{\gocurtype}{}
\newcommand{\gocurmethod}{}

% an abstraction for declaring types, i.e. \struc or \interface
% #1: the type of type, i.e Struct or Interface
% #2: the name of the type
% #3: prologue
% #4: properties. list of \property
% #5: methods. list of \method
%
\NewDocumentCommand{\gotype}{m m +m +g +g +d<>}{%
  \subsection{#1 \codeinline{#2}}\label{\gocurpackage.#2}%
  % Make the current type name available in the scope of the class.
  % Useful for making labels.
  \renewcommand{\gocurtype}[0]{#2}% 
  #3%
  \IfValueT{#5}{%
    \subsubsection*{Fields}%
    \begin{description}%
      #5% list of \property
    \end{description}%
   }%
   \IfValueT{#4}{%
      \subsubsection*{Methods}%
      \begin{description}%
        %\setlength{\itemsep}{1em}
        #4%list of \method
      \end{description}%
   }%
   \IfValueT{#6}{%
      \subsection*{Enum Values}%
      \begin{description}%
          #6% list of \goenumitem
      \end{description}%
   }%
   \renewcommand\gocurtype[0]{}%
}
\NewDocumentCommand{\struct}{m +m +g +g}{\gotype{Struct}{#1}{#2}{#4}{#3}}
\NewDocumentCommand{\interface}{m +m +m}{\gotype{Interface}{#1}{#2}{#3}}
\NewDocumentCommand{\goenum}{m +m +m}{\gotype{Enum}{#1}{#2}<#3>}
\NewDocumentCommand{\reftype}{m}{\refgo{\gocurpackage.#1}}

% \property is used to declare fields of a struct
\NewDocumentCommand{\property}{m o +m}{
  \item[\codeinline{#1\hspace{0.1cm}:\hspace{0.1cm}#2}]\label{\gocurpackage.\gocurtype.#1}\hfill %(line break for the poor)
  \par
  #3
}
\NewDocumentCommand{\refproperty}{m}{\refgo{\gocurpackage.\gocurtype.#1}}

\NewDocumentCommand{\goenumitem}{m +m}{
  \item [\codeinline{#1}]\label{\gocurpackage.\gocurtype.#1} #2
}

% \method is used to declare a struct's method set.
\NewDocumentCommand{\method}{m o d() +m}{
\renewcommand\gocurmethod[0]{#1}%
\item[\codeinline{#1(\IfValueT{#3}{$\cdot$})\IfValueT{#2}{\hspace{0.1cm}:\hspace{0.1cm}#2}}]\label{\gocurpackage.\gocurtype.#1}\hfill %(line break for the poor)
  \IfValueT{#3}{
    \begin{description}
      #3 % should only contain \param
    \end{description}
  }
  \par
  #4
  \renewcommand\gocurmethod[0]{}%
}
\NewDocumentCommand{\refmethod}{m}{
	\refgo{\gocurpackage.\gocurtype.#1}
}

% \param is used to declare paramters of a \method -> check \method for details
\NewDocumentCommand{\param}{m o +m}{%
  \item[\texttt{#1\hspace{0.1cm}:\hspace{0.1cm}#2}]\label{\gocurpackage.\gocurtype.\gocurmethod.#1} #3
}

\NewDocumentCommand{\refparam}{m}{
  \refgo{\gocurpackage.\gocurtype.\gocurmethod.#1}
}


%%%%%%%%%%%%%%%%%%%%%%%%%%%%%%%%%%%%%%%%%%%%%%%%%%%%%%%%%%%%%%%%%%%%%%%%%%%%%%%%

%%%%%%%%%%%%%%%%%%%%%%%%%%%%%%%%%%%%%%%%%%%%%%%%%%%%%%%%%%%%%%%%%%%%%%%%%%%%%%%%

%%%%%%%%%%%%%%%%%%%%%%%%%%%%%%%%%%%%%%%%%%%%%%%%%%%%%%%%%%%%%%%%%%%%%%%%%%%%%%%%

% alle Glossareintraege
\newacronym{gui}{GUI}{Graphical User Interface}
\newacronym{cli}{CLI}{Command Line Interface}

\newglossaryentry{cluster}{
  name={cluster},
  description={It's something.},
  plural={clusters}
}
\newglossaryentry{MongoDB}{
	name={MongoDB},
	description={It's something.},
	plural={MongoDB}
}
\newglossaryentry{replica set}{
	name={replica set},
	description={It's something.},
	plural={replica sets}
}
\newglossaryentry{administrator}{
	name={administrator},
	description={It's something.},
	plural={administrators}
}
\newglossaryentry{host}{
	name={host},
	description={It's something.},
	plural={hosts}
}
\newglossaryentry{slave}{
	name={slave},
	description={It's something.},
	plural={slaves}
}
\newglossaryentry{master}{
	name={master},
	description={It's something.},
	plural={masters}
}
\newglossaryentry{inventory}{
	name={inventory},
	description={It's something.},
	plural={inventories}
}
\newglossaryentry{disabled mode}{
	name={disabled mode},
	description={It's something.},
	plural={}
}
\newglossaryentry{physical interdependency}{
	name={physical interdependency},
	description={It's something.},
	plural={physical interdependencies}
}
\newglossaryentry{maintenance mode}{
	name={maintence mode},
	description={It's something.},
	plural={}
}
\newglossaryentry{persistent storage}{
	name={persistent storage},
	description={It's something.},
	plural={}
}
\newglossaryentry{volatile storage}{
	name={volatile storage},
	description={It's something.},
	plural={}
}
\newglossaryentry{root data directory}{
	name={root data directory},
	description={It's something.},
	plural={hosts}
}
\newglossaryentry{arbiter}{
	name={arbiter},
	description={It's something.},
	plural={hosts}
}
\newglossaryentry{degraded}{
	name={degraded},
	description={It's something.},
	plural={hosts}
}
\newglossaryentry{HTTP API}{
	name={HTTP API},
	description={It's something.},
	plural={hosts}
}
\newglossaryentry{active mode}{
	name={active mode},
	description={It's something.},
	plural={hosts}
}

\begin{titlepage}
\makeatletter
\begin{center}
~\\[4em]
{\Huge MAMID}\\[.8em]\huge{Monitor and Manager for In-memory Databases}\\[2em]
{\huge Design}\\[1em]
{\large\today}\\[2.5em]
{\LARGE
Niklas Fuhrberg\\
Anton Schirg\\
Christian Schwarz\\
Janis Streib\\
Bob Weinand\\[3em]}
{\Large supervised by}\\[2em]
{\LARGE
Dr Marek Szuba\\[1em]}
{\Large at}\\[1em]
{\LARGE
Karlsruhe Institute of Technology\\
SCC\\[2em]}
{\color{gray}
  \small Document Version: \newtheorem{theorem}{Theorem}


\section{master}

\renewcommand{\gocurpackage}{model}
\renewcommand{\gocurpackage}{master}

\struct{ClusterAllocator}{
  The \refstruct{ClusterAllocator} determines the layout of the cluster managed by \mamid.

  It attempts to fulfill the constraints defined through the model objects, in particular
  \begin{itemize}
    \item A \refgo{model.ReplicaSet}'s \refgo{VolatileNodeCount} \& \refgo{PersistentNodeCount}
    \item The \refgo{model.Slave}'s allowed number of Mongod instances
          (\refgo{MongodPortRangeBegin} to \refgo{MongodPortRangeEnd}).
    \item The \refgo{model.Slave.SlaveState}
    \item The configured \refgo{model.RiskGroup}s.
  \end{itemize}

  An iterative algorithm is employed to decide on a cluster layout described through
  \refgo{model.Mongod.DesiredState}s that
  \begin{itemize}
    \item attempts to fulfill the above constraints
    \item attempts an even distribution of Mongods on the different cluster hosts
    \item is a minimal change in comparison to the previous layout
  \end{itemize}

  \begin{theorem}{Idempotence of the ClusterAllocator}
    \label{theorem:idempotence_clusterallocator}
    Let $l$ be a layout of the cluster. Then $ClusterAllocator(ClusterAllocator(l)) = ClusterAllocator(l)$.
  \end{theorem}

}{
  \property{DB}[gorm.DB]{Initialized handle to the database.}
  \property{BusChannel}[chan interface{}]{Initialized channel to the application bus.} %TODO ref}
}{
  \method{LayoutCluster}{Lay out the cluster as described above.}
}

\subsubsection{Pseudocode}

The \refstruct{master.ClusterAllocator} is crucial to the stable operation of the \mamid-managed cluster.\\
Hence, it is worth defining the implementation of \refgo{master.ClusterAllocator.LayoutCluster} through pseudocode. %TODO parentheses

While studying the algorithms below, the reader should keep in mind that
\begin{itemize}
  \item changes in the cluster layout $\equiv$ change or creation of \refgo{model.Mongod.DesiredState} 
  \item the \refgo{model.Mongod.ObservedState} may change after an arbitrary amount or even never.\\
  \item changes to a ReplicaSet must not violate or further worsen the high-availibility constraints,
        in particular \refgo{model.ReplicaSet}'s \refgo{VolatileNodeCount} \& \refgo{PersistentNodeCount}\\
        $\impl$ Mongods in \refgo{model.MongodExecutionState.Recovering} are an important special-case.
\end{itemize}

\newcommand{\pluseq}{\mathrel{{+}{=}}}
\newcommand{\minuseq}{\mathrel{{-}{=}}}
\SetKwInOut{Input}{input}
\SetKwInOut{Output}{output}
\SetKwProg{Fn}{Function}{}{}
\SetKw{Invariant}{invariant}
\IncMargin{0.5em}

Invariant: Algorithm(Algorithm(x)) = Algorithm(x)

Otherwise oscillations could occur

Most interesting case: Set a slave to disabled. Then a new Mongod should be spawned and recover and when it is done the old disabled one can be deleted.

\begin{algorithm}

\caption{Count members of a Replica Set that are \& intended to be in stable state (running)}

\Input{ReplicaSet $r$}
\Output{Number of $p_e$ and $v_e$ member processes of $r$ that are fully operational and are planned to remain in that state.}
\BlankLine
\Fn{EffectiveMemberCount(r ReplicaSet)}{

$p_e, v_e = 0$

\For{$m \in \text{r.Mongods}$}{
	\If{$\text{m.ObservedState.ExecutionState} = Running$ \\ 	%	#this line evaluates to false if m.ObservedState = NULL
		$\land \text{m.DesiredState.ExecutionState} = Running$}{ %#this line evaluates to false if m.DesiredState = NULL 

		\uIf{$m.ParentSlave.PersistentStorage$}{
			$p_e \pluseq 1 $
		}\Else{
			$v_e \pluseq 1 $
		}
	}
}
\Return $p_e, v_e$
}
\end{algorithm}


\begin{algorithm}
\caption{Destroy members of disabled replica sets where possible without violating p/v constraints.}
\ForEach{r in ReplicaSets}{
	
	$p_e, v_e \gets $ EffectiveRunningMembers(r)
	
	\ForEach(// same for persistent and volatile){$\textbf{x}_e \in \{p_e, v_e\}$}{
		\While{$x_e > r.\textbf{x}$}{
			\textbf{destroy} any $m \in r.Mongods$ where $m.ParentSlave$ is $\textbf{x} \land \textbf{disabled}$\;
			$\textbf{x}_e \minuseq 1$
		}
		
		\Invariant minimum number of disabled slaves are running members of $r$\;
		
		\While{$\textbf{x}_e > r.\textbf{x}$}{
			\textbf{destroy} any $m \in r.Mongods$ where $m.ParentSlave$ is $\textbf{x}$\;
			$\textbf{x}_e \minuseq 1$
		}
		
		\Invariant at most $r.\textbf{x}$ members 
	}
	
	\Invariant \textit{desired state}: at most (r.p|r.v) member processes of $r$\;
	// desired state = the state that will be deployed
	
}
\end{algorithm}

\begin{algorithm}
\caption{Count recovering and active members of a Replica Set}

\Input{ReplicaSet $r$}
\Output{Number of $p_a$ and $v_a$ member processes of $r$ that \begin{itemize}
		\item are \textbf{recovering}, i.e. soon-to-be fully operational
		\item fully operational (\textbf{running})
	\end{itemize} and should actually be in that state.}
\BlankLine
\Fn{AlreadyAddedMemberCount(r ReplicaSet)}{
	$p_a, v_a = 0$
	
	\For{$m \in \text{r.Mongods}$}{
		\If{$\text{m.ParentSlave.ConfiguredState} != Disabled$ \\
			$\land \text{m.DesiredState.ExecutionState} != NotRunning$}{ 
			
			\uIf{$m.ParentSlave.PersistentStorage$}{
				$p_a \pluseq 1 $
			}\Else{
			$v_a \pluseq 1 $
		}
	}
}
\Return $p_a, v_a$
}
\end{algorithm}

\begin{algorithm}
\caption{Spawn Mongods on under-provisioned Replica Sets respecting RiskGroup \& p/v constraints.}

$p_a, v_a \gets $ AlreadyAddedMemberCount(r)

\ForEach(// same for persistent and volatile){$\textbf{x}_a \in \{p_a, v_a\}$}{

	%# TODO need to fill the queue. only ReplicaSets which 'need' (use p_* to define what need means) are in  the queue 
	$RQ \gets$ PriorityQueue(R)\;% "relative amount of missing members")\;
	$RGSQ \gets$ map[RiskGroup]PriorityQueue(Slaves of RiskGroup)\;%, "relative amount of available Mongod ports")\;
	
	\While{$r = RQ.pop()$; $r \neq nil$}{
	
		\Invariant $r$ needs a member of type $\textbf{x}$
		
		// find $x$-slave $s$ in a risk group $g$ with no other Mongod of $r$ in $g$\;
		$candidates \gets \{g \mid g \in RGSQ.keys \land \text{r has no member in g}\}$\;
		$g \gets$ ARGMAX($g \in candidates$, RGSQ[g].PeekPriority())\;
		\uIf{$g \neq nil$}{
			$s \gets RGSQ[g].pop()$\;
			\Invariant $s$ has at least one free port\;
			spawn new Mongod $m$ on $s$ and add it to $r.Mongods$\;
			compute $MongodState$ for $m$ and set the $DesiredState$ variable\;
			
			\If{$r$ needs another member of type $\textbf{x}$}{
				$RQ.insert(r)$ // recompute priority
			}
			
			\If{$s$ has free ports}{
				$RGSQ[g].insert(s)$ // recompute priority
			}
			
			
		} \Else{
			yield error\;
			continue\;
		}
		
	
	}

}

\end{algorithm}


}
\end{center}
\makeatother
\end{titlepage}
\newpage
\tableofcontents
\newpage

\section {Formatting \& Legend}\label{legend}

% NOTE Declaring the current package
\renewcommand{\gocurpackage}{legend}

This is some text about the package.

\struct{AStruct}{
    Overall description of \refgo{legend.AStruct}.
  
    Multi-paragraph descriptions are supported.
}{
    \property{variable}[Type]{The description of the struct's field \emph{variable}}
    \property{fieldtwo}[Type2]{
      The description of the struct's field \refgo{legend.AStruct.fieldtwo}.
  
      Multi-paragraph (\textbackslash long) descriptions are supported.
    }
}{
    \method{Methodname}[ReturnType]{Method without parameters}
    \method{Methodname2}[ReturnType](%% NOTE the parentheses ()
        \param{pname}[ParamType]{Description of the parameter}
        \param{p2name}[ParamType]{
          Description of the parameter.
          Cannot be multi-paragraph. 
        }
    ){
        Description of the method \refgo{legend.AStruct.Methodname2}.
        
        We can \emph{reference} parameters using absolute identifiers \refgo{legend.AStruct.Methodname2.pname} or relative identifiers \refgoparam{p2name}.
    }
}

\interface{AnInterface}{
  Overall description of the Interface. An interface has no field, hence there is no section for fields.

  Multi-paragraph descriptions are supported. References to other types, e.g. \refgo{legend.AStruct} are possible.
}{
  \method{Methodname}[ReturnType]{Method without parameters}
  \method{Methodname2}[ReturnType]( %% NOTE the parentheses ()
    \param{pname}[ParamType]{Description of the parameter}
    \param{p2name}[ParamType]{
      Description of the parameter.
      Cannot be multi-paragraph. 
    }
  ){
    Description of the method \refgo{legend.AnInterface.Methodname}.
   
    Multi-paragraph descriptions are supported.

    Referencing the package: 
    \refgo{legend}
  
    Referencing the interface: 
    \refgo{legend.AnInterface}
  
    Referencing the interface variable: 
    \refgo{legend.AnInterface.variable}
  
    Referencing the method: 
    \refgo{legend.AnInterface.Methodname2}
  
    Referencing the method parameter: 
    \refgo{legend.AnInterface.Methodname2.pname}



  }
}


\section{MasterSlaveProtocol}
\renewcommand{\gocurpackage}{msp}

\begin{figure}[H]
	\includegraphics[width=\textwidth]{msp_classes}
	\caption{MasterSlaveProtocol}
\end{figure}

The MasterSlaveProtocol \refgo{msp} implements communication between \refgo{slave} and \refgo{master}.

% server (slave) side structures
\struct{Listener}{
  Structure for handling communication through the MasterSlaveProtocol on the \refgo{slave} side.
}{
  \property{Consumer}[\reftype{Consumer}]{
    Reference to an event handler, typically a controller.
  }
}{
  \method{Run}{
    Start listening for connections using the MasterSlaveProtocol.
    \begin{itemize}
      \item Listen for incoming connections from the \refgo{master}.
      \item Decode / Encode the transport format.
      \item Validate incoming request format.
      \item Delegate valid requests to the \reftype{Consumer}.
    \end{itemize}
  }
}

\interface{Consumer}{
  A Consumer handles MasterSlaveProtocol requests received by an instance of \reftype{Listener}.
}{
  \method{RequestStatus}[(\reftype{SlaveStatus}, \reftype{SlaveError})]{
    %TODO reference request message object here -> proto specification
    Request to provide a status report of the slave's state.
    See \reftype{Mongod} for details on the data returned by this method.
  }
  \method{EstablishMongodState}[*\reftype{SlaveError}](
    \param{m}[\reftype{Mongod}]{State description of a mongod instance that shall be established}
  ){
    %TODO reference request message object here -> proto specification
    Request to establish the configuration state $m$ of a MongoDB process on the slave.
    May return an error in case the state could not be established.
  }
}


% client (master) side structures
\struct{Client}{
  Structure for handling communication through the MasterSlaveProtocol on the \refgo{master} side.
}{
  \property{Target}[\reftype{HostPort}]{Remote endpoint it holds a connection to}
}{
  \method{Connect}(
    \param{target}[\reftype{HostPort}]{Remote endpoint to connect to}
  ){
    Establish the connection to the server on the \refgo{slave} side.
  }
  \method{RequestStatus}[([]\reftype{Mongod}, \reftype{error})]{
    Send a status request to the \refgo{slave} to retrieve all the running \reftype{Mongod} instances. May return an instance of \reftype{error} in case overall retrieving failed.
  }
  \method{EstablishMongodState}[*\reftype{CommunicationError}](
    \param{m}[[]\reftype{Mongod}]{Array of \reftype{Mongod} instances to establish}
  ){
    Sends the array of \reftype{Mongod} instances to the \refgo{slave} which is tasked with creating and configuring the described processes.
  }
}


% protocol specific structures

\section{Slave}
\beginpackage{slave}

\begin{figure}[H]
	\includegraphics[width=\textwidth]{slave_classes}
	\caption{Slave}
\end{figure}

The \refgo{slave} is installed on the individual host nodes and started by the main service manager of the operating system during the boot process.

\subsection{Application / main()}{
  A Go application starts by initializing the Go 'main' package and then running the \textit{main()} function.
  This function performs early initialization of the slave's main datastructures:
  \begin{itemize}
    \item Parse command line flags using an external CLIFlagsParser.
    \item Initialize the MSPListener.
    \item Initialize the Controller with the MSPListener instance.
    \item Transfer control to MSPListener to wait for incoming connections.
  \end{itemize}
}

\struct{Controller}{
  Handles MSP requests by implementing the \refgo{msp.Consumer} interface.
  Hence, it coordinates the work required to fulfill requests from the master.
  Most importantly, it leverages the \reftype{ProcessManager} and \reftype{MongodConfigurator} to spawn and configure instances of MongoDB.
}{
  \property{BusyTable}[map[\refgo{msp.PortNumber}]bool]{
    Contains a boolean busy state per instance of MongoDB.\\
    \refproperty{BusyTable} must only be altered from a goroutine holding the \refproperty{busyTableLock}.\\
    \vspace{-0.4em}\\ % dirty hack because we do not support paragraphs in property descriptions
    \refgo{msp.Consumer.EstablishMongodState} may be called repeatedly by \refgo{master}
    while an earlier state establishment request is still executing.\\
    However, a MongoDB instance must not be configured concurrently.\\
    Hence, \refproperty{BusyTable} in combination with \refproperty{busyTableLock} is used to ensure sequential configuration.
  }
  \property{busyTableLock}[sync.Mutex]{
    Mutex controlling access to \refproperty{BusyTable}.
  }
}

\struct{ProcessManager}{
  Starts processes, may provide a list of alive processes and eventually kill these.
}{
  \property{runningProcesses}[map[\refgo{msp.PortNumber}]*exec.Cmd]{Holds the active process controls per instance}
}{
  \method{spawnProcess}[*\refgo{msp.SlaveError}](
    \param{m}[\refgo{msp.Mongod}]{Mongod information about what exactly to spawn}
    \param{dataDir}[string]{Root directory of MongoDB data}
  ){
    Spawns a Mongod process as requested by the given Mongod inside the data root directory
  }
  \method{runningProcesses}[[]\refgo{msp.PortNumber}]{
    Returns the PortNumbers of the currently running processes
  }
  \method{killProcess}[*\refgo{msp.SlaveError}](
    \param{p}[\refgo{msp.PortNumber}]{PortNumber to identify process}
  ){
    Kills process by given \refgo{msp.PortNumber}. Does not error if the process is already dead, only if it could not be killed.
  }
  \method{killProcesses}[*\refgo{msp.SlaveError}]{
    Kills all remaining Processes. Errors if it could not kill some processes.
  }
}

\interface{MongodConfigurator}{
  Applies or returns configuration of a mongod instance by PortNumber.
}{
  \method{MongodConfiguration}[{(\refgo{msp.Mongod}, \refgo{msp.SlaveError})}](
    \param{p}[\refgo{msp.PortNumber}]{PortNumber to connect to locally}
  ){
    Reads configuration from a local MongoDB instance
  }
  \method{ApplyMongodConfiguration}[*\refgo{msp.SlaveError}](
    \param{config}[\refgo{msp.Mongod}]{Configuration to apply}
  ){
    Connects to the local MongoDB instance given in \refparam{config} parameter and applies the configuration
  }
}

\struct{ConcreteMongodConfigurator}{
  Implements MongodConfigurator interface and uses an external MongoDB client to communicate with local instances
}
\newtheorem{theorem}{Theorem}


\section{master}

\renewcommand{\gocurpackage}{model}
\renewcommand{\gocurpackage}{master}

\struct{ClusterAllocator}{
  The \refstruct{ClusterAllocator} determines the layout of the cluster managed by \mamid.

  It attempts to fulfill the constraints defined through the model objects, in particular
  \begin{itemize}
    \item A \refgo{model.ReplicaSet}'s \refgo{VolatileNodeCount} \& \refgo{PersistentNodeCount}
    \item The \refgo{model.Slave}'s allowed number of Mongod instances
          (\refgo{MongodPortRangeBegin} to \refgo{MongodPortRangeEnd}).
    \item The \refgo{model.Slave.SlaveState}
    \item The configured \refgo{model.RiskGroup}s.
  \end{itemize}

  An iterative algorithm is employed to decide on a cluster layout described through
  \refgo{model.Mongod.DesiredState}s that
  \begin{itemize}
    \item attempts to fulfill the above constraints
    \item attempts an even distribution of Mongods on the different cluster hosts
    \item is a minimal change in comparison to the previous layout
  \end{itemize}

  \begin{theorem}{Idempotence of the ClusterAllocator}
    \label{theorem:idempotence_clusterallocator}
    Let $l$ be a layout of the cluster. Then $ClusterAllocator(ClusterAllocator(l)) = ClusterAllocator(l)$.
  \end{theorem}

}{
  \property{DB}[gorm.DB]{Initialized handle to the database.}
  \property{BusChannel}[chan interface{}]{Initialized channel to the application bus.} %TODO ref}
}{
  \method{LayoutCluster}{Lay out the cluster as described above.}
}

\subsubsection{Pseudocode}

The \refstruct{master.ClusterAllocator} is crucial to the stable operation of the \mamid-managed cluster.\\
Hence, it is worth defining the implementation of \refgo{master.ClusterAllocator.LayoutCluster} through pseudocode. %TODO parentheses

While studying the algorithms below, the reader should keep in mind that
\begin{itemize}
  \item changes in the cluster layout $\equiv$ change or creation of \refgo{model.Mongod.DesiredState} 
  \item the \refgo{model.Mongod.ObservedState} may change after an arbitrary amount or even never.\\
  \item changes to a ReplicaSet must not violate or further worsen the high-availibility constraints,
        in particular \refgo{model.ReplicaSet}'s \refgo{VolatileNodeCount} \& \refgo{PersistentNodeCount}\\
        $\impl$ Mongods in \refgo{model.MongodExecutionState.Recovering} are an important special-case.
\end{itemize}

\newcommand{\pluseq}{\mathrel{{+}{=}}}
\newcommand{\minuseq}{\mathrel{{-}{=}}}
\SetKwInOut{Input}{input}
\SetKwInOut{Output}{output}
\SetKwProg{Fn}{Function}{}{}
\SetKw{Invariant}{invariant}
\IncMargin{0.5em}

Invariant: Algorithm(Algorithm(x)) = Algorithm(x)

Otherwise oscillations could occur

Most interesting case: Set a slave to disabled. Then a new Mongod should be spawned and recover and when it is done the old disabled one can be deleted.

\begin{algorithm}

\caption{Count members of a Replica Set that are \& intended to be in stable state (running)}

\Input{ReplicaSet $r$}
\Output{Number of $p_e$ and $v_e$ member processes of $r$ that are fully operational and are planned to remain in that state.}
\BlankLine
\Fn{EffectiveMemberCount(r ReplicaSet)}{

$p_e, v_e = 0$

\For{$m \in \text{r.Mongods}$}{
	\If{$\text{m.ObservedState.ExecutionState} = Running$ \\ 	%	#this line evaluates to false if m.ObservedState = NULL
		$\land \text{m.DesiredState.ExecutionState} = Running$}{ %#this line evaluates to false if m.DesiredState = NULL 

		\uIf{$m.ParentSlave.PersistentStorage$}{
			$p_e \pluseq 1 $
		}\Else{
			$v_e \pluseq 1 $
		}
	}
}
\Return $p_e, v_e$
}
\end{algorithm}


\begin{algorithm}
\caption{Destroy members of disabled replica sets where possible without violating p/v constraints.}
\ForEach{r in ReplicaSets}{
	
	$p_e, v_e \gets $ EffectiveRunningMembers(r)
	
	\ForEach(// same for persistent and volatile){$\textbf{x}_e \in \{p_e, v_e\}$}{
		\While{$x_e > r.\textbf{x}$}{
			\textbf{destroy} any $m \in r.Mongods$ where $m.ParentSlave$ is $\textbf{x} \land \textbf{disabled}$\;
			$\textbf{x}_e \minuseq 1$
		}
		
		\Invariant minimum number of disabled slaves are running members of $r$\;
		
		\While{$\textbf{x}_e > r.\textbf{x}$}{
			\textbf{destroy} any $m \in r.Mongods$ where $m.ParentSlave$ is $\textbf{x}$\;
			$\textbf{x}_e \minuseq 1$
		}
		
		\Invariant at most $r.\textbf{x}$ members 
	}
	
	\Invariant \textit{desired state}: at most (r.p|r.v) member processes of $r$\;
	// desired state = the state that will be deployed
	
}
\end{algorithm}

\begin{algorithm}
\caption{Count recovering and active members of a Replica Set}

\Input{ReplicaSet $r$}
\Output{Number of $p_a$ and $v_a$ member processes of $r$ that \begin{itemize}
		\item are \textbf{recovering}, i.e. soon-to-be fully operational
		\item fully operational (\textbf{running})
	\end{itemize} and should actually be in that state.}
\BlankLine
\Fn{AlreadyAddedMemberCount(r ReplicaSet)}{
	$p_a, v_a = 0$
	
	\For{$m \in \text{r.Mongods}$}{
		\If{$\text{m.ParentSlave.ConfiguredState} != Disabled$ \\
			$\land \text{m.DesiredState.ExecutionState} != NotRunning$}{ 
			
			\uIf{$m.ParentSlave.PersistentStorage$}{
				$p_a \pluseq 1 $
			}\Else{
			$v_a \pluseq 1 $
		}
	}
}
\Return $p_a, v_a$
}
\end{algorithm}

\begin{algorithm}
\caption{Spawn Mongods on under-provisioned Replica Sets respecting RiskGroup \& p/v constraints.}

$p_a, v_a \gets $ AlreadyAddedMemberCount(r)

\ForEach(// same for persistent and volatile){$\textbf{x}_a \in \{p_a, v_a\}$}{

	%# TODO need to fill the queue. only ReplicaSets which 'need' (use p_* to define what need means) are in  the queue 
	$RQ \gets$ PriorityQueue(R)\;% "relative amount of missing members")\;
	$RGSQ \gets$ map[RiskGroup]PriorityQueue(Slaves of RiskGroup)\;%, "relative amount of available Mongod ports")\;
	
	\While{$r = RQ.pop()$; $r \neq nil$}{
	
		\Invariant $r$ needs a member of type $\textbf{x}$
		
		// find $x$-slave $s$ in a risk group $g$ with no other Mongod of $r$ in $g$\;
		$candidates \gets \{g \mid g \in RGSQ.keys \land \text{r has no member in g}\}$\;
		$g \gets$ ARGMAX($g \in candidates$, RGSQ[g].PeekPriority())\;
		\uIf{$g \neq nil$}{
			$s \gets RGSQ[g].pop()$\;
			\Invariant $s$ has at least one free port\;
			spawn new Mongod $m$ on $s$ and add it to $r.Mongods$\;
			compute $MongodState$ for $m$ and set the $DesiredState$ variable\;
			
			\If{$r$ needs another member of type $\textbf{x}$}{
				$RQ.insert(r)$ // recompute priority
			}
			
			\If{$s$ has free ports}{
				$RGSQ[g].insert(s)$ // recompute priority
			}
			
			
		} \Else{
			yield error\;
			continue\;
		}
		
	
	}

}

\end{algorithm}


% Args: Method, URI, 
\newcommand{\apicall}[7][]{
	\subsubsection{\uppercase{#1} - #2}
	\begin{description}[leftmargin=!,labelwidth=\widthof{\bfseries Return Codesaa}] %aa intentioal -  more distance
		\item[Parameters]
		\begin{tabularx}{\linewidth}{c|c|*1{>{\centering\arraybackslash}X}@{}}
			\textbf{Parameter} & \textbf{Data Type} & \textbf{Description}\\
			\hline
			#3
		\end{tabularx}
		
		\item[Returns]
		\begin{description}[leftmargin=!,labelwidth=\widthof{\bfseries Status Codeaa}] %aa intentioal -  more distance
			\item[Status Code] #4
			\item[Headers] \begin{tabularx}{\linewidth}{c|*1{>{\centering\arraybackslash}X}@{}}
				\textbf{HTTP Header} & \textbf{Content}\\
				\hline
				#5
			\end{tabularx}
			\item[Body] #6
		\end{description}
		
		\item[Error Codes]
		\begin{tabularx}{\linewidth}{c|*1{>{\centering\arraybackslash}X}@{}}
			\textbf{HTTP Status Code} & \textbf{Reason}\\
			\hline
			#7
		\end{tabularx}
	\end{description}
}

\section{Master API}
\subsection{slaves}
\apicall[put]{/slave}
	{slave & slave object & slave to add}
	{201}
	{Location & Reference to added slave}
	{}
	{400 & Invalid parameter}
\section{GUI}
\renewcommand{\gocurpackage}{gui}

%TODO remark about AngularJS in the beginning of the package description


%\section{Notifier}
\beginpackage{notifier}

The notifier is installed with the master. 
Notifies a predefined list of contacts whenever a new \refgo{master.Problem} occurs.

\struct{Controller}{
	Imports CLIFFagParser to parse command line parameters, such as the path to the contact file.
	Imports os, io and fmt to parse the contacts file.
	Receives problems via \reftype{APIClient}.\\
	Determines which problems should be send.\\
	Starts implementations of notifier to send the problems.	
}{
	\property{lastProblems}[[]Problem]{
		List of recent problems.
	}
}{
	\method{main}
	{
		A Go application starts by initializing the Go 'main' package and then running the \textit{main()} function.
		\begin{itemize}
			\item Receives problems through the \reftype{APIClient}.
			\item Uses reftype{diffProblems} to decide which should be send.
			\item Uses reftype{notify} to hand over problems to a \reftype{Notifier}.
		\end{itemize}
	}
	\method{diffProblems}[[]Problem](
		\param{received}[[]Problems]{List of recently send problems}
		){
		Compares \refproperty{lastProblems} to the problems currently staged for notification.
		To prevent spam only a new \refgo{master.Problem} will be send
	}
	\method{notify}(
		\param{problem}[Problem]{The problem to be send}
	){
		Hands over problems to a \reftype{Notifier}.
	}
}

\struct{Parser}{
	Used by the \reftype{Controller} to parse the contact file.
}{
	\method{Parse}[[]\reftype{Contact}, error](
		\param{path}[string]{Path of the contacts file}
		){
			Parses the contacts file.
			Returns an array of contacts.	
		}
}

\struct{APIClient}{
	Receives new problems from the master.
	}{
	\method{Receive}[[]Problem](
		\param{host}[string]{
			Hostname of the APIServer
			}
		){
		}
}

\interface{Notifier}{
	}{
	\method{Send}(
		\param{problem}[Problem]
		){Receive the problem and send it to the contacts}
}

\struct{EmailNotifier}{
	Imports smtp.
	Implements \reftype{Notifier} to send emails via smtp to a list of EmailContacts.
}{
	\property{Contact}[[]EmailContact]
}{
	\method{Send}(
		\param{problem}[Problem]{The \refgo{master.Problem} to be send}
		){
			Translates the problem into an email message.
		}
	\method{sendMailToContacts}[](
		\param{msg}[[]byte]{}
		){
			Sends the messages created by \refmethod{Send} to a list of email contacts.
			Uses the GO smtp package.
		}
}

\interface{Contact}{
	Interface for contactdata of the recipents of problem notifications.\newline
	Contactdata is read from the contacts file.
}{
	\property{contactData}[string]
}

\struct{EmailContact}{
	Implements \reftype{Contact}.
}{
	\property{Address}[string]{Email address of one contact}
}
\struct{SMTPRelay}{
		An email relay using SMTP.
}{
	\property{Hostname}[string]{Hostname of the SMTP Relay}
	\property{MailFrom}[string]{The email address from which the notifications will be send}
}


\glsaddallunused
\makeatletter
\newglossarystyle{myAltlist}{
  \glossarystyle{altlist} % base this style on altlist
  \renewcommand*{\glossaryentryfield}[5]{
  \item[\glsentryitem{##1}\glstarget{##1}{##2}]
    \mbox{}\par\nobreak\@afterheading
    ##3\glspostdescription\space On page ##5.
  }
}
\makeatother
\printglossary[type=main, title={Glossary}, toctitle={Glossary}, style=myAltlist]

\end{document}
