\newtheorem{theorem}{Theorem}


\section{Master: Model}% cannot use subsections here because helpers macros have fixed sectioning-hierarchy
\beginpackage{model}

\begin{figure}[H]
	\includegraphics[width=\textwidth]{model_classes}
	\caption{Model}
\end{figure}

The \refgo{model} package contains all datastructures used to model the cluster managed by \mamid.

The structures implementing the \codeinline{StatusMessage} interface are used to pass status information on the application internal bus (\refgo{master.Bus}).

The other structures in \refgo{model} are stored in the database using a database abstraction layer.

\struct{DB}{
  The \reftype{DB} contains a lightweight wrapper around the gorm DB.
}{
  \property{Driver}[string]{Driver backend.}
  \property{gormDB}[*gorm.DB]{The gorm DB object.}
  \property{dbName}[*string]{Name of database.}
  \property{connDSN}[*string]{DSN string for connection.}
}{
  \method{InitializeDB}[{(*\reftype{DB}, error)}](
    \param{driver}[string]{Driver backend to use (e.g. postgres).}
    \param{dsn}[string]{DSN string for connection.}
  ){
    Function returning an instance of \reftype{DB} after establishing a connection.
  }
  \method{Begin}[*gorm.DB]{
    Starts a transaction and returns the \codeinline{*gorm.DB} object to work with.
  }
  \method{CloseAndDrop}{
    Closes the connection and removes the database from the filesystem.
  }
}

\struct{Slave}{
  A \reftype{Slave} models an instance of the \mamid slave application running on a host.
}{
  \property{Id}[int]{Unique ID generated and used by the ORM layer (primary \& foreign key) and Master API (as simple identifier).}
  \property{Hostname}[string]{Hostname of the host the slave is running on. \refproperty{Hostname} uniquely identifies a host in the cluster.\\
                              \textbf{Note:} IPv4 addresses in quad-dotted notation and IPv6 addresses as in RFC2732 Section 3.1 notation (\texttt{[<16bit-colon-separated>]}) are valid hostnames.} 
  \property{Port}[PortNumber]{Port on which the slave listens for connections from the master.}
  \property{MongodPortRangeBegin}[PortNumber]{Beginning of the range of ports \reftype{Mongod} spawned by the slave may listen on (inclusive).}
  \property{MongodPortRangeEnd}[PortNumber]{End of the range of ports \reftype{Mongod} spawned by the slave may listen on (exclusive).
      Implicitly specifies --- in combination with \refproperty{MongodPortRangeBegin} ---
      how many concurrently running \reftype{Mongod}s can be spawned by a slave.}
  \property{PersistentStorage}[bool]{Flag indicating whether the slave's data directory is located on persistent or volatile storage.}
  \property{ObservationError}[*\refgo{msp.Error}]{Slave error not related to a particular mongod or replicaset.}
  \property{ConfiguredState}[\reftype{SlaveState}]{The state of the slave set by the user.}
  \property{RiskGroup}[\reftype{RiskGroup}]{The risk group the slave is a member of.}
  \property{Mongods}[[]\reftype{Mongod}]{The \reftype{Mongod}s that are running or should be running on this slave.}
  \property{Problems}[[]\reftype{Problem}]{The \reftype{Problem}s this slave is related to.}
}

\goenum{SlaveState}
{}
{
  \goenumitem{Active}{Slaves in active mode are available to host Mongod instances. They are monitored and the administrator is notified of problems on the slave.}
  \goenumitem{Maintenance}{Slaves in maintenance mode are not monitored but running Mongods will not be stopped.}
  \goenumitem{Disabled}{
    Slaves in disabled mode are not monitored.
    The \refgo{master.ClusterAllocator} attempts to migrate Mongod instances from disabled slaves to active slaves.\\
    However, it will not violate or worsen availability constraints imposed by the administrator.
  }
}

\goenum{ShardingRole}
{}
{
  \goenumitem{None}{No sharding on the Replica Set.}
  \goenumitem{ShardServer}{Replica Set is a normal Shard.}
  \goenumitem{ConfigServer}{Replica Set is a config Shard.}
}

\struct{RiskGroup}{
  A \reftype{RiskGroup} models a set of slaves sharing a common risk of failure.
  
  A \reftype{Slave} always belongs to exactly one \reftype{RiskGroup}.
  Hence, \codeinline{RiskGroup}s are mutually disjoint.
}{
  \property{Id}[int]{Unique ID generated and used by the ORM layer (primary \& foreign key) and Master API (as simple identifier).}
  \property{Name}[string]{Unique name of the risk group.}
  \property{Slaves}[[]\reftype{Slave}]{The \reftype{Slave}s that are members of a the given instance of \reftype{RiskGroup}.}
}

\struct{ReplicaSet}{
  A \reftype{ReplicaSet} models a MongoDB Replica Set consisting of multiple \reftype{Mongod}s.

  The degree of desired availbility \& redundancy of data stored in the Replica Set is modeled through 
  \refproperty{PersistentMemberCount} and \refproperty{VolatileMemberCount}.
}{
  \property{Id}[int]{Unique ID generated and used by the ORM layer (primary \& foreign key) and Master API (as simple identifier).}
  \property{Name}[string]{Name of the Replica Set.}
  \property{PersistentMemberCount}[uint]{Number of persistent \reftype{Mongod}s that should be part of this \reftype{ReplicaSet}.}
  \property{VolatileMemberCount}[uint]{Number of volatile \reftype{Mongod}s that should be part of this \reftype{ReplicaSet}.}
  \property{Initiated}[bool]{Whether rs.initiate() has been successfully executed on this Replica Set.}
  \property{ShardingRole}[\reftype{ShardingRole}]{Sharding type of the Replica Set.}
  \property{Problems}[[]\reftype{Problem}]{The \reftype{Problem}s this Replica Set is related to.}
}

\struct{Mongod}{
  An instance of \reftype{Mongod} is uniquely identified by
  \begin{itemize}
    \item its parent slave (the machine it is running on)
    \item the port it is listening on
    \item the Replica Set it belongs to.
  \end{itemize}

  The two attributes \refproperty{DesiredState} and \refproperty{ObservedState} are both optional.

  The role of a Mongod process depends on the permutations of \codeinline{nil} and $\not=\codeinline{nil}$
    assignments and is depicted in the table below:
 
  \begin{figure}[H]
  \centering
  \begin{tabularx}{0.6\linewidth}{|c|X|X|}
          \hline
          \diaghead{lengthofthediagheads}{\codeinline{DesiredState}}{\codeinline{ObservedState}} & $=\codeinline{nil}$ & $\not= \codeinline{nil}$ \\\hline
          $=\codeinline{nil}$       & \text{not existent} & zombie  \\\hline
          $\not= \codeinline{nil}$  & not spawned & present \\\hline
  \end{tabularx}
  \caption{Mongod role by \codeinline{DesiredState} \& \codeinline{ObservedState}}
  \end{figure}

}{
  \property{Id}[int]{Unique ID generated and used by the ORM layer (primary \& foreign key) and Master API (as simple identifier).}
  \property{Port}[\reftype{PortNumber}]{TCP Port number the Mongod is listening on.}
  \property{ReplSetName}[string]{Name of the Mongods Replica Set. A Mongod is always in exactly one Replica Set.}
  \property{ObservationError}[\refgo{msp.Error}]{
    Latest error that occurred when attempting to update the \refproperty{ObservedState}.\\
    \codeinline{nil} if no error occurred on last attempt.
  }
  \property{LastEstablishStateError}[\refgo{msp.Error}]{
    Latest error reported by the slave when attempting to establish the \refproperty{DesiredState}.\\
    \codeinline{nil} if no error occurred on last attempt.
  }
  \property{DesiredState}[*\reftype{MongodState}]{State the Mongod should be in. \codeinline{nil} if no desired state has been set.}
  \property{ObservedState}[*\reftype{MongodState}]{State the Mongod was last observed to be in. \codeinline{nil} if no observation has been made.}
  \property{ParentSlave}[\reftype{Slave}]{Slave this Mongod is or should be running on.}
  \property{ReplicaSet}[*\reftype{ReplicaSet}]{Replica Set this Mongod is participating in.}
}

\struct{MongodState}{
  Description of state of a particular \reftype{Mongod} that is not already contained in \refgo{model.Mongod}.
}{
  \property{ParentMongod}[\reftype{Mongod}]{Mongod this State is part of.}
  \property{ShardingRole}[\reftype{ShardingRole}]{Sharding configuration of the Mongod.}
  \property{ExecutionState}[\reftype{MongodExecutionState}]{Execution state of the Mongod process.}
}


\goenum{MongodExecutionState}{
  The different states of a Mongod that are relevant to the implementation of \mamid.

  The following value definitions are ordered by availibility of data in the particular state.
}{
  \goenumitem{Destroyed}{Mongod data directory not present and Mongod not running.}
  \goenumitem{NotRunning}{Data directory present but not Mongod running.}
  \goenumitem{Recovering}{Mongod running but not available for reads because data needs to be synced from other Replica Set members.
                          Common when adding new members to an existing Replica Set.}
  \goenumitem{Running}{Mongod running and in sync with other Replica Set members. }
}

\typealias{PortNumber}[uint16]{
  A TCP port number as specified in \href{https://tools.ietf.org/html/rfc793\#section-3.1}{RFC 793 Section 3.1}.
}

\struct{Problem}{
  Representation of a problem or error that occurred during operation of the \refgo{master}.

  The references to \reftype{Slave}, \reftype{ReplicaSet} and \reftype{Mongod} enable searchability
  of the \reftype{Problem} list. Exactly one of these must not be \codeinline{nil}.

  See \refgo{master.ProblemManager} for the transformation from errors to this structure.
}{
  \property{Id}[int]{
    Unique ID of the problem generated and used by the ORM layer (primary \& foreign key) and Master API (as simple identifier).\\
    If the \emph{same} problem occurrs after it was resolved before,
    its different incarnations are distinguishable by their \refproperty{Id}.
  }
  \property{Description}[string]{Short human-readable description of the problem, suitable for subject lines of error messages.}
  \property{LongDescription}[string]{Long human-readable description of the problem.}
  \property{ProblemType}[uint]{Identifies the \codeinline{StatusMessage} type this problem originated from. Used to check if a problem of the same type for the same \codeinline{ReplicaSet}/\codeinline{Slave}/\codeinline{Mongod} already exists.}
  \property{FirstOccurred}[time.Time]{The point in time when the particular problem first occurred.}
  \property{LastUpdated}[time.Time]{The point in time when the particular problem was last observed, e.g. the time of the last monitor run.}
  \property{Slave}[*\reftype{Slave}]{A reference to a \reftype{Slave} from which the problem originated. May be \codeinline{nil}}
  \property{ReplicaSet}[*\reftype{ReplicaSet}]{A reference to a \reftype{ReplicaSet} form which the problem originated. May be \codeinline{nil}}
  \property{Mongod}[*\reftype{Mongod}]{A reference to a \reftype{Mongod} from where the problem originated. May be \codeinline{nil}}
}

%data transfer objects (used to communicate via the master.Bus)

\interface{StatusMessage}{
  \refgo{StatusMessage} is the interface implemented by all message objects that are used to communicate state between different \refgo{master} components.
  
  Since receiving messages from the \refgo{master.Bus} is implemented via Go type assertions which cannot be modeled in UML, the authors chose
  an empty interface for grouping the message objects in UML.
}

\struct{MongodMatchStatus}
{
  A message object implementing \reftype{StatusMessage}.
Represents whether the observed and desired state of a \reftype{Mongod} do not match.
}{
  \property{Mismatch}[bool]{Whether the states do not match.}
  \property{Mongod}[\reftype{Mongod}]{The Mongod whose states do/do not match.}
}

\struct{ConnectionStatus}
{
  A message object implementing \reftype{StatusMessage}.
Represents whether a \reftype{Slave} cannot be reached from the master.
}{
  \property{Unreachable}[bool]{Whether the slave cannot be reached.}
  \property{Slave}[\reftype{Mongod}]{The slave that can/cannot be reached.}
  \property{CommunicationError}[\refgo{msp.CommunicationError}]{The error that occurred while trying to reach the slave. \codeinline{nil} if the slave can be reached.}
}

\struct{ReplicaSetInitiationStatus}
{
  A message object implementing \reftype{StatusMessage}.
Represents whether the Replica Set has been initiated on a \reftype{Slave}.
}{
  \property{Initiated}[bool]{Whether the Replica Set has been successfully initiated.}
  \property{Slave}[\reftype{Slave}]{The Slave whose Replica Set has/has not been initiated.}
}

\struct{DesiredReplicaSetConstraintStatus}
{
  A message object implementing \reftype{StatusMessage}.
  Represents whether the \refgo{master.ClusterAllocator} can satisfy the constraints on a Replica Set given by
  \begin{itemize}
    \item the persistent and volatile member counts of the \reftype{ReplicaSet}
    \item the maximum number of \reftype{Mongod}s on the \reftype{Slave}s
    \item the \reftype{RiskGroup}s
  \end{itemize}
}{
  \property{Unsatisfied}[bool]{Whether the constraint cannot be satisfied.}
  \property{ReplicaSet}[\reftype{ReplicaSet}]{The Replica Set whose constraints can/cannot be satisfied.}
  \property{ActualVolatileCount}[uint]{The count of volatile members the Replica Set currently has been allocated. If the constraint is satisfied this is the same as \reftype{ReplicaSet.VolatileMemberCount}.}
  \property{ActualPersistentCount}[uint]{The count of persistent members the Replica Set currently has been allocated. If the constraint is satisfied this is the same as \reftype{ReplicaSet.PersistentMemberCount}.}
}

\struct{ObservedReplicaSetConstraintStatus}
{
A message object implementing \reftype{StatusMessage}.
  Represents whether the observed member counts of a \reftype{ReplicaSet} do not match \reftype{ReplicaSet.VolatileMemberCount} and \reftype{ReplicaSet.PersistentMemberCount}
}{
  \property{Unsatisfied}[bool]{Whether the counts do not match.}
  \property{ReplicaSet}[\reftype{ReplicaSet}]{The Replica Set whose counts do not match.}
  \property{ActualVolatileCount}[uint]{The observed volatile member count of the Replica Set.}
  \property{ActualPersistentCount}[uint]{The observed persistent member count of the Replica Set.}
}


\section{Master: Main Package}
\beginpackage{master}

\begin{figure}[H]
	\includegraphics[width=\textwidth]{master_classes}
	\caption{Master}
\end{figure}

The Master (\refgo{master}) is the most complex component of \mamid.

It fulfills many different tasks, running in separate goroutines.

These tasks need to communicate with each other. Two different mechanism are employed to achieve this:
\begin{itemize}
        \item A shared persistent database with an \emph{object-relational mapping} (ORM) abstraction layer
	\item A \reftype{Bus} that leverages Go's channels to send status updates via \refgo{model.StatusMessage}
\end{itemize}
This results in loose coupling of the \refgo{master} components and increases maintainability of the code base.

\struct{Bus}{

        \begin{figure}[H]
	\includegraphics[width=\textwidth]{activity_diagram}
	\caption{Communication inside the master over the Bus}
        \end{figure}

	The \reftype{Bus} is used by the different components of the master to share and react to status updates.
	At the start of the application channels for all the components are created and passed to them.
	The components can then use this channel to broadcast messages to the other components and receive messages
	from the \codeinline{Bus} from which they can filter the ones they are interested in.
	
        All messages sent over the \codeinline{Bus} carry a description of some state --- not events or sender-specific data.
        This decouples the implementation of sender and receiver and enhances receiver testability.

        Some senders may communicate state descriptions repeatedly while others may only send them occassionally.

        \textbf{Example}: mismatch detection is decoupled from deployment and problem reporting.\\
        \reftype{Monitor} sends \refgo{model.MongodMatchStatus} messages every monitoring interval while
        \reftype{ClusterAllocator} does so only when it changes the desired deployment.\\
        Neither \reftype{Monitor} nor \reftype{ClusterAllocator} need to know (and call) \reftype{Deployer} and \reftype{ProblemManager}
        directly.\\
        Instead, the latter two components simply subscribe to the bus.

}{
	\property{readChannels}[[]chan interface\{\}]{The channels the Bus will read messages from.}
	\property{writeChannels}[[]chan interface\{\}]{The channels the Bus will broadcast messages to.}
}{
	\method{GetNewReadChannel}[<-chan interface\{\}]{Creates a new read channel that is attached to the Bus.
		The new channel will be added to \refproperty{readChannels} and returned to the caller.}
	\method{GetNewWriteChannel}[chan<- interface\{\}]{Creates a new write channel that is attached to the Bus.
		The new channel will be added to \refproperty{writeChannels} and returned to the caller.}
	\method{Run}{Runs the main loop of the Bus.}
	\method{Kill}{Stops the main loop of the Bus.}
}

\struct{Deployer}{
	The \reftype{Deployer} listens on the bus channel for \refgo{model.MongodMatchStatus} objects with \refgoalt{model.MongodMatchStatus.Mismatch}{Mismatch} set to true and deployes the desired state of the affected Mongod using the \refgo{msp.MSPClient}
}{
	\property{DB}[*\refgo{model.DB}]{Initialized handle to the database.}
	\property{BusReadChannel}[<-chan interface\{\}]{Read channel to the application bus.}
	\property{MSPClient}[\refgo{msp.MSPClient}]{MSP talking client talking to the slave servers.}
}{
	\method{Run}{Waits for \refgo{model.MongodMatchStatus} and \refgo{model.ReplicaSetInitiationStatus} objects and runs \refmethod{pushMongodState} when a mismatch occurrs or \refmethod{handleReplicaSetInitiationStatus} when not yet initiated.}
	\method{pushMongodState}
	(
		\param{mongod}[\refgo{model.Mongod}]{The Mongod to push the state to.}
	){
		Gets the desired state of \refparam{mongod} from the database and pushes it using the \refgo{msp.MSPClient}.
	}
	\method{handleReplicaSetInitiationStatus}
	(
		\param{s}[\refgo{model.ReplicaSetInitiationStatus}]{The slave to push an initiation request to (if not already initiated).}
	){
		Tells a slave to initialize a Replica Set using the \refgo{msp.MSPClient}.
	}
}

\struct{Monitor}{
	The \reftype{Monitor} uses the \refgo{msp.MSPClient} to query all slaves for their Mongod's states, saves the observed state in the database and compares the observed state to the desired state from the database. It publishes \refgo{model.MongodMatchStatus} objects with the result of each comparison on the bus. It also checks whether the constraints of the involved \refgo{model.ReplicaSet} are fulfilled and accordingly publishes a \refgo{model.ObservedReplicaSetConstraintStatus} object.
}{
	\property{DB}[*\refgo{model.DB}]{Initialized handle to the database.}
	\property{BusWriteChannel}[chan<- interface\{\}]{Write channel to the application bus.}
	\property{MSPClient}[\refgo{msp.MSPClient}]{MSP talking client talking to the slave servers.}
	\property{Interval}[time.Duration]{Refresh frequency of the slave observing.}
}{
	\method{Run}{Periodically fetches the Mongods' states from the slaves, saves it in the database, compares it to the desired state using \refmethod{compareStates} and publishes the result on the bus as a \refgo{model.MongodMatchStatus} object.}
	\method{handleObservation}(
		\param{m}[[]\refgo{msp.Mongod}]{Array of Mongods received from the slave.}
		\param{e}[*\refgo{msp.Error}]{Potential generic slave (communication) error.}
		\param{s}[\refgo{model.Slave}]{Slave which has been observed.}
	){
		Processes and stores the result of a status request to the slave.
	}
	\method{observeReplicaSets}{
		Checks the state of the Replica Sets and sends \refgo{model.ObservedReplicaSetConstraintStatus} and \refgo{model.ReplicaSetInitiationStatus} messages over the Bus.
	}
	\method{sendMongodMismatchStatusToBus}(
		\param{tx}[*gorm.DB]{Database handle for active transaction.}
		\param{s}[\refgo{model.Slave}]{Slave whose Mongods need to be checked for mismatches.}
		\param{modelToObservedMap}[map[int]\refgo{msp.Mongod}]{Map from Mongod Ids to observed Mongods.}
	){
		Iterates over all Mongods, compares them via \refmethod{compareStates} and sends the returned \refgo{model.MongodMatchStatus} messages over the Bus.
	}
	\method{compareStates}[\refgo{model.MongodMatchStatus}](
		\param{tx}[*gorm.DB]{Database handle for active transaction.}
		\param{m}[\refgo{model.Mongod}]{The locally stored Mongod.}
		\param{observed}[\refgo{msp.Mongod}]{The observed Mongod to compare the states against.}
	){
		Compares the \refgo{model.Mongod.ObservedState} and \refgo{model.Mongod.DesiredState} for mismatches and returns the result as a \refgo{model.MongodMatchStatus}.
	}
}

\struct{ProblemManager}{
	The \reftype{ProblemManager} listens on the bus for \refgo{model.StatusMessage}s, checks if they represent an error and generates or removes a problem and saves the changes in the database.
}{
	\property{DB}[*\refgo{model.DB}]{Initialized handle to the database.}
	\property{BusReadChannel}[<-chan interface\{\}]{Read channel to the application bus.}
}{
	\method{Run}{Listens for \refgo{model.StatusMessage}s on the bus and checks whether they represent an error.
		If they do, generates a problem using \refmethod{generateProblem} and stores it in the database.
		If they don't, checks if the database contains a \refgo{model.Problem} of the same type for the same object and removes it.}
}

\struct{ClusterAllocator}{
  The \reftype{ClusterAllocator} determines the layout of the cluster managed by \mamid.

  It attempts to fulfill the constraints defined through the model objects, in particular
  \begin{itemize}
    \item A \refgo{model.ReplicaSet}'s \refgoalt{model.ReplicaSet.VolatileMemberCount}{VolatileMemberCount} \& \refgoalt{model.ReplicaSet.PersistentMemberCount}{PersistentMemberCount}
    \item The \refgo{model.Slave}'s allowed number of Mongod instances
          (\refgoalt{model.Slave.MongodPortRangeBegin}{MongodPortRangeBegin} to \refgoalt{model.Slave.MongodPortRangeEnd}{MongodPortRangeEnd}).
    \item The \refgo{model.Slave.ConfiguredState}
    \item The configured \refgo{model.RiskGroup}s.
  \end{itemize}

  An iterative algorithm is employed to decide on a cluster layout described through
  \refgo{model.Mongod.DesiredState}s that
  \begin{itemize}
    \item attempts to fulfill the above constraints
    \item attempts an even distribution of Mongods on the different cluster hosts
    \item is a minimal change in comparison to the previous layout
  \end{itemize}

  \label{theorem:idempotence_clusterallocator}
  \begin{theorem}[Idempotence of the ClusterAllocator]
    \item Let $l$ be a layout of the cluster. Then:
          \[\text{ClusterAllocator}(\text{ClusterAllocator}(l)) = \text{ClusterAllocator}(l)\]
  \end{theorem}

}{
  \property{BusWriteChannel}[<-chan interface\{\}]{Write channel to the application bus.}
}{
  \method{Run}(
    \param{DB}[*\refgo{model.DB}]{Initialized handle to the database.}
  ){
    Main loop of the ClusterAllocator, running \refmethod{CompileMongodLayout} in a fixed interval.
  }
  \method{InitializeGlobalSecrets}(
    \param{tx}[*gorm.DB]{Database handle for active transaction.}
  ){
    Generates keyfile (shared secret) as well as MongoDB credentials for the admin user.
  }
  \method{CompileMongodLayout}(
    \param{tx}[*gorm.DB]{Database handle for active transaction.}
  ){
    Lay out the cluster as described above.
  }
}

\subsubsection{Pseudocode}

The \refgo{master.ClusterAllocator} is crucial to the stable operation of the \mamid-managed cluster.\\
Hence, it is worth defining the implementation of \refgo{master.ClusterAllocator.LayoutCluster} through pseudocode. %TODO parentheses

While studying the algorithms below, the reader should keep in mind that
\begin{itemize}
  \item changes in the cluster layout $\equiv$ change, creation or removal of \refgo{model.Mongod.DesiredState}
  \item the \refgo{model.Mongod.ObservedState} may change after an arbitrary amount of time or even not at all
  \item changes to a ReplicaSet must not violate or further worsen the high-availibility constraints,
        in particular \refgo{model.ReplicaSet}'s \refgoalt{model.ReplicaSet.VolatileMemberCount}{VolatileMemberCount} \& \refgoalt{model.ReplicaSet.PersistentMemberCount}{PersistentMemberCount}\\
        $\implies$ Mongods in \refgo{model.MongodExecutionState.Recovering} are an important special-case.
\end{itemize}

\newcommand{\pluseq}{\mathrel{{+}{=}}}
\newcommand{\minuseq}{\mathrel{{-}{=}}}
\SetKwInOut{Input}{input}
\SetKwInOut{Output}{output}
\SetKwProg{Fn}{Function}{}{}
\SetKw{Invariant}{invariant}
\IncMargin{0.5em}

Invariant: Algorithm(Algorithm(x)) = Algorithm(x)

Otherwise oscillations could occur

Most interesting case: Set a slave to disabled. Then a new Mongod should be spawned and recover and when it is done the old disabled one can be deleted.

\begin{algorithm}

\caption{Count members of a Replica Set that are \& intended to be in stable state (running)}

\Input{ReplicaSet $r$}
\Output{Number of $p_e$ and $v_e$ member processes of $r$ that are fully operational and are planned to remain in that state.}
\BlankLine
\Fn{EffectiveMemberCount(r ReplicaSet)}{

$p_e, v_e = 0$

\For{$m \in \text{r.Mongods}$}{
	\If{$\text{m.ObservedState.ExecutionState} = Running$ \\ 	%	#this line evaluates to false if m.ObservedState = NULL
		$\land \text{m.DesiredState.ExecutionState} = Running$}{ %#this line evaluates to false if m.DesiredState = NULL 

		\uIf{$m.ParentSlave.PersistentStorage$}{
			$p_e \pluseq 1 $
		}\Else{
			$v_e \pluseq 1 $
		}
	}
}
\Return $p_e, v_e$
}
\end{algorithm}


\begin{algorithm}
\caption{Destroy members of disabled replica sets where possible without violating p/v constraints.}
\ForEach{r in ReplicaSets}{
	
	$p_e, v_e \gets $ EffectiveRunningMembers(r)
	
	\ForEach(// same for persistent and volatile){$\textbf{x}_e \in \{p_e, v_e\}$}{
		\While{$x_e > r.\textbf{x}$}{
			\textbf{destroy} any $m \in r.Mongods$ where $m.ParentSlave$ is $\textbf{x} \land \textbf{disabled}$\;
			$\textbf{x}_e \minuseq 1$
		}
		
		\Invariant minimum number of disabled slaves are running members of $r$\;
		
		\While{$\textbf{x}_e > r.\textbf{x}$}{
			\textbf{destroy} any $m \in r.Mongods$ where $m.ParentSlave$ is $\textbf{x}$\;
			$\textbf{x}_e \minuseq 1$
		}
		
		\Invariant at most $r.\textbf{x}$ members 
	}
	
	\Invariant \textit{desired state}: at most (r.p|r.v) member processes of $r$\;
	// desired state = the state that will be deployed
	
}
\end{algorithm}

\begin{algorithm}
\caption{Count recovering and active members of a Replica Set}

\Input{ReplicaSet $r$}
\Output{Number of $p_a$ and $v_a$ member processes of $r$ that \begin{itemize}
		\item are \textbf{recovering}, i.e. soon-to-be fully operational
		\item fully operational (\textbf{running})
	\end{itemize} and should actually be in that state.}
\BlankLine
\Fn{AlreadyAddedMemberCount(r ReplicaSet)}{
	$p_a, v_a = 0$
	
	\For{$m \in \text{r.Mongods}$}{
		\If{$\text{m.ParentSlave.ConfiguredState} != Disabled$ \\
			$\land \text{m.DesiredState.ExecutionState} != NotRunning$}{ 
			
			\uIf{$m.ParentSlave.PersistentStorage$}{
				$p_a \pluseq 1 $
			}\Else{
			$v_a \pluseq 1 $
		}
	}
}
\Return $p_a, v_a$
}
\end{algorithm}

\begin{algorithm}
\caption{Spawn Mongods on under-provisioned Replica Sets respecting RiskGroup \& p/v constraints.}

$p_a, v_a \gets $ AlreadyAddedMemberCount(r)

\ForEach(// same for persistent and volatile){$\textbf{x}_a \in \{p_a, v_a\}$}{

	%# TODO need to fill the queue. only ReplicaSets which 'need' (use p_* to define what need means) are in  the queue 
	$RQ \gets$ PriorityQueue(R)\;% "relative amount of missing members")\;
	$RGSQ \gets$ map[RiskGroup]PriorityQueue(Slaves of RiskGroup)\;%, "relative amount of available Mongod ports")\;
	
	\While{$r = RQ.pop()$; $r \neq nil$}{
	
		\Invariant $r$ needs a member of type $\textbf{x}$
		
		// find $x$-slave $s$ in a risk group $g$ with no other Mongod of $r$ in $g$\;
		$candidates \gets \{g \mid g \in RGSQ.keys \land \text{r has no member in g}\}$\;
		$g \gets$ ARGMAX($g \in candidates$, RGSQ[g].PeekPriority())\;
		\uIf{$g \neq nil$}{
			$s \gets RGSQ[g].pop()$\;
			\Invariant $s$ has at least one free port\;
			spawn new Mongod $m$ on $s$ and add it to $r.Mongods$\;
			compute $MongodState$ for $m$ and set the $DesiredState$ variable\;
			
			\If{$r$ needs another member of type $\textbf{x}$}{
				$RQ.insert(r)$ // recompute priority
			}
			
			\If{$s$ has free ports}{
				$RGSQ[g].insert(s)$ // recompute priority
			}
			
			
		} \Else{
			yield error\;
			continue\;
		}
		
	
	}

}

\end{algorithm}

