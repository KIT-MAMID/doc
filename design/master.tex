\newtheorem{theorem}{Theorem}


\section{Master: Model}% cannot use subsections here because helpers macros have fixed sectioning-hierarchy
\renewcommand{\gocurpackage}{model}

The \refgo{model} package encapsulates all datastructures used to model the cluster managed by mamid

stored in database:
\struct{Slave}{
	A \refstruct{Slave} models an instance of the \mamid slave application running on a host.
}{
\property{Id}[uint]{Unique id.}
\property{Hostname}[string]{Hostname of the host the slave is running on.}
\property{Port}[uint]{Port where the slave listens for connections from the master.}
\property{Port}[uint]{Port where the slave listens for connections from the master.}
\property{MongodPortRangeBegin}[uint]{Beginning of the range of ports spawned \refstruct{Mongod} may listen on (inclusive).}
\property{MongodPortRangeEnd}[uint]{End of the range of ports spawned \refstruct{Mongod} may listen on (exclusive). Implicitly specifies together with \refproperty{MongodPortRangeBegin} how many \refstruct{Mongod}s may be spawned on the slave.}
\property{PersistentStorage}[bool]{Whether the slaves data directory is on persistent storage or volatile storage.}
\property{ConfiguredState}[\refenum{SlaveState}]{The state of the slave set by the user.}
\property{RiskGroup}[\refstruct{RiskGroup}]{The risk group the slave is in.}
\property{Mongods}[[]\refstruct{Mongod}]{The \refstruct{Mongod}s that are running or should run on this slave.}
}

\goenum{SlaveState}
{}
{
	\goenumitem{Active}{Slaves in active mode are available to host Mongod instances. They are monitored and the administrator is notified of problems on the slave.}
	\goenumitem{Maintenance}{Slaves in maintenance mode are not monitored but running Mongods will not be stopped.}
	\goenumitem{Disabled}{Slaves in disabled mode are not monitored. The Mongod instances running on disabled slaves will be migrated to active slaves}
}

\struct{RiskGroup}{
	A \refstruct{RiskGroup} models a set of slaves sharing a common risk of failure. A \refstruct{Slave} belongs to one \refstruct{RiskGroups}
}{
\property{Id}[uint]{Unique id.}
\property{Name}[string]{Name of the risk group.}
\property{Slaves}[[]\refstruct{Slave}]{The \refstruct{Slave}s belonging to this \refstruct{RiskGroup}}
}

\struct{ReplicaSet}{
	A \refstruct{ReplicaSet} models a MongoDB replica set consisting of multiple \refstruct{Mongod}s.
}{
\property{Id}[uint]{Unique id.}
\property{Name}[string]{Name of the replica set.}
\property{PersistentMemberCount}[uint]{Number of persistent \refstruct{Mongod}s that should be part of this \refstruct{ReplicaSet}}
\property{VolatileMemberCount}[uint]{Number of volatile \refstruct{Mongod}s that should be part of this \refstruct{ReplicaSet}}
}
  MongodState
  MongodExecutionState
  HostPort
  Problem

\struct{Mongod}{
  An instance of \refstruct{Mongod} is uniquely identified by
  \begin{itemize}
    \item its parent slave (the machine it is running on)
    \item the port it is listening on
    \item the replica set it belongs to.
  \end{itemize}

  The two attributes \refproperty{DesiredState} and \refproperty{ObservedState} are both optional.

  The role of a Mongod process depends on the permutations of \codeinline{nil} and $\not=\codeinline{nil}$
    assignments and is depicted in the table below:
 
  \begin{figure}[H]
  \centering
  \begin{tabularx}{0.6\linewidth}{|c|X|X|}
          \hline
          \diaghead{lengthofthediagheads}{\codeinline{DesiredState}}{\codeinline{ObservedState}} & $=\codeinline{nil}$ & $\not= \codeinline{nil}$ \\\hline
          $=\codeinline{nil}$       & \text{not existent} & zombie  \\\hline
          $\not= \codeinline{nil}$  & not spawned & present \\\hline
  \end{tabularx}
  \caption{Mongod role by \codeinline{DesiredState} \& \codeinline{ObservedState}}
  \end{figure}

}{
  \property{DesiredState}[MongodState]{State the Mongod should be in. \codeinline{nil} if no desired state has been set.}
  \property{ObservedState}[MongodState]{State the Mongod was last observed to be in. \codeinline{nil} if no observation has been made.}
}

data transfer objects (used to communicate via the master.Bus)

  DesiredReplicaSetConstraintStatus
  MongodMatchStatus
  ConnectionStatus
  ObservedReplicaSetConstraintStatus


\section{Master: Main Package}
\renewcommand{\gocurpackage}{master}

The Master (\refgo{master}) is the most complex component of \mamid.

fulfills many diff tasks, running in separate goroutines

different tasks need to communicate with each other
  shared persistent database with ORM abstraction 'gorm' => see model for which objcets
  status information (similar to observer pattern), realized through a Bus that leverages Golang's channels
    messages sent over bus are structs suffixed with .*Status
    generally, bus messages are sent repeatedly => != edge-triggered notifications
    encapsulate enough information about a type of status of an object in the database
      other components on the bus can react to information
=> loose coupling of components

\struct{Deployer}{
	The \refstruct{Deployer} listens on the bus for \refgo{model.MongodMismatchStatus} objects with \refgo{Mismatch} set to true and deployes the desired state of the affected mongod using the \refgo{msp.MSPClient}
}{
	\property{DB}[gorm.DB]{Initialized handle to the database.}
	\property{BusChannel}[chan interface\{\}]{Initialized channel to the application bus.}
}{
	\method{Run}{Waits for \refgo{model.MongodMismatchStatus} objects and runs \refmethod{pushMongodState} when a mismatch occurrs.}
	\method{pushMongodState}
	(
		\param{mongod}[model.Mongod]{The mongod to push the state to.}
	){
		Gets the desired state of \refparam{mongod} from the database and pushes it using the \refgo{msp.MSPClient}.
	}
}

\struct{Monitor}{
	The \refstruct{Monitor} uses the \refgo{msp.MSPClient} to ask all slaves for their mongod's states, saves the observed state in the database and compares the observed state to the desired state from the database. It publishes \refgo{model.MongodMismatchStatus} objects with the result of each comparison on the bus.
}{
	\property{DB}[gorm.DB]{Initialized handle to the database.}
	\property{BusChannel}[chan interface\{\}]{Initialized channel to the application bus.}
}{
	\method{Run}{Periodically fetches the mongods' states from the slaves, saves it in the database, compares it to the desired state using \refmethod{compareStates} and publishes the result on the bus as a \refgo{model.MongodMismatchStatus} object.}
	\method{compareStates}[MongodMatchStatus]
	(
		\param{mongod}[model.Mongod]{The mongod to compare the state of}
	){
		Compares the \refgo{model.Mongod.ObservedState} and \refgo{model.Mongod.DesiredState} for mismatches and returns the result as a \refgo{model.MongodMismatchStatus}.
	}
}

\struct{ProblemManager}{
	The \refstruct{ProblemManager} listens on the bus for \refgo{model.StatusMessage}s, checks if they represent an error and generates or removes a problem and saves the changes in the database.
}{
	\property{DB}[gorm.DB]{Initialized handle to the database.}
	\property{BusChannel}[chan interface\{\}]{Initialized channel to the application bus.}
}{
	\method{Run}{Listens for \refgo{model.StatusMessage}s on the bus and checks if they respresent an error.
		If they do, generates a problem using \refmethod{generateProblem} and stores it in the database.
		It they don't, checks if the database contains a problem of the same type for the same object and removes it.}
	\method{generateProblem}[Problem]
	(
		\param{e}[StatusMessage]{}
	){
		
	}
}

\struct{ClusterAllocator}{
  The \refstruct{ClusterAllocator} determines the layout of the cluster managed by \mamid.

  It attempts to fulfill the constraints defined through the model objects, in particular
  \begin{itemize}
    \item A \refgo{model.ReplicaSet}'s \refgo{VolatileNodeCount} \& \refgo{PersistentNodeCount}
    \item The \refgo{model.Slave}'s allowed number of Mongod instances
          (\refgo{MongodPortRangeBegin} to \refgo{MongodPortRangeEnd}).
    \item The \refgo{model.Slave.SlaveState}
    \item The configured \refgo{model.RiskGroup}s.
  \end{itemize}

  An iterative algorithm is employed to decide on a cluster layout described through
  \refgo{model.Mongod.DesiredState}s that
  \begin{itemize}
    \item attempts to fulfill the above constraints
    \item attempts an even distribution of Mongods on the different cluster hosts
    \item is a minimal change in comparison to the previous layout
  \end{itemize}

  \begin{theorem}{Idempotence of the ClusterAllocator}
    \label{theorem:idempotence_clusterallocator}
    Let $l$ be a layout of the cluster. Then $ClusterAllocator(ClusterAllocator(l)) = ClusterAllocator(l)$.
  \end{theorem}

}{
  \property{DB}[gorm.DB]{Initialized handle to the database.}
  \property{BusChannel}[chan interface\{\}]{Initialized channel to the application bus.} %TODO ref}
}{
  \method{LayoutCluster}{Lay out the cluster as described above.}
}

\subsubsection{Pseudocode}

The \refstruct{master.ClusterAllocator} is crucial to the stable operation of the \mamid-managed cluster.\\
Hence, it is worth defining the implementation of \refgo{master.ClusterAllocator.LayoutCluster} through pseudocode. %TODO parentheses

While studying the algorithms below, the reader should keep in mind that
\begin{itemize}
  \item changes in the cluster layout $\equiv$ change or creation of \refgo{model.Mongod.DesiredState} 
  \item the \refgo{model.Mongod.ObservedState} may change after an arbitrary amount of time or even not at all
  \item changes to a ReplicaSet must not violate or further worsen the high-availibility constraints,
        in particular \refgo{model.ReplicaSet}'s \refgo{VolatileNodeCount} \& \refgo{PersistentNodeCount}\\
        $\implies$ Mongods in \refgo{model.MongodExecutionState.Recovering} are an important special-case.
\end{itemize}

\newcommand{\pluseq}{\mathrel{{+}{=}}}
\newcommand{\minuseq}{\mathrel{{-}{=}}}
\SetKwInOut{Input}{input}
\SetKwInOut{Output}{output}
\SetKwProg{Fn}{Function}{}{}
\SetKw{Invariant}{invariant}
\IncMargin{0.5em}

Invariant: Algorithm(Algorithm(x)) = Algorithm(x)

Otherwise oscillations could occur

Most interesting case: Set a slave to disabled. Then a new Mongod should be spawned and recover and when it is done the old disabled one can be deleted.

\begin{algorithm}

\caption{Count members of a Replica Set that are \& intended to be in stable state (running)}

\Input{ReplicaSet $r$}
\Output{Number of $p_e$ and $v_e$ member processes of $r$ that are fully operational and are planned to remain in that state.}
\BlankLine
\Fn{EffectiveMemberCount(r ReplicaSet)}{

$p_e, v_e = 0$

\For{$m \in \text{r.Mongods}$}{
	\If{$\text{m.ObservedState.ExecutionState} = Running$ \\ 	%	#this line evaluates to false if m.ObservedState = NULL
		$\land \text{m.DesiredState.ExecutionState} = Running$}{ %#this line evaluates to false if m.DesiredState = NULL 

		\uIf{$m.ParentSlave.PersistentStorage$}{
			$p_e \pluseq 1 $
		}\Else{
			$v_e \pluseq 1 $
		}
	}
}
\Return $p_e, v_e$
}
\end{algorithm}


\begin{algorithm}
\caption{Destroy members of disabled replica sets where possible without violating p/v constraints.}
\ForEach{r in ReplicaSets}{
	
	$p_e, v_e \gets $ EffectiveRunningMembers(r)
	
	\ForEach(// same for persistent and volatile){$\textbf{x}_e \in \{p_e, v_e\}$}{
		\While{$x_e > r.\textbf{x}$}{
			\textbf{destroy} any $m \in r.Mongods$ where $m.ParentSlave$ is $\textbf{x} \land \textbf{disabled}$\;
			$\textbf{x}_e \minuseq 1$
		}
		
		\Invariant minimum number of disabled slaves are running members of $r$\;
		
		\While{$\textbf{x}_e > r.\textbf{x}$}{
			\textbf{destroy} any $m \in r.Mongods$ where $m.ParentSlave$ is $\textbf{x}$\;
			$\textbf{x}_e \minuseq 1$
		}
		
		\Invariant at most $r.\textbf{x}$ members 
	}
	
	\Invariant \textit{desired state}: at most (r.p|r.v) member processes of $r$\;
	// desired state = the state that will be deployed
	
}
\end{algorithm}

\begin{algorithm}
\caption{Count recovering and active members of a Replica Set}

\Input{ReplicaSet $r$}
\Output{Number of $p_a$ and $v_a$ member processes of $r$ that \begin{itemize}
		\item are \textbf{recovering}, i.e. soon-to-be fully operational
		\item fully operational (\textbf{running})
	\end{itemize} and should actually be in that state.}
\BlankLine
\Fn{AlreadyAddedMemberCount(r ReplicaSet)}{
	$p_a, v_a = 0$
	
	\For{$m \in \text{r.Mongods}$}{
		\If{$\text{m.ParentSlave.ConfiguredState} != Disabled$ \\
			$\land \text{m.DesiredState.ExecutionState} != NotRunning$}{ 
			
			\uIf{$m.ParentSlave.PersistentStorage$}{
				$p_a \pluseq 1 $
			}\Else{
			$v_a \pluseq 1 $
		}
	}
}
\Return $p_a, v_a$
}
\end{algorithm}

\begin{algorithm}
\caption{Spawn Mongods on under-provisioned Replica Sets respecting RiskGroup \& p/v constraints.}

$p_a, v_a \gets $ AlreadyAddedMemberCount(r)

\ForEach(// same for persistent and volatile){$\textbf{x}_a \in \{p_a, v_a\}$}{

	%# TODO need to fill the queue. only ReplicaSets which 'need' (use p_* to define what need means) are in  the queue 
	$RQ \gets$ PriorityQueue(R)\;% "relative amount of missing members")\;
	$RGSQ \gets$ map[RiskGroup]PriorityQueue(Slaves of RiskGroup)\;%, "relative amount of available Mongod ports")\;
	
	\While{$r = RQ.pop()$; $r \neq nil$}{
	
		\Invariant $r$ needs a member of type $\textbf{x}$
		
		// find $x$-slave $s$ in a risk group $g$ with no other Mongod of $r$ in $g$\;
		$candidates \gets \{g \mid g \in RGSQ.keys \land \text{r has no member in g}\}$\;
		$g \gets$ ARGMAX($g \in candidates$, RGSQ[g].PeekPriority())\;
		\uIf{$g \neq nil$}{
			$s \gets RGSQ[g].pop()$\;
			\Invariant $s$ has at least one free port\;
			spawn new Mongod $m$ on $s$ and add it to $r.Mongods$\;
			compute $MongodState$ for $m$ and set the $DesiredState$ variable\;
			
			\If{$r$ needs another member of type $\textbf{x}$}{
				$RQ.insert(r)$ // recompute priority
			}
			
			\If{$s$ has free ports}{
				$RGSQ[g].insert(s)$ // recompute priority
			}
			
			
		} \Else{
			yield error\;
			continue\;
		}
		
	
	}

}

\end{algorithm}

