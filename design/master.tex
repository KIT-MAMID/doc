\newtheorem{theorem}{Theorem}


\section{master}

\renewcommand{\gocurpackage}{model}
\renewcommand{\gocurpackage}{master}

\struct{ClusterAllocator}{
  The \refstruct{ClusterAllocator} determines the layout of the cluster managed by \mamid.

  It attempts to fulfill the constraints defined through the model objects, in particular
  \begin{itemize}
    \item A \refgo{model.ReplicaSet}'s \refgo{VolatileNodeCount} \& \refgo{PersistentNodeCount}
    \item The \refgo{model.Slave}'s allowed number of Mongod instances
          (\refgo{MongodPortRangeBegin} to \refgo{MongodPortRangeEnd}).
    \item The \refgo{model.Slave.SlaveState}
    \item The configured \refgo{model.RiskGroup}s.
  \end{itemize}

  An iterative algorithm is employed to decide on a cluster layout described through
  \refgo{model.Mongod.DesiredState}s that
  \begin{itemize}
    \item attempts to fulfill the above constraints
    \item attempts an even distribution of Mongods on the different cluster hosts
    \item is a minimal change in comparison to the previous layout
  \end{itemize}

  \begin{theorem}{Idempotence of the ClusterAllocator}
    \label{theorem:idempotence_clusterallocator}
    Let $l$ be a layout of the cluster. Then $ClusterAllocator(ClusterAllocator(l)) = ClusterAllocator(l)$.
  \end{theorem}

}{
  \property{DB}[gorm.DB]{Initialized handle to the database.}
  \property{BusChannel}[chan interface{}]{Initialized channel to the application bus.} %TODO ref}
}{
  \method{LayoutCluster}{Lay out the cluster as described above.}
}

\subsubsection{Pseudocode}

The \refstruct{master.ClusterAllocator} is crucial to the stable operation of the \mamid-managed cluster.\\
Hence, it is worth defining the implementation of \refgo{master.ClusterAllocator.LayoutCluster} through pseudocode. %TODO parentheses

While studying the algorithms below, the reader should keep in mind that
\begin{itemize}
  \item changes in the cluster layout $\equiv$ change or creation of \refgo{model.Mongod.DesiredState} 
  \item the \refgo{model.Mongod.ObservedState} may change after an arbitrary amount or even never.\\
  \item changes to a ReplicaSet must not violate or further worsen the high-availibility constraints,
        in particular \refgo{model.ReplicaSet}'s \refgo{VolatileNodeCount} \& \refgo{PersistentNodeCount}\\
        $\implies$ Mongods in \refgo{model.MongodExecutionState.Recovering} are an important special-case.
\end{itemize}

\newcommand{\pluseq}{\mathrel{{+}{=}}}
\newcommand{\minuseq}{\mathrel{{-}{=}}}
\SetKwInOut{Input}{input}
\SetKwInOut{Output}{output}
\SetKwProg{Fn}{Function}{}{}
\SetKw{Invariant}{invariant}
\IncMargin{0.5em}

Invariant: Algorithm(Algorithm(x)) = Algorithm(x)

Otherwise oscillations could occur

Most interesting case: Set a slave to disabled. Then a new Mongod should be spawned and recover and when it is done the old disabled one can be deleted.

\begin{algorithm}

\caption{Count members of a Replica Set that are \& intended to be in stable state (running)}

\Input{ReplicaSet $r$}
\Output{Number of $p_e$ and $v_e$ member processes of $r$ that are fully operational and are planned to remain in that state.}
\BlankLine
\Fn{EffectiveMemberCount(r ReplicaSet)}{

$p_e, v_e = 0$

\For{$m \in \text{r.Mongods}$}{
	\If{$\text{m.ObservedState.ExecutionState} = Running$ \\ 	%	#this line evaluates to false if m.ObservedState = NULL
		$\land \text{m.DesiredState.ExecutionState} = Running$}{ %#this line evaluates to false if m.DesiredState = NULL 

		\uIf{$m.ParentSlave.PersistentStorage$}{
			$p_e \pluseq 1 $
		}\Else{
			$v_e \pluseq 1 $
		}
	}
}
\Return $p_e, v_e$
}
\end{algorithm}


\begin{algorithm}
\caption{Destroy members of disabled replica sets where possible without violating p/v constraints.}
\ForEach{r in ReplicaSets}{
	
	$p_e, v_e \gets $ EffectiveRunningMembers(r)
	
	\ForEach(// same for persistent and volatile){$\textbf{x}_e \in \{p_e, v_e\}$}{
		\While{$x_e > r.\textbf{x}$}{
			\textbf{destroy} any $m \in r.Mongods$ where $m.ParentSlave$ is $\textbf{x} \land \textbf{disabled}$\;
			$\textbf{x}_e \minuseq 1$
		}
		
		\Invariant minimum number of disabled slaves are running members of $r$\;
		
		\While{$\textbf{x}_e > r.\textbf{x}$}{
			\textbf{destroy} any $m \in r.Mongods$ where $m.ParentSlave$ is $\textbf{x}$\;
			$\textbf{x}_e \minuseq 1$
		}
		
		\Invariant at most $r.\textbf{x}$ members 
	}
	
	\Invariant \textit{desired state}: at most (r.p|r.v) member processes of $r$\;
	// desired state = the state that will be deployed
	
}
\end{algorithm}

\begin{algorithm}
\caption{Count recovering and active members of a Replica Set}

\Input{ReplicaSet $r$}
\Output{Number of $p_a$ and $v_a$ member processes of $r$ that \begin{itemize}
		\item are \textbf{recovering}, i.e. soon-to-be fully operational
		\item fully operational (\textbf{running})
	\end{itemize} and should actually be in that state.}
\BlankLine
\Fn{AlreadyAddedMemberCount(r ReplicaSet)}{
	$p_a, v_a = 0$
	
	\For{$m \in \text{r.Mongods}$}{
		\If{$\text{m.ParentSlave.ConfiguredState} != Disabled$ \\
			$\land \text{m.DesiredState.ExecutionState} != NotRunning$}{ 
			
			\uIf{$m.ParentSlave.PersistentStorage$}{
				$p_a \pluseq 1 $
			}\Else{
			$v_a \pluseq 1 $
		}
	}
}
\Return $p_a, v_a$
}
\end{algorithm}

\begin{algorithm}
\caption{Spawn Mongods on under-provisioned Replica Sets respecting RiskGroup \& p/v constraints.}

$p_a, v_a \gets $ AlreadyAddedMemberCount(r)

\ForEach(// same for persistent and volatile){$\textbf{x}_a \in \{p_a, v_a\}$}{

	%# TODO need to fill the queue. only ReplicaSets which 'need' (use p_* to define what need means) are in  the queue 
	$RQ \gets$ PriorityQueue(R)\;% "relative amount of missing members")\;
	$RGSQ \gets$ map[RiskGroup]PriorityQueue(Slaves of RiskGroup)\;%, "relative amount of available Mongod ports")\;
	
	\While{$r = RQ.pop()$; $r \neq nil$}{
	
		\Invariant $r$ needs a member of type $\textbf{x}$
		
		// find $x$-slave $s$ in a risk group $g$ with no other Mongod of $r$ in $g$\;
		$candidates \gets \{g \mid g \in RGSQ.keys \land \text{r has no member in g}\}$\;
		$g \gets$ ARGMAX($g \in candidates$, RGSQ[g].PeekPriority())\;
		\uIf{$g \neq nil$}{
			$s \gets RGSQ[g].pop()$\;
			\Invariant $s$ has at least one free port\;
			spawn new Mongod $m$ on $s$ and add it to $r.Mongods$\;
			compute $MongodState$ for $m$ and set the $DesiredState$ variable\;
			
			\If{$r$ needs another member of type $\textbf{x}$}{
				$RQ.insert(r)$ // recompute priority
			}
			
			\If{$s$ has free ports}{
				$RGSQ[g].insert(s)$ // recompute priority
			}
			
			
		} \Else{
			yield error\;
			continue\;
		}
		
	
	}

}

\end{algorithm}

