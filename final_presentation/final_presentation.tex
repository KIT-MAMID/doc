\documentclass[11pt,aspectratio=169]{beamer}
\usepackage[utf8]{inputenc}
\usepackage[T1]{fontenc}
\usetheme{default}
\usepackage{pdfpcnotes}
\usepackage{todonotes}
\usepackage{xspace}

% MAMID macro
\newcommand{\mamid}{\textit{MAMID}\xspace}

% insert section title at \section{}
% http://tex.stackexchange.com/questions/178800/creating-sections-each-with-title-pages-in-beamers-slides
\AtBeginSection[]{
    \begin{frame}
        \vfill
        \centering
        \begin{beamercolorbox}[sep=8pt,center,shadow=true,rounded=true]{title}
            \usebeamerfont{title}\insertsectionhead\par%
        \end{beamercolorbox}
        \vfill
    \end{frame}
}

\begin{document}
   	\author{Niklas Fuhrberg, Anton Schirg,\\ Christian Schwarz, Janis Streib, Bob Weinand}
    \title{MAMID}
    \subtitle{Monitor and Manager for In-memory Databases}
    %\logo{}
    \institute{\textbf{Supervisor}\\Dr Marek Szuba\\SCC}
    \date{8 June 2016}
    \subject{Final Presentation}
    %\setbeamercovered{transparent}
    %\setbeamertemplate{navigation symbols}{}
    
    \frame[plain]{\maketitle}

    \begin{frame}[label=waterfall]{PSE: Where are we?}
        \centering\huge
        \textbf{5-stages image of the waterfall model}
    \end{frame}
    
    \section{Requirements Elicitation}
    
    \begin{frame}{Motivation}
        \begin{itemize}
            \item Envisat earth observation satellite
            \item Archive of data from MIPAS instrument\pnote{MIPAS records trace gasses in the atmosphere}
            \item Research project at IMK: %TODO what is its name?
            \item \pnote{Original data not that big but }Processed data: $96+$ TB
            \item Periodic reprocessing \& archiving \pnote{Reprocessing happens periodically, old results need to be kept for reference purposes => dataset is growing}
            \item Use MongoDB for storage\pnote{use MongoDB's sharding feature (more in a second) to distribute the data over multiple machines, easy way to access data and use it in downstream applications}
        \end{itemize}
    \end{frame}
    
    \begin{frame}{MongoDB}
        TODO
        \begin{itemize}
            \item MongoDs / Mongods
            \item Replica Set
            \item Sharding
        \end{itemize}
    \end{frame}
    
    \begin{frame}[allowframebreaks]{MongoDB on IMK Cluster}
        
        \begin{figure}
            \centering
            \missingfigure[figwidth=0.8\linewidth]{PSU image, no replica sets or risk groups}
        \end{figure}

        \begin{itemize}
            \item Few big machines with (slow?) persistent storage\\%TODO ask marek whcih one it is
                  + 4 cabinets à 20 blades à $98$ GB RAM %TODO check this
            \item Runs OpenIndiana \pnote{openindiana, who has heard of it? it's an illumos distro. which is? anyone??}
            \item Blades have small boot-only HDDs
            \item Cabinets have independent PSUs
        \end{itemize}

        \framebreak
        
        Performance: Replica Sets \pnote{so what can we do to maximize performance, in particular read performance for downstream applications?}
        
        \begin{itemize}
            \item Primaries: on blades with in-memory storage (\emph{performance})
            \item Secondaries: on machines with HDDs (\emph{persistence}) \pnote{this can be arranged by configuring the Priority of a Mongod in Replica Set elections}
        \end{itemize}
        \pnote{indicate where primaries and secondaries go using a pointing device}
        \begin{figure}
            \centering
            \missingfigure[figwidth=0.8\linewidth]{PSU image, replica sets, each replica set only on one PSU. Second version: failure of one PSU kills one replset}
        \end{figure}
        
        \framebreak
        
        Availability \& Redundancy: Risk Groups \pnote{so far we have performance, but what happens in case a PSU dies and takes a cabinet offline?}
        
        \begin{itemize}
            \item Distribute Replica Set members over different cabinets
            \item Assert enough Replica Set members have persistent storage\pnote{requirements may vary, depending on importance of data, ... generally a tradeoff between space usage and safety}
        \end{itemize}
        
        \begin{figure}
            \centering
            \missingfigure[figwidth=0.8\linewidth]{PSU image, replica sets, risk groups marked, each replica set distributed}
        \end{figure}
        
        \framebreak
        
        Operation
        
        \begin{itemize}
           \item Monitor Mongod processes \& configuration
           \item Notify administrators
           \item Continuous redeployment\pnote{blades may fail and loose their state \& data(remember: they are in-memory only)}
           \item Self-healing using hot spares\footnote{not in current release though (time constraints)}\pnote{Idea: have unused machines available to replace failing ones.}
        \end{itemize}
        
    \end{frame}
    
    \section{Related Work}
    
    \begin{frame}{Related Work}
        
        \begin{itemize}
            \item MongoDB In-Memory Engine ($<$ 3.2.6, \textit{enterprise} subscription)
            \item MongoDB Ops Manager (\textit{enterprise} subscription)
            \item MongoDB Cloud Manager (\textit{enterprise} subscription)
            \item Configuration Management (Puppet, etc.): cannot model dependencies, no monitoring, no integrated solution
        \end{itemize}
        
        Problems
        
        \begin{itemize}
            \item No MongoDB \textit{enterprise} binaries for OpenIndiana
            \item No solution to the mix of persistent \& volatile storage 
        \end{itemize}
        \pnote{regex matching host names appears to be possible, adequately named host schema would probably work out ok}
        
    \end{frame}
    
   \againframe<2>{waterfall}
    
    \section{Requirements Analysis}
   
   \begin{frame}{Requirements Analysis}
       \begin{itemize}
           \item declarative administration
           \item automation!
           \begin{itemize}
               \item cluster layout
               \item persistence requirements
               \item deployment
               \item re-configuration
            \end{itemize}
            \item monitoring
            \item alerting
        \end{itemize}
    \end{frame}
    
    \begin{frame}{Declarative Administration}
        
        \begin{itemize}
           \item<1-> Administrator describes cluster topology \& desired Replica Sets
           \item<2-> Replica Set description $\equiv$ set of constraints
           \item<3-> $\implies$ \alert<3>{greedy}, \alert<4>{priority driven}, \alert<5>{iterative} allocation algorithm
        \end{itemize}
        
        \begin{columns}
            \begin{column}{0.5\linewidth}
                \missingfigure[figwidth=\linewidth]{screenshot risk group view}
            \end{column}
            \begin{column}{0.5\linewidth}
                \missingfigure[figwidth=\linewidth]{screenshot replica set creation view}
            \end{column}
        \end{columns}
       
        
    \end{frame}
    
    
    \begin{frame}{Declarative Administration: Screenshots}
         %again in big
         \begin{columns}
             \begin{column}{0.5\linewidth}
                 \missingfigure[figwidth=\linewidth]{screenshot risk group view}
                \end{column}
                \begin{column}{0.5\linewidth}
                    \missingfigure[figwidth=\linewidth]{screenshot replica set creation view}
                \end{column}
        \end{columns}
    \end{frame}
    
    \begin{frame}
        
    
     % showcase slave states, necessary for understanding live demo
       % all below should be fine
       
%       \item automatic deployment
%       
%       \item continuous monitoring
%       
%       \item write management software called MAMID (expand acronym)
%       => be installed as an unprivileged daemon on all machines in the cluster
%       => take care of deployment, i.e. starting, stopping of Mongods
%       
%       \item simple administration through a web GUI
%       \item declarative approach\pnote{}administrator describes cluster topology and desired replica sets}
%           \begin{itemize}
%               \item describe cluster topology (slaves, risk groups) and desired Replica Sets
%               \item enable slave state 
%            \end{itemize}
%       Slaves, Risk Groups, Replica Sets => new screenshots
%       Slave States to control deployment of Mongods on this Slave
%       - explain what's written in the GUI
%       
%       automatic
%       enforcement of the constraints presented above
%       deployment of the Mongds and ReplicaSet configuration
%       re-configuration in case volatile nodes need to reboot and loose their state
%       
%       monitoring of the deployment instances
%       alerting in case of problems
%       
%       provide an HTTP API to script the cluster deployment if necessary
%       (later emphasize GUI is built on top of this)
%        
    
    \end{frame}
    
    
    
\end{document}


