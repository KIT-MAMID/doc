\section{Criteria}
\subsection{Mandatory Criteria}
\subsubsection{Cluster Description by Administrator}
% TODO analyze whether the given criteria can be phrased more abstractly \& move the details to (new) functional requirements. (-> forward 
%lookup references from criteria to functional requirement if required)
\begin{description}
	
	\oitem{MC}{} The administrator interacts with \mamid through a web gui. \done
	
	% inventory ops
	\oitem{MC}{inventory_definition} The master maintains a list of slaves. \done
	\begin{description}
		\oitem{MC}{} The gui visualizes the list of Slaves. \done
		\oitem{MC}{} The administrator can add Slaves to the list. \done
		\oitem{MC}{} The administrator can remove a Slave that does not host any MongoDB processes from the 
		list. \done
		\oitem{MC}{spec_risk_groups} The administrator can model a shared risk of failure between hosts, e.g. a shared power 
		supply. \done
		\oitem{MC}{available_slave_types} The administrator can specify whether the Slave has \glslink{persistent 
		storage}{persistent (typically HDD-/SSD-backed)} or \glslink{volatile storage}{volatile (RAM-backed)} storage. \done
		\oitem{MC}{root_data_directory} The administrator can specify in which filesystem directory on the host the Slave 
		and its MongoDB processes store data. \done[Done as commandline paramter for the Slave]
		% inventory ops -> Slave
		\oitem{MC}{slave_mode_active} The administrator can announce to \mamid that a Slave is ready to host MongoDB processes. 
		\done
		\oitem{MC}{slave_mode_maintenance} The administrator can announce to \mamid that a Slave is under maintenance to inhibit 
		automatic reconfiguration of its MongoDB processes. \done % TODO discuss
		\oitem{MC}{slave_mode_disabled} The administrator can announce to \mamid that a Slave should not host any MongoDB processes. 
		\done
	\end{description}
	
	% Replica Set ops
	\oitem{MC}{replica_set_create} The administrator can describe a new MongoDB Replica Set by specifying constraints on 
	how it should be configured by \mamid. \done
	\begin{description}
		\oitem{MC}{replica_set_config_profiles} The administrator can --- on creation of a Replica Set 
		(\ref{MC:replica_set_create}) --- specify that it must be usable as a configuration server for MongoDB sharding. 
		\done
		\oitem{MC}{replica_set_member_total_counts} The administrator can select the number of MongoDB instances (members) of a 
		Replica Set. \done
		\oitem{MC}{replica_set_member_pv_counts} Volatile and persistent member count of a Replica Set can be independently 
		configured, under constraints described in \ref{F:master_alloc_resp_pv_counts}. \done
	\end{description}
	\oitem{MC}{} The gui visualizes the list of configured Replica Sets. \done
	\oitem{MC}{} The administrator can destroy a Replica Set. \done
	
\end{description}

\subsubsection{MongoDB Configuration \& Monitoring}
\begin{description}
	\oitem{MC}{mongod_deployment1} \mamid asserts that the Replica Sets described by the administrator are configured on the cluster 
	(see \ref{MC:replica_set_create}). \done
	\oitem{MC}{mongod_deployment2} To achieve \ref{MC:mongod_deployment1}, \mamid spawns \& controls MongoDB processes on the hosts 
	using a Slave process. \done
	\begin{description}
		\oitem{MC}{mongod_redeployment} \mamid redeploys configured MongoDB processes to hosts where the Slave process reports 
		a situation different from what is expected by the master. \done
		\oitem{MC}{mongod_redeployment_powercycle_specific} Specifically, a host with volatile data storage can lose all data originating 
		from the Slave process or MongoDB and is automatically redeployed with correctly configured MongoDB instances 
		(\ref{MC:mongod_deployment2}). \done
	\end{description}
	% Todo old \oitem{MC}{} The master deploys the Replica Set configuration described by the administrator to the cluster.
	
	% monitoring features
	\oitem{MC}{detect_slave_unexpected_behavior} \mamid detects when a Slave in the inventory behaves unexpectedly, e.g. when 
	it becomes unreachable and the administrator did not announce maintenance to \mamid beforehand. \done
	\oitem{MC}{} \mamid informs the administrator by e-mail about problems in the cluster 
	(\ref{MC:detect_slave_unexpected_behavior}). \done
	\oitem{MC}{} The gui visualizes Slaves behaving unexpectedly (\ref{MC:detect_slave_unexpected_behavior}). \done
\end{description}

\subsection{Optional Criteria}\label{OptionalCriteria}
\begin{description}
	
	% master
	\oitem{OC}{api_authentication} \mamid requires authentication from the user for all operations. \done[Done with client certs
	issued/signed by a CA. Optional CLI flag for the master]
	
	% inventory
	\oitem{OC}{manual_autodiscovery} \mamid auto-discovers new Slaves on the administrator's request. \notdone
	\oitem{OC}{continuous_autodiscovery} \mamid continuously auto-discovers new Slaves. \notdone
	\oitem{OC}{monitor_icmp} \mamid recognizes when the Slave software does not respond but the corresponding host is still 
	connected to the network.  \notdone
	\oitem{OC}{export_import_snapshot} The administrator can back up and restore the cluster description. \done[Possible with 
	PostgreSQL tools by dumping the mamid PostgreSQL databse]
	
	% Slaves
	\oitem{OC}{tweak_performance_parameters} The administrator can customize performance-relevant parameters of MongoDB 
	processes. \notdone
	
	% automatic repair
	\oitem{OC}{auto_repair} \mamid supports automatic reconfiguration when detecting unexpected behavior of Slaves. The failing 
	Slave is marked as unsuitable to host MongoDB processes and redeployment is triggered to repair the degraded \glspl{replica 
	set} (extends \ref{MC:detect_slave_unexpected_behavior}). \notdone
	
	% Replica Sets
	\oitem{OC}{deploy_arbiters} \mamid deploys MongoDB arbiters for configured Replica Sets as needed, removing the 
	restriction to an odd count of Replica Set members in \ref{F:master_alloc_resp_pv_counts}.  \notdone
	\oitem{OC}{extended_monitoring} \mamid supports extended monitoring of hosts, i.e. metrics beyond the processes managed by \mamid.  
	\notdone
	
	% other
	\oitem{OC}{} The administrator can interact with \mamid via a cli.  \notdone
	\oitem{OC}{http_api} The administrator can interact with \mamid via a stable, documented HTTP API.  \done
\end{description}

\subsection{Demarcation Criteria}
\begin{description}
	\oitem{DMC}{} The administrator does not directly configure individual MongoDB instances spawned \& controlled by \mamid. 
	All configuration happens through \mamid, either automatically (e.g. Replica Set deployment) or through a \mamid interface (e.g. 
	GUI, CLI, HTTP API). \done[In some error cases administrative intervention may be required]
	\oitem{DMC}{} \mamid does not deploy MongoDB query routers.  \done
	\oitem{DMC}{} \mamid deploys neither the operating system nor other required software (such as MongoDB binaries) to the 
	cluster hosts.  \done
\end{description}