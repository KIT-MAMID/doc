\section{Development Process}
\subsection{CI (Jenkins)}\label{ci}
On each push a GitHub webhook triggers a Jenkins instance\footnote{\url{https://jenkins.dogcraft.de/job/mamid/}} to build the head 
of the pushed commits. While building, Jenkins marks the commit on GitHub to indicate whether the respective target passed.

Jenkins utilzes two remote build nodes: 
\begin{itemize}
	\item A node on a Linux server
	\item A node on a openindiana server 
\end{itemize}
Therefore artifacts for both Linux and OpenIndiana can be produced and tested.

Jenkins executes the folowing Makefile targets on both nodes:
\begin{itemize}
	\item \codeinline{check\_format}
	\item \codeinline{vet}
	\item \codeinline{test}
	\item \codeinline{build}
	\item \codeinline{cover}
\end{itemize}
\subsection{Workflow}
Each change of the code have to be implemented in a seperate feature branch since the master branch is protected on GitHub. The protection 
only allows changes to be pushed on master if the following constraints are fulfilled:
\begin{itemize}
	\item The change can be applied fast forward
	\item The change passes all required checks on the \hyperref[ci]{CI}, which are:
	\begin{itemize}
		\item the \codeinline{check\_format} makefile target succeeds
		\item the \codeinline{vet} makefile target succeeds
		\item the \codeinline{test} makefile target succeeds on both Linux and openindiana
		\item the \codeinline{build} makefile target succeeds
	\end{itemize}
\end{itemize}

\subsection{Local Staging Environment}
A local staging environment can be spawned using Docker and the Makefile target \codeinline{testbed\_up}.

The target
\begin{itemize}
  \item creates a host-only test network
  \item builds \mamid in a container
  \item builds Docker images with the newly built binaries for master, slave and notifier
  \item starts Docker containers: 1 master, 1 PostgreSQL instance, 3 slaves, 1 notifier
\end{itemize}

Re-executing the \codeinline{testbed\_up} wipes the previously created environment, allowing for quick reproducible testing of changes.

\subsection{Communication}

Team communication was handled mostly via the Slack\footnote{\url{https://slack.com}} chat platform.
Automatic posting of commits and CI results into separate channels was configured to increase overivew of the team's activity.

Github's issue tracker was used to track bugs, enhancements and deviations from the design document.
However, bugs that were fixed as they were encountered were not necessarily tracked in order to reduce time overhead.

